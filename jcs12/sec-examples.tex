Suppose the storage server keeps a user's social security number (SSN)
and the attacker sends the query $\sif{\mathit{SSN} =
  012345678}{\sassign{\mathit{result}}{1}}{\sassign{\mathit{result}}{0}}$.
Suppose the attacker's pre-belief about $\mathit{SSN}$ is a uniform
distribution.  If $\mathit{result}$ is $0$ (i.e., the user's SSN is
not $012345678$) then running the query will not reveal much
information.  But if $\mathit{result}$ is $1$, the running the query
reveals \emph{everything}.  Unfortunately, if we reject the query on
this basis, the attacker knows that the SSN must be $012345678$ since
no other SSN would increase his knowledge above a reasonable
threshold.

\subsection{Example queries}
\label{sec:examples}

Here we consider some example queries that might be used by an
application like Facebook.  
% In our proposed model, Facebook would send
% these queries to an agent that acts on behalf of the user, e.g., a
% storage server that runs on his home network).  The agent examines the
% query and either responds with an answer or declines to do so if it
% believes that an answer could reveal too much information.  
% In the examples below, 
We use capitalized variables to represent enumerations which would
ultimately be compiled to integers; notably we assume $\strue = 1$ and
$\sfalse = 0$.  For brevity, we assume that all local variables are
initialized to 0, missing $\mathsf{else}$ branches are equivalent to
$\sskip$, expressions $E$ when used as boolean expressions $B$
are equivalent to $E = \strue$, and all local variables other than
$\var{output}$ are zeroed out before the query returns.

\begin{example}
\label{ex1}
The following program determines whether to display an ad for a
hypothetical company CM Photographics; it is an example Facebook
gives to potential advertisers.\footnote{See
  \url{http://www.facebook.com/advertising/?campaign_id=402047449186&placement=pf&extra_1=0}}
\begin{displaymath}
\begin{array}{l}
\sassign{\var{age}}{2010 - \var{birth\_year}}; \\
\sifnoelse{\var{age} \leq 30 \wedge \var{age} \geq
  24}{\sassign{\var{age\_sat}}{\strue}};\\
\sifk\; \var{gender} = \sconst{Female}\; \wedge \\
\quad \var{relationship\_status} = \sconst{Engaged} \wedge \var{age\_sat}\; \sthenk \\
\quad \quad \sassign{\var{output}}{\strue};\\
\end{array}
\end{displaymath}
Here, the secret variables (containing a user's private data) are
\var{birth\_year}, \var{gender}, and 
\var{relationship\_status}.  The advertiser is targeting young women,
aged 24 to 30, who are engaged to be married.  If the \var{output}
variable is \strue, the ad will be displayed, otherwise not.
\end{example}

One possible disadvantage of the above query is that it strictly excludes
women outside the given age range.  To provide some opportunity for
serendipity, we might insert the following statement just after the first
$\sifk$ statement above:
\begin{displaymath}
\begin{array}{l}
\spifnoelse{0.1}{\sassign{\var{age\_sat}}{\strue}}; \\
\end{array}
\end{displaymath}
\begin{example}
  Here is another example for a hypothetical company, this time trying
  to sell pizza to college students and/or young people that live
  within a ``box'' around where the restaurant is located.
\begin{displaymath}
\begin{array}{l}
\sifnoelse{\var{in\_school\_type} \geq \sconst{University}}{
  \sassign{\var{in\_school}}{\strue}}; \\
~\\
\sassign{\var{age}}{2010 - \var{birth\_year}}; \\
\sifnoelse{\var{age} \geq 18 \wedge \var{age} \leq 28}{
  \sassign{\var{age\_sat}}{\strue}}; \\
~\\
\sassign{\var{ul\_lat}}{39003178}; \\
\sassign{\var{ul\_long}}{-76958199}; \\
\sassign{\var{lr\_lat}}{38967884}; \\
\sassign{\var{lr\_long}}{-76925926}; \\
\sifk\; \var{address\_lat} \leq \var{ul\_lat}\; \wedge  \\
\quad  \var{address\_lat} \geq \var{lr\_lat}\; \wedge \\
\quad  \var{address\_long} \leq \var{lr\_long}\; \wedge \\
\quad  \var{address\_long} \geq \var{ul\_long}\;\sthenk \\
\quad \quad \sassign{\var{in\_range}}{\strue}; \\
~\\
\sifk\; (\var{in\_school} \vee \var{age\_criteria}) \wedge
\var{in\_range}\; \sthenk \\
\quad \sassign{\var{output}}{\strue};
\end{array}
\end{displaymath}
  Here the secret variables are as with the previous query, along with
  the variables \var{address}* that indicate where the user lives,
  and \var{in\_school\_type} as an enumeration indicating whether the
  user is in school, and at what level.  The restaurant's location is
  given by the \var{ul}* and \var{lr}* variables.
\end{example}
