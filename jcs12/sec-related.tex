\section{Discussion}
\label{sec:future-work}

This section considers the some of the tradeoffs, challenges, design
alternatives, and possibilities for future work on knowledge-based
security policies and probabilistic polyhedra.

\subsection{Knowledge-based policies}

Employing knowledge-based policies successfully requires maintaining a
reasonable estimate of queriers' beliefs.  Two difficulties that arise
in doing so are (1) establishing the initial belief, and (2)
accounting for collusion among queriers.  Here we discuss
these two issues, and suggest further applications of knowledge-based
policies.

\paragraph*{Initial belief.}

For $P_1$ to enforce a knowledge-based policy on queries by $P_2$
requires that $P_1$ estimate $P_2$'s belief about the possible
valuations of $P_1$'s secret data.  When $P_2$ has no particular
knowledge of $P_1$'s secrets, statistical demographic data can be
used; e.g., the US Census and other public sources could be used for
personal information. When $P_2$ might have inside knowledge, $P_1$
must expend more effort to account for it.  For example, in a military
setting, estimating a coalition partner's estimate of one's resources
could be derived from shared information, and from the likelihood of
$P_2$ having acquired illicit information.  
Our benchmarks largely considered personal, private information, e.g.,
gender, age, level of education, etc.  This information can be drawn
from public sources (e.g., Facebook
demographics~\cite{facebook-demographics}).  However, the
distributions we experimented with were oversimplifications of the
actual, reported distributions: they were largely simple, uniform
distributions.  At present, the syntax of our query language
itself only allows specification of a single distribution. 
% Our
% abstraction has the benefit, however, of an approximate
% representation, which can be taken advantage of to easy the burden
% initial belief specification. A representation that attributes maximal
% probability of $ 1 $ to every state is technically sound (though
% not very useful). 
Of course, we could permit users to more directly specify
distributions of interest in a separate syntax.  A more interesting
alternative is to define the initial distribution using a
probabilistic program employing conditioning based on observations,
using the same mechanisms already developed to determining whether to
answer queries.  This is done in languages like IBAL~\cite{pfeffer07ibal} and
Fun~\cite{borgstrom11measure} and could be adapted to our setting.

% Machine learning techniques could be useful for our purposes
% here. Large bodies of work in ML deals with constructing models of
% probability distributions from samples, purported to come from said
% distributions. Recent work \cite{gordon13model} explores how to
% perform model learning using probabilistic language techniques. An
% interesting capability of this work is ability to specify models as
% programs, perform learning of the model given samples, and ascertain
% the model's imprecision via uncertainty in its parameters. Though the
% representations of distributions used there are different than the
% ones we use here, the same techniques could be applied here to learn
% an initial belief as a model from publicly visible samples.

If the initial belief estimate is (very) inaccurate, then $P_1$ risks
releasing information to $P_2$ contrary to his intended policy.  To
``hedge his bets'' against this possibility, he might choose to
maintain a set of possible beliefs $\Delta$, rather than a single
belief $\delta$, and only release if the threshold was satisfied for
every $\delta \in \Delta$.  We return to this idea in
Section~\ref{sec:diffpriv}. In some sense we already do maintain
multiple beliefs, as our implementation models/abstracts powersets of
probabilistic polyhedra, rather than single polyhedra, for performance
reasons.  We leave to future work the exploration of the practical
implementation of this idea.

\paragraph*{Collusion.}

Assuming that $P$ maintains separate beliefs and thresholds for
distinct queriers $Q_1$ and $Q_2$, the possibility of collusion
arises.  Following our motivating example in the Introduction, $P$'s
privacy would be thwarted if he shared only his birth day with $Q_1$
and only his birth year with $Q_2$ but then $Q_1$ and $Q_2$ shared
their information. This problem is not unique to our work; the same
problem would arise if $P$ used an access control policy to protect
birth day and birth year in the same way --- the policy will be
thwarted if $Q_1$ and $Q_2$ pool their information.

A simple approach to preventing this would be to model adversary
knowledge globally, effectively assuming that all queriers share their
query results; doing so would prevent either $Q_1$'s or $Q_2$'s query
(whichever was last). This approach is akin to having a global privacy
budget in differential privacy (see Section~\ref{sec:diffpriv}) or a
single, global access control policy and would obviously harm utility.
Moreover, the confidence of an agent can actually decrease as a result
of observing query outputs. This can occur when the agent is initially
confident in something that is not true, or as a result of improbable
outputs of probabilistic queries (like Example~\ref{ex:specyear}),
even when the agent is not mistakenly confident. Due to this
non-monotonicity in knowledge, if agents purported to be colluding are
in fact not colluding then a global belief might under-approximate a
non-colluding agent's true level of knowledge. \pxm{Second part of
  this para has changed.}

One possible compromise would be to consider both the potential of
collusion and non-collusion by tracking a global belief \emph{and} a
set of individual beliefs. When considering a query, a rejection would
be issued if either belief fails the policy check.

It is important to note that despite the possibility of a decreased
certainty due to queries, rational adversaries, interested in
maximizing their chances of guessing the secret value, will take all
outputs into account, even unlikely outcomes of probabilistic
queries. Though they might be detrimental to certainty, in
expectation, all queries increase chances of guessing the secret
correctly. This point is discussed further in the section on Accuracy,
Uncertainty, and Misinformation of
\cite{clarkson09quantifying}. \pxm{para changed}

\paragraph*{Further applications}

We have used the goal of decentralizing social networking applications
as a motivator for our technique.  But knowledge-based policies have
other applications as well.  Here are four examples.  

The first application is \emph{secure multiparty computation}
(SMC)~\cite{Yao86}.  Such computations allow a set of mutually
distrusting parties to compute a function~$f$ of their private inputs
while revealing nothing about their inputs beyond what is implied by
the result.  Depending on~$f$, however, the result itself may reveal
more information than parties are comfortable with.  Knowledge-based
policies can generalized to this setting: each party $X$ can assess
whether the other parties $Y_i$ could have secret values such that
$f$'s result would exceed a knowledge threshold about $X$'s secret.
We have explored two methods for implementing the generalized
knowledge threshold check; details are elsewhere~\cite{mardziel12smc}.

Another application is to protecting user browsing history.  With the
advent of ``do not track'' guidelines that forbid storing cookies to
track users' browsing habits across sites~\cite{dnt}, service
providers have turned to \emph{fingerprinting}, which aims to identify
a user based on available browser
characteristics~\cite{boda11fingerprint}.  We can protect these
attributes with knowledge-based policies, and enforce them by
analyzing the javascript code on browsed pages.  The flexibility of
knowledge-based policies is useful: with access control we would have
to choose, in advance, which attributes to reveal, but with
knowledge-based policies we can set a threshold on the entire tuple of
the most sensitive attributes and a web page can have access to
whatever (legal) subset of information it likes, for a better user
experience.

A third application is to protect sensing capabilities.  In
particular, we can treat sensor readings as samples from a random
process parameterized by confidential characteristics of the sensor.
Each reading provides information about these parameters, in addition
to information from the reading itself.  For example, suppose we want
to share mobility traces for traffic planning, but want to protect
individuals' privacy.  We can view each trace as a series of samples
from a random process that determines the individual's location based
on periodic factors, like previous location, time of day, day of week,
etc.~\cite{shokri11quantifying}.  We can define the sampling function in our simple
language, involving probabilistic choices over the hidden parameters.
Belief tracking can be used to narrow down the set of possible
parameters to the model that could have produced the observed traces;
if the observer's certainty about these parameters exceeds the
threshold, then the trace elements are not revealed.  Note that trace
obfuscation techniques are easily accounted for---they can simply be
composed with the sampling function and reasoned about together.

Finally, we observe that we can simply track the \emph{amount} of
released information due to an interaction as a degenerate case of
enforcing knowledge-based policies.  In particular, we can set the
pre-belief over some sensitive information to be the uniform
distribution, and set the threshold to be 1.  In this case, we will
always answer any query, and at any time we can compute the entropy of
the current belief estimate to calculate the (maximum) number of bits
leaked.  This information may be used to evaluate the susceptibility
to attack by gauging the confidence an attacker might have in their
belief about secret information.  It can be used to gauge what aspects
of secret information were of interest to the attacker, giving hints
to as their motive, or maybe even differentiating an honest querier
from a malicious one.  Tracking information in this manner is less
expensive than enforcing threshold policies directly, since not all
possible outputs need to be considered, and carries no risk of a
mis-estimate of the pre-belief: the number of reported leaked bits
will be conservatively high.

\subsection{Improving the Performance of Probabilistic Polyhedra}

While the performance of probabilistic polyhedra compares favorably to
alternative approaches, it can nevertheless be improved; this will be
important for applying it to the deployment scenarios listed above.  Here we
present several ideas for improving the implementation that we hope to
explore in future work.

\paragraph*{Handling nominal values.}  

Our probabilistic domains are based on polyhedra, octagons, and
intervals, which are best for analyzing programs with \emph{numeric}
variables, that contain linear conditionals. Most of the variables in
the benchmark programs were of the numeric variety. However, some were
\emph{nominal}, e.g., the variable encoding a user's language in the
travel query, and these are unordered.  Some variables are nominal but
partially ordered, like education level in the pizza query.  While we
can encode nominal values as integers, they may be better handled via
other means, perhaps even via na\"ive enumeration. Handling of large
quantities of nominal values could be performed symbolically using
some of the tools used by other probabilistic languages:
\cite{claret12bayesian}, graphical models \cite{milch05blog}, or
factor graphs \cite{borgstrom11measure, pfeffer07ibal}. Ideally
abstract domains like used in our system and ones better suited for
nominal values, could be integrated (i.e. via reduced product
\cite{cousot79systematic}) to effectively process programs that
contain both types of variables.

\paragraph*{Region splitting.}

As the performance experiments in the previous section show, intervals
can be far more efficient than polyhedra.  While a single interval may
be more imprecise than a single polyhedron, an interesting idea is
consider \emph{splitting} a polyhedron into many intervals, aiming for
the best of both worlds.  The simplest means of implementing this idea
is to modify the handling of the $ \suniformname $ statement for the
powerset domains to result not in one, but several intervals. Though a
single interval is sufficient to exactly represent distributions
produced by $ \suniformname $, it would be insufficient if, later, the
program introduces relations not representable by intervals. 

The challenge is to find the right tradeoff---increasing the number of
intervals will slowly degrade performance and may hurt precision if we
use an unfortunate merging order at join points, as seen in
Figure~\ref{fig:random_box}.  Heuristically picking the right merge
order is a known challenge in abstract interpretation-based analyses.

\paragraph*{Querier belief tracking.} 

Recall a vital property of our knowledge-based policy enforcement is
\emph{simulatability}: the adversary has (or is allowed to have) all
the information necessary to decide the policy that governs its
access, without interacting with the data owner. As such, the
computation of policy enforcement could, in theory, by done by the
querier.  Naturally, they should not be trusted in this regard
completely.  One general direction is to imagine the querier
performing the entire computation, and providing proof of the outcome,
via something like proof-carrying
code~\cite{necula97pcc}. Alternatively, the querier could provide
hints to the data owner to improve the speed of his computation.  For
example, the querier could determine the optimal choices for merge
order, and send a digest of these choices.

\section{Related work} \label{sec:related}
We consider four areas of work related to ours: systems aimed at
protecting access to users' private data; methods for quantifying
information flow from general-purpose programs; methods for
privacy-preserving computation, most notably \emph{differential
  privacy}; and finally approaches to performing general-purpose
probabilistic computation.

\subsection{Privacy enhancing network services}
Several recent proposals have considered alternative service
architectures for better ensuring the privacy of individual
participants.  These systems tend to enforce access control policies.
%
For example, PrPl~\cite{prpl} is a decentralized social networking
infrastructure aimed to permit participants to participate in social
networking without losing ownership of their data.  The system uses
\emph{Personal-Cloud Butler} services to store data and enforce access
control policies.  Persona~\cite{persona} uses a centralized
Facebook-style service, but users can store personal data on
distributed storage servers that use attribute-based encryption.
Access to data is granted to those parties who have the necessary
attribute keys.  XBook~\cite{xbook} mediates social network data
accesses by third-party application extensions.  Privad~\cite{privad}
is a privacy-preserving advertising platform that, like our running
examples, runs the ad selection algorithm over the user's data on the
user's platform.  Privad \emph{presumes} that outputs of ad selection
reveal little about the inputs.

Knowledge-based security policies generalize access control policies:
if we maintain a belief estimate for each principal $P$, then the
equivalent of granting $P$ access to data $d$ is to just set $P$'s
knowledge threshold for $d$ to 1; for principals $R$ who should not
have access, the threshold for $d$ is 0.  Knowledge-based policies
afford greater flexibility by allowing \emph{partial} access, i.e.,
when a threshold less than 1.  Moreover, as mentioned in the
introduction, we can set a policy on multiple data items, and thus
grant more access to one item than another (e.g., birth day and/or
year) as long as knowledge of the aggregate is controlled.

\subsection{Quantified information flow}
Others have considered how an adversary's knowledge of private data
might be informed by a program's output.  Clark, Hunt, and
Malacaria~\cite{clark2005QIF} define a static analysis that bounds the
secret information a straight-line program can leak in terms of
equivalence relations between the inputs and outputs.  Backes et
al.~\cite{backes09automatic} automate the synthesis of such
equivalence relations and quantify leakage by computing the exact size
of equivalence classes.  K\"opf and
Rybalchenko~\cite{kopf:rybalchenko} extend this approach, improving
its scalability by using sampling to identify equivalence classes and
using under- and over-approximation to obtain bounds on their size.
% \sbmcomment{Here we imply that the size of these blocks are important,
% but then Boris said that it is only the number of blocks that is of interest.
% Could we clarify this?} \mwh{They worry about class sizes to compute
% shannon entropy.  It's mentioned here because we care about sizes
% too.  I think we can just leave it given clarifications I make below.}
Mu and Clark~\cite{Mu:2009:inverval-qif} present a similar analysis
that uses over-approximation only.  In all cases, the inferred
equivalence classes can be used to compute entropy-based metrics of
information leakage.

We differ from this work in two main ways.  First, we implement a
different security criterion.  The most closely related metric is
\emph{conditional vulnerability} $V$ as proposed by
Smith~\cite{smith09foundations}, which can be defined using our
notation as follows:%\footnote{Smith actually proposes \emph{min
%    entropy}, which is $-\mathit{log}\;V$.}
\begin{definition}
\label{def:vul-threshold}
Let $\delta' = \eval{S}{\delta}$, where $\delta$ is the model of the
querier's initial belief.
%, and let $\overline{\delta_X} \defeq
%\normal{\project{\delta}{X}}$.

Then query $S$ is \emph{vulnerability threshold secure} iff for
%$$V = \sum_{\sigma_L \in
%  \nzset{\delta'_L}}\; \overline{\delta'_L}(\sigma_L) \cdot \max_{\sigma_H \in \states_H}\;
%  \overline{(\dcond{\delta'}{\sigma_L})_H}(\sigma_H)$$
$$V = \sum_{\sigma_L \in \support{\project{\delta'}{L}}}\;
  (\project{\delta'}{L})(\sigma_L) \cdot \max_{\sigma_H \in \states_H}\;
  (\project{\drevise{\delta'}{\sigma_L}}{H})(\sigma_H)$$
  we have $V \leq t$ for some threshold $t$.
\end{definition}
% $Pr[H=h] = (\project{\delta}{H})(h)$
% $Pr[H=h|L=l] = (\project{\dcond{\pevalp{S}{\delta}}{(L=l)}}{H})(h)$
% $Pr[L=l] = \dcond{\pevalp{S}{\delta}}{(L=l)}$
The above definition is an \emph{expectation} over all possible
outputs $\sigma_L$, so unlikely outputs have less influence.  Our
notion of threshold security (Definition~\ref{def:threshold}), in
contrast, is equivalent to a bound on the following quantity:
$$V^* = \max_{\sigma_L \in \support{\project{\delta'}{L}}}\;
\max_{\sigma_H \in \states_H}\;
(\project{\drevise{\delta'}{\sigma_L}}{H})(\sigma_H)$$

This notion of security is strictly stronger as it considers each
output individually: if \emph{any} output, however unlikely, would
increase knowledge beyond the threshold, the query would be rejected.
For example, recall the query from Example~\ref{ex:bday} where the
secret data $\var{bday}$ is (assumed by the querier to be) uniformly
distributed; call this query $S_1$.  According to
Definition~\ref{def:vul-threshold}, the minimum acceptable threshold
for which the query would be considered safe is $ V = \frac{7}{365} *
\frac{1}{7} + \frac{358}{365} * \frac{1}{358} = 2/365 \approx 0.005$,
whereas according to Definition~\ref{def:threshold}, the minimum
threshold is $ V^* = 1/7 \approx 0.143$.

The idea of strengthening of an entropy measure by eliminating the
expectation has been briefly considered by K{\"o}pf and Basin
\cite{koepfbasin07}. In their work this measure is proposed as a
stronger alternative, a choice ultimately dependent on the
application. In our case, however, the worst-case measure is
absolutely necessary in order to prevent the leakage of information
when rejecting a query.

The other distinguishing feature of our approach is that we keep an on-line model of
adversary knowledge according to prior, actual query results. Once a
query is answered, the alternative possible outputs of this query no
longer matter. To see the benefit of this
query-by-query approach, consider performing query $S_1$ followed by a
query $S_2$ which uses the same code as $S_1$ (from Example~\ref{ex:bday}) but has
$\var{today} = 265$.  With our system and $\var{bday} = 270$ the
answer to $S_1$ is $\sfalse$ and with the revised belief the query
$S_2$ will be accepted as below threshold $t_d = 0.2$.  If instead we
had to model this pair of queries statically they would be rejected
because (under the assumption of uniformity) the pair of outputs
$\strue$,$\strue$ is possible and implies $\var{bday} \in \{ 265, 266
\}$ which would require $t_d \geq 0.5$.  Our approach also inherits
from the belief-based approach the ability to model a querier who is
misinformed or incorrect, which can arise following the result of a
probabilistic query or because of a change to the secret data between
queries~\cite{clarkson09quantifying}.
% Probabilistic queries are not
% handled by the previously-described existing work that statically
% computes conditional min-entropy.  
We note these advantages come at the cost
of maintaining on-line belief models.

Our proposed abstract domains are useful beyond the application of
belief-based threshold security; e.g., they could be used to model
uncertainty off-line (as in the above work) rather than beliefs
on-line, with the advantage that they are not limited to uniform
distributions (as required
by~\cite{backes09automatic,kopf:rybalchenko}).

McCamant and Ernst's \flowcheck{} tool~\cite{McCamantE2008} measures
the information released by a particular execution.  However, it
measures information release in terms of \emph{channel capacity},
rather than remaining uncertainty which is more appropriate for our
setting.  For example, \flowcheck{} would report a query that tries to
guess a user's birthday leaks one bit regardless of whether the guess
was successful, whereas the belief-based model (and the other models
mentioned above) would consider a failing guess to convey very little
information (much less than a bit), and a successful guess conveying
quite a lot (much more than a bit).  

\subsection{Differential privacy}
\label{sec:diffpriv}

A recent thread of research has aimed to enforce the privacy of
database queries.  Most notably, Dwork and colleagues have proposed
\emph{differential privacy}~\cite{diffpriv}: a differentially private
query $Q$ over a database of individuals' records is a randomized
function that produces roughly the same answer whether a particular
individual's data is in the database or not.  An appealing feature of
this definition is that the querier's knowledge is not considered
directly; rather, we are merely assured that $Q$'s answer will not
differ by much whether a particular individual is in the database or
not. On the other hand, differentially private databases require that
individuals trust the database curator, a situation we would prefer to
avoid, as motivated in the introduction.

We can compare differential privacy to our notion of threshold
security by recasting the differential privacy definition into the
form of an \emph{adversarial privacy} definition, which was formulated
by Rastogi and Suciu to compare their own privacy definition on
databases to differential privacy~\cite{rastogi09relationship}.
Rastogi and Suciu's definition is in terms of the presence or
absence of a record in a database, whereas our notion is defined on
secret states over variables in $H$.  To bridge this gap suppose that,
without loss of generality, variables $x \in H$ range over $\{ 0, 1
\}$, and we say a variable $x$ is ``in a state'' $\sigma$ when
$\sigma(x) = 1$. (We can always encode an integer $i$ as a sequence of
bits.)  Define $\delta(\{x_1,...,x_n\})$ to be the probability of all
states $\sigma$ such that $\sigma(x_i) = 1$ for all $1 \leq i \leq n$;
i.e.,
$$\delta(\{x_1,...,x_n\}) \defeq 
\sum_{\sigma \mid
  \sigma(x_1) = 1 \wedge ... \wedge \sigma(x_n) = 1} \delta(\sigma) 
$$
Then we can state adversarial privacy as follows.

\begin{definition}[Adversarial Privacy]
\label{def:adv-privacy}
Given a query statement $\stmt$, let $O \subseteq \states_{L\cup H}$
be the set of all possible output states of $\stmt$ (over some set of
secret variables $H$ and public variables $L$), and $\Delta_H \subseteq
\dists_H$ be a set of distributions over secret states.
%
We say that program $\stmt$ is $\varepsilon$-\emph{adversarially
  private} w.r.t. $\Delta_H$ iff for all adversary beliefs $\delta_H
\in \Delta_H$, all secret variables $x \in H$, and all possible
outputs $\sigma_o \in O$, that $\delta_H(\{x\}) \leq e^{\varepsilon}\,
\delta'_H(\{x\})$.  Here, $\delta'_H =
\project{(\evalp{\stmt}{(\delta_H \times \dot{\sigma}_L)}) |
  \sigma_o}{H}$; that is, it is the revised adversary belief after
seeing that public output state of $\stmt$ is $\sigma_o$.
\end{definition}

This definition says that query $\stmt$ is acceptable when the output
of $S$ increases only by a small multiplicative factor an adversary's
certainty that some variable $x$ is in the secret input state, for all
adversaries whose prebeliefs are characterized by $\Delta_H$.  Rastogi
and Suciu show that defining $\Delta_H$ to be the class of
\emph{planar, total sub-modular} (or $PTLM$) distributions renders an
adversarial privacy definition equivalent to differential privacy.
Among other conditions, $PTLM$ distributions require fixed-sized
databases (which comes by construction in our setting), and require
the probability of one tuple's presence to not be positively
correlated with another's.  We can transfer this latter condition to
our setting by requiring that for all $\delta \in \Delta_H$, and all
$x_1,x_2 \in H$, we have $\delta(\{x_1\}) \delta(\{x_2\}) \leq
\delta(\{x_1,x_2\})$.

Comparing Definition~\ref{def:adv-privacy} to
Definition~\ref{def:threshold} we can see that the former makes a
\emph{relative} assertion about the increased certainty of any one of
a \emph{set} of possible adversarial beliefs, while the latter makes
an \emph{absolute} assertion about the certainty of a \emph{single}
adversarial belief.  The absolute threshold of our definition is
appealing, as is the more flexible prebelief, which allows positive
correlations among state variables.  On the other hand, our definition
assumes we have correctly modeled the adversary's belief.  While this
is possible in some cases, e.g., when demographic information is
available, differential privacy is appealing in that it is robust a
much larger number of adversaries, i.e., all those whose prebeliefs
are in $PTLM$.

Unfortunately, this strong privacy guarantee seems to come at strong
cost to utility.  For example, deterministic queries are effectively
precluded, eliminating some possibly useful applications.  
For the application of social
networks, Machanavajjhala et al.~\cite{machanavajjhala11personal} have
shown that good private social recommendations are feasible only for a
small subset of the users in the social network or for a lenient
setting of privacy parameters.  We can see the utility loss in our
motivating scenario as well.  Consider the birthday query from
Example~\ref{ex:bday}.  Bob's birthday being/not being in the query
range influences the output of the query only by 1 (assuming yes/no is
1/0). One could add an appropriate amount of (Laplacian) noise to the
query answer to hide what the true answer was and make the query
differentially private. However, this noise would be so large compared
to the original range $\{0,1\}$ that the query becomes essentially
useless---the user would be receiving a birthday announcement most
days.\footnote{By our calculations, with privacy parameter $\varepsilon =
  0.1$ recommended by Dwork~\cite{diffpriv}, the probability the query
  returns the correct result is approximately $0.5249$.}  

By contrast, our approach permits answering (deterministic or
randomized) queries exactly if the release of information is below the
threshold.  Moreover, by tracking beliefs between queries, there is no
artificial limit on the number of queries since we can track the
released information exactly.  Differential privacy imposes an artificial
limit on the number of queries because the post-query, revised beliefs
of the adversary are not computed.  Rather, each query conservatively
adds up to $\varepsilon$ to an adversary's certainty about a tuple,
and since the precise release is not computed, we must assume the
worst.  Eventually, performing further queries will pose too great of a risk.

As mentioned in the previous section, we could gain some of the
privacy advantages of differential privacy by modeling a (larger) set
of beliefs rather than a single one.  Our abstractions $\ppolys$ and
$\ppowers$ already model sets of distributions, rather than a single
distribution, so it remains interesting future work to exploit this
representation toward increasing privacy.

\subsection{Probabilistic programming}
The core of our methodology relies on probabilistic computation. A
variety of tools exist for specifying random processes as computer
programs and performing inference on them.

Implementations based on partial sampling \cite{goodman08church,
  park08sampling} or full enumeration \cite{radul07probscheme} of the
state space are unsuitable in our setting. Such tools are either too
inefficient or too imprecise. Works based on smarter representations
of probability distributions are promising alternatives. Projects
based on algebraic decision diagrams \cite{claret12bayesian},
graphical models \cite{milch05blog}, and factor graphs
\cite{borgstrom11measure, pfeffer07ibal} translate programs into
convenient structures and take advantage of efficient algorithms for
their manipulation or inference.

Our implementation for probabilistic computation and inference differs
from existing works in two main ways. Firstly, we are capable of
approximation and hence can trade off precision for performance, while
maintaining soundness in terms of a strong security policy. The second
difference is the nature of our representation of probability
distributions. Our work is based on numerical abstractions: intervals,
octagons, and polyhedra. These abstractions are especially well suited
for analysis of imperative programs with numeric variables and linear
conditionals. Other probabilistic languages
might serve as better choices when nominal, rather than numeric,
variables are used in queries. The comparative study of the power and effectiveness
of various representations in probabilistic computation is a topic of
our ongoing research.

We are not the first to propose probabilistic abstract
interpretation. Monniaux~\cite{Monniaux_these} gives an abstract
interpretation for probabilistic programs based on over-approximating
probabilities of points. Di Pierro describes abstract interpretation
for probabilistic lambda calculus
\cite{dipierro05probabilistic}. Smith~\cite{smith08probabilistic}
describes probabilistic abstract interpretation for verification of
quantitative program properties. Cousot~\cite{cousot12probabilistic}
unifies these and other probabilistic program analysis tools. However,
these do not deal with sound distribution conditioning, the unique
element of our approach, which is crucial for belief-based information
flow analysis.

% Following the flow of our example from Section~\ref{sec:overview},
% when advertiser $X$ submits Example~\ref{ex:specyear}, a result of
% $\strue$ will lead $X$ to believe that $1980$ (Bob's actual birthday)
% is a less likely birth year ($1/11814 \times 268 = 0.023$) than he
% previously thought ($1/37 = 0.027$).  Suppose we are using $X$'s
% modeled belief (call it $\delta_X$) to also model $Y$'s knowledge,
% where $Y$ would otherwise give $1980$ has probability $1/37$ (i.e., if
% we were modeling $Y$'s knowledge separately as $\delta_Y$).  If Bob's
% threshold for his birth year is $t_{yr} = 0.03$, then under $\delta_X$
% a query \mwh{need to choose a query in which we'd reject under
%   $\delta_Y$ but not under $\delta_X$}.

% One way to address this problem is to allow only deterministic
% queries.  Then there is no misinformation.  Obviously doing this
% reduces the system's expressive power.  Another possibility is to
% force knowledge to be monotonic: if an accepted query Q would cause
% the posterior certainty to be reduced, we do not change the modeled
% belief, but leave it as is.  I haven't thought carefully about the
% ramifications of doing so, though.  I suspect the result is simply
% conservative, but I need to think harder about whether it's unsafe.

% So, in short, I think that DP has problems with collusion, but these
% problems are entirely in terms of utility, not performance or
% safety.  That is, DP, can have a global privacy budget and while
% using such a mechanism will severely limit utility, it will not harm
% privacy.  In our case, we could model many belief sets to avoid the
% safety problems, but at an enormous performance cost, and at some
% cost to utility (the same flavor as DP, i.e., if principals were not
% really colluding).  Or, we could ignore the possibility of collusion
% to regain performance and utility, but impose some risk on safety.

% But there is a key difference to your approach: [25,26] use *conditional* min-entropy as a measure, which is (in some sense) the expected min-entropy over all possible choices of secrets. An iterative application (and summing conditional min-entropy, as suggested above) implicitly assumes that the secret is freshly chosen in each run. While this is a sound over-approximation of what the adversary may learn, its knowledge will never level off (as it may do in your approach). This is a clear advantage of the belief-based approach.

%%In any case, we believe there is much to be said for both approaches, and hope that this paper serves as another useful step on the path toward practical privacy-preserving computation.


%%For example, if a client
%%wants to determine what proportion of Facebook users are male, this
%%query will be permitted, as the removal of a particular user's data
%%will not significantly affect the result.  However in the course of
%%computing this result, substantial access to individual user data is
%%required.   In the absence of a system for secure multi-party
%%computation \cite{fairplaymp}, the agent computing the result must
%%have full access to gender information for each user, so that the
%%number of male users and female users can be determined.  Such queries
%%would leak too much individual information.

%%Thus, it would appear that we need a trusted third party to execute
%5these queries.  However, in our approach one can use the probabilistic conditional
%%\textsf{pif} to permit certain types of data aggregation.  For our
%%Facebook gender example, an estimate of the proportion of users who
%%are male can be obtained by executing the query below.  We assume that
%%gender is encoded as a binary value, where 0 represents ``male.''

%%\[\spif{0.6}{output := gender}{output := not(gender)}\]

%% This gives the client imperfect information about each user, but this
%5information is still useful in aggregate. For 60\% of the queries, the
%%client will get the correct gender.  If all users are male, the client would
%%expect 60\% of the queries to return 0.  If all users are female, the client
%%would expect 40\% of the queries to return 0.  The percentage in between
%%correspond to varying percentages of male users, with the crossover point
%%being 50\%---if more than 50\% of the queries return 0, then there are likely
%%to be more males than females (where the likelihood depends on the number
%%of users that are sampled).

%%Second, our approach investigates a knowledge-based policy that succinctly captures the semantics of the information that has thus far been revealed to a querier via one or more queries. While differential privacy allows the queried entity to cumulatively track loss in privacy over one or more queries~\cite{PINQ-SIGMOD09}, it fails to consider semantic overlap between such queries $-$ thereby weakening the tightness of their estimates.


%% to pass the individual privacy
%%monitoring \sbmcomment{better term for this?} system described
%%here.



%% Recent advances have explored the notion of adversarial privacy~\cite{rastogi09relationship} to partially bridge this semantic gap by explicitly modeling the adversary's belief on the input dataset. In their setting privacy is defined by the closeness between the posterior and prior distribution on the likelihood of a tuple after seeing the output of query. Their approach is shown to offer tighter estimates by proving that adversarial privacy under a bounded belief model subsumes differential privacy. While their approach quantitatively estimates privacy loss wrt to the querier's belief (as against differential privacy that is agnostic to the querier's belief), they fail to capture the semantics of the information released via one or more queries. Further, similar to differential privacy, adversarial privacy is designed to operate on aggregate data from multiple individuals.

%%Differential privacy~\cite{diffpriv}: works on aggregate data, not on
%%individual data; fixed number of possible queries allowed until the
%%data must be destroyed (cannot even answer the same query twice).  By
%%contrast, our approach models semantic knowledge, so you can ask the
%%same question in different ways and still get an answer.

%%Also want to talk about Rastogi and Suciu's stuff.

%%Backes et al.~\cite{backes09automatic} model information leakage in a program as (the size of) an equivalence relationship between the set of input and output symbols. While their approach quantifies per-program entropy measures, they also fail to offer fine-grained (per query) semantically richer approach to track information release.

%%\mwh{More rambling below}

%%Diffpriv different in that they don't model belief explicitly, but they do have a privacy ``budget'' $\epsilon$.  In effect this is a counter of the amount of knowledge they are willing to let an attacker gain.  While they are not required to model a belief explicitly, which is an advantage, they need to have the assumed level of knowledge in mind in order to set $\epsilon$.  A disadvantage of the indirect modeling of knowledge is that it is difficult to reason how much a query $q$ overlaps with a prior query $q'$.  As such, the default is to presume all queries reveal (equivalent) knowledge, and so there is a fixed number of queries a particular principal may pose.  Also, equally vulnerable to collusion as our approach.  I wonder: how does diffpriv handle changing values?  Work on declassification/anonymization: picks values to release in advance, so less flexibility for the querier.  In scenarios like SMC, this could be a showstopper.  But this is simpler to implement and resistant to collusion.  Could imagine combining pro-active anonymization with belief-based policies.

%%\mwh{Taken from introduction, to work in here}:


%%Returning exact results is in contrast to approaches like differential privacy~\cite{diffpriv}
%%that, while also limiting which queries may be performed, must add
%%noise to increase uncertainty, since knowledge is not modeled
%%directly.

%%\todo{Mention limitations somewhere: what about collusion?
%%  What if you get the belief models wrong?  How to deal with
%%  non-discrete data, or fixed-length loops?}

%%\mwh{Work the following into the discussion of differential privacy?}



%%\mwh{and more ...}

%%Differential privacy guarantees that privacy is preserved by ensuring
%%that sufficient data aggregation occurs.  

%%\sbmcomment{Would be good to make this more formal.  What are the equations
%%that yield the likely percentage of males and the confidence in this estimate?}

