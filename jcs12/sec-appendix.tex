%\appendices
\appendix

%\vspace*{.1in}
%\section{Concrete probabilistic semantics}
%\label{appendix:concrete}
%
%Here we briefly explain the concrete probabilistic semantics given in
%Figure~\ref{fig-sem-nondet2-core}.  More details can be found in
%Clarkson et al.~\cite{clarkson09quantifying}.
%
%The semantics of $\sskip$ is straightforward: it is the identity on
%distributions.  The semantics of sequences $\sseq{\stmt_1}{\stmt_2}$
%is also straightforward: the distribution that results from executing
%$\stmt_1$ with $\delta$ is given as input to $\stmt_2$ to produce
%the result.
%
%The semantics of assignment is $\delta \bparen{x \ra \aexp}$, which is
%defined as follows: 
%$$ \delta \bparen{x \ra \aexp} \defeq \lambda \sigma \lsep \sum_{\tau \; | \; \tau
%  \bparen{x \ra \eeval{\aexp}{\tau}} = \sigma} \delta (\tau) $$ In
%words, the result of substituting an expression $\aexp$ for $x$ is a
%distribution where state $\sigma$ is given a probability that is the
%sum of the probabilities of all states $\tau$ that are equal to
%$\sigma$ when $x$ is mapped to the distribution on $\aexp$ in $\tau$.
%For implementation purposes, it will be useful to consider separately the
%case where assignment is invertible.
%
%When $x \ra \aexp$ is an invertible transformation, the formula for
%assignment can be simplified to the following, where $x \ra \aexp'$ is
%the inverse of $x \ra \aexp$.
%\[
%\delta \bparen{x \ra \aexp} \defeq \lambda \sigma \lsep \delta (\sigma\bparen{x \ra \eeval{\aexp'}{\sigma}})
%\]
%
%When $x \ra \aexp$ is not invertible, the original definition is
%equivalent to a projection followed by an assignment.  Let $V'
%= \fv{\delta} - \{x\}$ and let $\delta' = \project{\delta}{V'}$.
%Then we have the following for a non-invertible assignment.
%\[\delta \bparen{x \ra \aexp} \defeq \lambda \sigma \lsep \aif \sigma(x) = \eeval{E}{\sigma}\athen
%\delta'(\project{\sigma}{V'}) \aelse 0 \]
%We prove this definition by cases is
%equivalent to the original definition in the companion technical report~\cite{TR}.
%
%The semantics for conditionals makes use of two operators on
%distributions which we now define.  First, given distributions
%$\delta_1$ and $\delta_2$ we define the \emph{distribution sum} as
%follows:
%$$ \delta_1 + \delta_2 \defeq \lambda \sigma \lsep \delta_1(\sigma) +
%\delta_2(\sigma) $$ In words, the probability mass for a given state
%$\sigma$ of the summed distribution is just the sum of the masses from
%the input distributions for $\sigma$.  Second, given a distribution
%$\delta$ and a boolean expression $\bexp$, we define the
%\emph{distribution conditioned on $\bexp$} to be
%$$ \dcond{\delta}{\bexp} \defeq \lambda \sigma \lsep \aif \eeval{\bexp}{\sigma} \athen
%\delta(\sigma) \aelse 0 $$
%In short, the resulting distribution retains only the probability mass
%from $\delta$ for states $\sigma$ in which $\bexp$
%holds.
%
%With these two operators, the semantics of conditionals can be stated
%simply: the resulting distribution is the sum of the distributions of
%the two branches, where the first branch's distribution is conditioned
%on $\bexp$ being true, while the second branch's distribution is
%conditioned on $\bexp$ being false.
%
%The semantics for probabilistic conditionals like that of conditionals
%but makes use of \emph{distribution scaling}, which is defined as
%follows: given $\delta$ and some scalar $p$ in $[0,1]$, we have
%$$ p \cdot \delta \defeq \lambda \sigma \lsep p \cdot \delta(\sigma) $$
%In short, the probability ascribed to each state is just the
%probability ascribed to that state by $\delta$ but multiplied by $p$.
%For probabilistic conditionals, we sum the distributions of the two
%branches, scaling them according to the odds $q$ and $1 - q$.
%
%The semantics of a single iteration of a while loop is
%essentially that of $\sif{B}{S}{\sskip}$ and the semantics of the
%entire loop is the fixed point of a function that composes the
%distributions produced by each iteration.  That such a fixed point exists
%is proved by Clarkson et al.~\cite{clarkson09quantifying}.
%
%Finally, the semantics of $\suniform{x}{n_1}{n_2}$, introduced in
%Section~\ref{sec:absinterp} is given as
%$$
%\begin{array}{rcl}
%\pevalp{\suniform{x}{n_1}{n_2}}{\delta} & = &
%\paren{\project{\delta}{V - \set{x}}} \times \delta'
%\end{array}
%$$
%Where $ V $ is the set of variables of $ \delta $, and $ \delta' $ is
%defined as follows.
%$$ \delta ' = \lambda \sigma \lsep \aif n_1 \leq \sigma(x) \leq n_2 \athen
%\frac{1}{n_2-n_1+1} \aelse 0 $$
%
\vspace*{.1in}
\section{Example queries} \label{appendix:queries}

%rendering latex
$$ \ssecretk: $$
\begin{displaymath}{\small
\begin{array}{l}
  \sassign{country}{1}\; ;\\
  \sassign{birth\text{\_}year}{1983}\; ;\\
  \sassign{completed\text{\_}school\text{\_}type}{4}\; ;\\
  \sassign{language}{5}\\
\end{array}
}
\end{displaymath}
$$ \sbeliefk: $$
\begin{displaymath}{\small
\begin{array}{l}
  \suniform{country}{1}{200}\; ;\\
  \suniform{birth\text{\_}year}{1900}{2011}\; ;\\
  \suniform{language}{1}{50}\; ;\\
  \suniform{completed\text{\_}school\text{\_}type}{0}{5}\\
\end{array}
}
\end{displaymath}

$$ \squerydefk \; travel \; : \; \; \ra \; output$$
\begin{displaymath}{\small
\begin{array}{l}
  \sifk \; country \; = \; 1 \; \vee \; country \; = \; 3 \; \vee \; country \; = \; 8 \; \vee \; country \; = \; 10 \; \vee \; country \; = \; 18 \; \sthenk\\
  \;\;\;\sassign{main\text{\_}country}{1}\\
  \selsek\\
  \;\;\;\sassign{main\text{\_}country}{0}\; ;\\
  \sifk \; country \; = \; 169 \; \vee \; country \; = \; 197 \; \vee \; country \; = \; 194 \; \vee \; country \; = \; 170 \; \vee \; country \; = \; 206 \; \vee \; country \; = \; 183 \; \vee \; country \; = \; 188 \; \sthenk\\
  \;\;\;\sassign{island}{1}\\
  \selsek\\
  \;\;\;\sassign{island}{0}\; ;\\
  \sassign{age}{\ebinop{-}{2010}{birth\text{\_}year}}\; ;\\
  \sifk \; language \; = \; 1 \; \wedge \; main\text{\_}country \; = \; 1 \; \vee \; island \; = \; 1 \; \wedge \; age \; \geq \; 21 \; \wedge \; completed\text{\_}school\text{\_}type \; \geq \; 4 \; \sthenk\\
  \;\;\;\sassign{output}{1}\\
  \selsek\\
  \;\;\;\sassign{output}{0}\\
\end{array}
}
\end{displaymath}


$$ \squeryk \; travel \; : $$
\begin{displaymath}{\small
\begin{array}{l}
  \sskip\\
\end{array}
}
\end{displaymath}

no errors


We provide here the queries and prebeliefs we used for the experiments
in Section~\ref{sec:impl}. The queries are described as functions
from some set of inputs to some set of outputs. The exact syntax is as
follows.
$$
\begin{array}{l}
\squerydefk \; queryname \; in_1 \cdots in_n \ra out_1 \cdots out_m \; : \\
\;\; querybody \\
\end{array}
$$

Query definitions that follow sometimes include
pre-processing statements of the form:
$$ \psdefine{x}{exp} $$
Such statements result in any occurrence of a variable $ x $ being
replaced by the expression $exp$. This is used for convenience to
refer to common expressions without actually requiring the analysis to
track additional variables/dimensions.

To specify a query invocation we use the following syntax.
$$
\begin{array}{l}
\squeryk \; queryname \; : \\
\;\; in_1 \; := \; val_1; \\
\;\; \cdots \\
\;\; in_n \; := \; val_n
\end{array}
$$
Each experiment must also specify the values of the secrets being
queried, and the querier's prebelief.  Each specification is a merely
a program that sets the values of these variables.  For the actual
secret values this program begins with the declaration $ \ssecretk$;
the resulting state of executing program is taken to be the secret
state.  The program to set the prebelief begins $ \sbeliefk$ and has
the same format; note that this program will use $\spifk$ or
$\suniform{x}{n_1}{n_2}$ to give secrets different possible values
with different probabilities.  

We now give the content of the queries used in the experiments.

\subsection{Birthday}
For the small stateset size birthday experiments we used the
following secret and prebelief.
$$ \ssecretk: $$
\begin{displaymath}{\small
\begin{array}{l}
  \sassign{s\text{\_}bday}{270}\; ;\\
  \sassign{s\text{\_}byear}{1980}\\
\end{array}
}\end{displaymath}
$$ \sbeliefk: $$
\begin{displaymath}{\small
\begin{array}{l}
  \suniform{s\text{\_}bday}{0}{364}\; ;\\
  \suniform{s\text{\_}byear}{1956}{1992}\\
\end{array}
}\end{displaymath}

The two queries used were as follows.

$$ \squerydefk \; bday \; : \;c\text{\_}day \; \ra \; out$$
\begin{displaymath}{\small
\begin{array}{l}
  \sifk \; \ebinop{\wedge}{\ebinop{\geq}{s\text{\_}bday}{c\text{\_}day}}{\ebinop{>}{\ebinop{+}{c\text{\_}day}{7}}{s\text{\_}bday}} \; \sthenk\\
  \;\;\;\sassign{out}{1}\\
  \selsek\\
  \;\;\;\sassign{out}{0}\\
\end{array}
}\end{displaymath}

$$ \squerydefk \; spec \; : \;c\text{\_}year \; \ra \; out$$
\begin{displaymath}{\small
\begin{array}{l}
  \psdefine{age}{\ebinop{-}{c\text{\_}year}{s\text{\_}byear}}\\
  \sifk \; \lbinop{\vee}{\lbinop{\vee}{\lbinop{\vee}{\lbinop{\vee}{\lbinop{\vee}{\lbinop{\vee}{\lbinop{\vee}{\lbinop{\vee}{\lbinop{\vee}{\lreln{=}{age}{10}}{\lreln{=}{age}{20}}}{\lreln{=}{age}{30}}}{\lreln{=}{age}{40}}\\\;}{\lreln{=}{age}{50}}}{\lreln{=}{age}{60}}}{\lreln{=}{age}{70}}}{\lreln{=}{age}{80}}\\\;}{\lreln{=}{age}{90}}}{\lreln{=}{age}{100}}\; \sthenk\\
  \;\;\;\sassign{out}{1}\\
  \selsek\\
  \;\;\;\sassign{out}{0}\; ;\\
  \spifk\; 1/10 \; \sthenk\\
  \;\;\;\sassign{out}{1}\\
\end{array}
}
\end{displaymath}

The statistics described in comparison to the enumeration approach
(Section~\ref{sec:enum}) include the time spent processing this
initial setup as well as time processing one birthday
query. Figure~\ref{fig:bench_domains_bday_large} benchmarks two bday
queries followed by a spec year query.

\begin{itemize}
\item{} A single bday query alone.
$$ \squeryk \; bday \; : $$
\begin{displaymath}{\small
\begin{array}{l}
  \sassign{c\text{\_}day}{260}\\
\end{array}
}\end{displaymath}

\item{} Two bday queries followed by a spec query.
$$ \squeryk \; bday \; : $$
\begin{displaymath}{\small
\begin{array}{l}
  \sassign{c\text{\_}day}{260}\\
\end{array}
}\end{displaymath}
$$ \squeryk \; bday \; : $$
\begin{displaymath}{\small
\begin{array}{l}
  \sassign{c\text{\_}day}{261}\\
\end{array}
}\end{displaymath}
$$ \squeryk \; spec \; : $$
\begin{displaymath}{\small
\begin{array}{l}
  \sassign{c\text{\_}year}{2011}\\
\end{array}
}\end{displaymath}

\end{itemize}

\subsection{Birthday (large)}

For the larger statespace birthday example we used the following
secret and prebelief generators.
$$ \ssecretk: $$
\begin{displaymath}{\small
\begin{array}{l}
  \sassign{s\text{\_}bday}{270}\; ;\\
  \sassign{s\text{\_}byear}{1980}\\
\end{array}
}\end{displaymath}
$$ \sbeliefk: $$
\begin{displaymath}{\small
\begin{array}{l}
  \suniform{s\text{\_}bday}{0}{364}\; ;\\
  \suniform{s\text{\_}byear}{1910}{2010}\\
\end{array}
}\end{displaymath} The queries used were identical to the ones for the
smaller statespace birthday example. For our benchmarks we analyzed
the vulnerability of the pair of secrets $ s\text{\_}bday,
s\text{\_}byear $.

\subsection{Pizza}

The pizza example is slightly more complicated, especially in the
construction of the prebelief.  This example models a targeted
Facebook advertisement for a local pizza shop.  There are four
relevant secret values.  The level of school currently being attended
by the Facebook user is given by \verb|s_in_school_type|, which is an
integer ranging from 0 (not in school) to 6 (Ph.D. program).  Birth
year is as before and \verb|s_address_lat| and \verb|s_address_long|
give the latitude and longitude of the user's home address
(represented as decimal degrees scaled by a factor of $10^6$ and
converted to an integer).

The initial belief models the fact that each subsequent level of
education is less likely and also captures the correlation between
current educational level and age.  For example, a user is given an
approximately 0.05 chance of currently being an undergraduate in
college, and college attendees are assumed to be born no later than
1985 (whereas elementary school students may be born as late as 2002).

$$ \ssecretk: $$
\begin{displaymath}{\small
\begin{array}{l}
  \sassign{s\text{\_}in\text{\_}school\text{\_}type}{4}\; ;\\
  \sassign{s\text{\_}birth\text{\_}year}{1983}\; ;\\
  \sassign{s\text{\_}address\text{\_}lat}{39003178}\; ;\\
  \sassign{s\text{\_}address\text{\_}long}{-76958199}\\
\end{array}
}
\end{displaymath}
$$ \sbeliefk: $$
\begin{displaymath}{\small
\begin{array}{l}
  \spifk\; 4/24 \; \sthenk\\
  \;\;\;\suniform{s\text{\_}in\text{\_}school\text{\_}type}{1}{1}\; ;\\
  \;\;\;\suniform{s\text{\_}birth\text{\_}year}{1998}{2002}\\
  \selsek\\
  \;\;\;\spifk\; 3/19 \; \sthenk\\
  \;\;\;\;\;\;\suniform{s\text{\_}in\text{\_}school\text{\_}type}{2}{2}\; ;\\
  \;\;\;\;\;\;\suniform{s\text{\_}birth\text{\_}year}{1990}{1998}\\
  \;\;\;\selsek\\
  \;\;\;\;\;\;\spifk\; 2/15 \; \sthenk\\
  \;\;\;\;\;\;\;\;\;\suniform{s\text{\_}in\text{\_}school\text{\_}type}{3}{3}\; ;\\
  \;\;\;\;\;\;\;\;\;\suniform{s\text{\_}birth\text{\_}year}{1985}{1992}\\
  \;\;\;\;\;\;\selsek\\
  \;\;\;\;\;\;\;\;\;\spifk\; 1/12 \; \sthenk\\
  \;\;\;\;\;\;\;\;\;\;\;\;\suniform{s\text{\_}in\text{\_}school\text{\_}type}{4}{4}\; ;\\
  \;\;\;\;\;\;\;\;\;\;\;\;\suniform{s\text{\_}birth\text{\_}year}{1980}{1985}\\
  \;\;\;\;\;\;\;\;\;\selsek\\
  \;\;\;\;\;\;\;\;\;\;\;\;\suniform{s\text{\_}in\text{\_}school\text{\_}type}{0}{0}\; ;\\
  \;\;\;\;\;\;\;\;\;\;\;\;\suniform{s\text{\_}birth\text{\_}year}{1900}{1985}\; ;\\
  \suniform{s\text{\_}address\text{\_}lat}{38867884}{39103178}\; ;\\
  \suniform{s\text{\_}address\text{\_}long}{-77058199}{-76825926}\\
\end{array}
}
\end{displaymath}

The query itself targets the pizza advertisement at users who are
either in college or aged 18 to 28, while living close to the pizza
shop (within a square region that is 2.5 miles on each side and
centered on the pizza shop).  If this condition is satisfied, then the
query returns 1, indicating that the ad should be displayed.  The full
text of the query is given below.

$$ \squerydefk \; pizza \; : \; \; \ra \; out$$
\begin{displaymath}{\small
\begin{array}{l}
  \psdefine{age}{\ebinop{-}{2010}{s\text{\_}birth\text{\_}year}}\\
  \psdefine{lr\text{\_}lat}{38967884}\\
  \psdefine{ul\text{\_}lat}{39003178}\\
  \psdefine{lr\text{\_}long}{-76958199}\\
  \psdefine{ul\text{\_}long}{-76925926}\\
  \sifk \; \lreln{\geq}{s\text{\_}in\text{\_}school\text{\_}type}{4} \; \sthenk\\
  \;\;\;\sassign{in\text{\_}school}{1}\\
  \selsek\\
  \;\;\;\sassign{in\text{\_}school}{0}\; ;\\
  \sifk \; \lbinop{\wedge}{\lreln{\geq}{age}{18}}{\lreln{\leq}{age}{28}} \; \sthenk\\
  \;\;\;\sassign{age\text{\_}criteria}{1}\\
  \selsek\\
  \;\;\;\sassign{age\text{\_}criteria}{0}\; ;\\
  \sifk \; \lbinop{\wedge}{\lbinop{\wedge}{\lbinop{\wedge}{\lreln{\leq}{s\text{\_}address\text{\_}lat}{ul\text{\_}lat}\\\;\;}{\lreln{\geq}{s\text{\_}address\text{\_}lat}{lr\text{\_}lat}\\\;\;}}{\lreln{\geq}{s\text{\_}address\text{\_}long}{lr\text{\_}long}\\\;\;}}{\lreln{\leq}{s\text{\_}address\text{\_}long}{ul\text{\_}long}}\\\sthenk\\
  \;\;\;\sassign{in\text{\_}box}{1}\\
  \selsek\\
  \;\;\;\sassign{in\text{\_}box}{0}\; ;\\
  \sifk \; \lbinop{\wedge}{\lbinop{\vee}{(\lreln{=}{in\text{\_}school}{1}}{\lreln{=}{age\text{\_}criteria}{1}})\\\;\;}{\lreln{=}{in\text{\_}box}{1}} \; \sthenk\\
  \;\;\;\sassign{out}{1}\\
  \selsek\\
  \;\;\;\sassign{out}{0}\\
\end{array}
}
\end{displaymath}

\subsection{Photo}

The photo query is a direct encoding of a case study that Facebook
includes on their advertising information
page~\cite{wedding-case-study}.  The advertisement was for CM
Photographics, and targets offers for wedding photography packages at
women between the ages of 24 and 30 who list in their profiles that
they are engaged.  The secret state consists of birth year, as before,
gender (0 indicates male, 1 indicates female), and ``relationship
status,'' which can take on a value from 0 to 9.  Each of these
relationship status values indicates one of the status choices
permitted by the Facebook software.  The example below involves only
four of these values, which are given below.
\begin{description}
\item[0] No answer
\item[1] Single
\item[2] In a relationship
\item[3] Engaged
\end{description}
The secret state and prebelief are as follows.
$$ \ssecretk: $$
\begin{displaymath}{\small
\begin{array}{l}
  \sassign{s\text{\_}birth\text{\_}year}{1983}\; ;\\
  \sassign{s\text{\_}gender}{0}\; ;\\
  \sassign{s\text{\_}relationship\text{\_}status}{0}\\
\end{array}
}
\end{displaymath}
$$ \sbeliefk: $$
\begin{displaymath}{\small
\begin{array}{l}
  \suniform{s\text{\_}birth\text{\_}year}{1900}{2010}\; ;\\
  \suniform{s\text{\_}gender}{0}{1}\; ;\\
  \suniform{s\text{\_}relationship\text{\_}status}{0}{3}\\
\end{array}
}
\end{displaymath}

The query itself is the following.
$$ \squerydefk \; cm\text{\_}advert \; : \; \; \ra \; out$$
\begin{displaymath}{\small
\begin{array}{l}
  \psdefine{age}{\ebinop{-}{2010}{s\text{\_}birth\text{\_}year}}\\
  \sifk \; \lbinop{\wedge}{\lreln{\geq}{age}{24}}{\lreln{\leq}{age}{30}} \; \sthenk\\
  \;\;\;\sassign{age\text{\_}sat}{1}\\
  \selsek\\
  \;\;\;\sassign{age\text{\_}sat}{0}\; ;\\
  \sifk \; \lbinop{\wedge}{\lbinop{\wedge}{\lreln{=}{s\text{\_}gender}{1}\\\;}{\lreln{=}{s\text{\_}relationship\text{\_}status}{3}\\\;}}{\lreln{=}{age\text{\_}sat}{1}} \; \sthenk\\
  \;\;\;\sassign{out}{1}\\
  \selsek\\
  \;\;\;\sassign{out}{0}\\
\end{array}
}
\end{displaymath}

\vspace{5mm} % !!! formatting

\subsection{Travel}

This example is another Facebook advertising case
study~\cite{visitbritain-case-study}.  It is based on an ad campaign
run by Britain's national tourism agency, VisitBritain.  The campaign
targeted English-speaking Facebook users currently residing in
countries with strong ties to the United Kingdom.  They further
filtered by showing the advertisement only to college graduates who
were at least 21 years of age.

We modeled this using four secret values: country, birth year, highest
completed education level, and primary language.  As with other
categorical data, we represent language and country using an
enumeration.  We ranked countries by number of Facebook users as
reported by socialbakers.com.  This resulted in the US being country
number 1 and the UK being country 3.  To populate the list of
countries with ``strong connections'' to the UK, we took a list of
former British colonies.  For the language attribute, we consider a
50-element enumeration where 0 indicates ``no answer'' and 1 indicates
``English'' (other values appear in the prebelief but are not used in
the query).

$$ \ssecretk: $$
\begin{displaymath}{\small
\begin{array}{l}
  \sassign{country}{1}\; ;\\
  \sassign{birth\text{\_}year}{1983}\; ;\\
  \sassign{completed\text{\_}school\text{\_}type}{4}\; ;\\
  \sassign{language}{5}\\
\end{array}
}
\end{displaymath}

$$ \sbeliefk: $$
\begin{displaymath}{\small
\begin{array}{l}
  \suniform{country}{1}{200}\; ;\\
  \suniform{birth\text{\_}year}{1900}{2011}\; ;\\
  \suniform{language}{1}{50}\; ;\\
  \suniform{completed\text{\_}school\text{\_}type}{0}{5}\\
\end{array}
}
\end{displaymath}

$$ \squerydefk \; travel \; : \; \; \ra \; out$$
\begin{displaymath}{\small
\begin{array}{l}
  \psdefine{age}{\ebinop{-}{2010}{birth\text{\_}year}}\\
  \sifk \; \lbinop{\vee}{\lbinop{\vee}{\lbinop{\vee}{\lbinop{\vee}{\lreln{=}{country}{1}}{\lreln{=}{country}{3}\\\;\;}}{\lreln{=}{country}{8}}}{\lreln{=}{country}{10}\\\;\;}}{\lreln{=}{country}{18}} \; \sthenk\\
  \;\;\;\sassign{main\text{\_}country}{1}\\
  \selsek\\
  \;\;\;\sassign{main\text{\_}country}{0}\; ;\\
  \sifk \; \lbinop{\vee}{\lbinop{\vee}{\lbinop{\vee}{\lbinop{\vee}{\lbinop{\vee}{\lbinop{\vee}{\lreln{=}{country}{169}}{\lreln{=}{country}{197}\\\;\;}}{\lreln{=}{country}{194}}}{\lreln{=}{country}{170}\\\;\;}}{\lreln{=}{country}{206}}}{\lreln{=}{country}{183}\\\;\;}}{\lreln{=}{country}{188}} \; \sthenk\\
  \;\;\;\sassign{island}{1}\\
  \selsek\\
  \;\;\;\sassign{island}{0}\; ;\\
  \sifk \; \lbinop{\wedge}{\lbinop{\wedge}{\lbinop{\wedge}{\lreln{=}{language}{1}\\\;\;}{\lbinop{\vee}{(\lreln{=}{main\text{\_}country}{1}}{\lreln{=}{island}{1})\\\;\;}}}{\lreln{\geq}{age}{21}\\\;\;}}{\lreln{\geq}{completed\text{\_}school\text{\_}type}{4}} \; \sthenk\\
  \;\;\;\sassign{out}{1}\\
  \selsek\\
  \;\;\;\sassign{out}{0}\\
\end{array}
}
\end{displaymath}

\section{Relational Queries} \label{sec-appendix-relational}

The two queries below, $ is\text{\_}target\text{\_}close $ and $
who\text{\_}is\text{\_}closer $ introduce relations between variables
after revision, even though the example initial belief had no such
relations. The initial belief stipulates the location of 2 objects is
somewhere within a rectangular region. The $
is\text{\_}target\text{\_}close $ query determines if a given location
is within $ dist $ of the first object, measured using Manhattan
distance. This query introduces relations between only 2 variables
(latitude and longitude) which can be exactly represented using
octagons but cannot using intervals.

The $ who\text{\_}is\text{\_}closer $ query performs a similar
computation, but instead determines which of the two objects in the
initial belief is closer to the new target location. The post belief
can be handled by use of polyhedra, but not octagons, as it introduces
relationships between more than 2 variables (latitudes and longitudes
of 2 different objects).

% ../prob --pmock --bench tasks/task_1348957585_259022/timing.csv
% --precision 4 --domain poly --simplify halfs --seed 0
% ../examples/bench/closer.pol
% the above breaks latte

$$ \sbeliefk: $$
\begin{displaymath}{\small
\begin{array}{l}
  \suniform{loc\text{\_}lat1}{29267245}{36332852}\; ;\\
  \suniform{loc\text{\_}long1}{41483216}{46405563}\; ;\\
  \suniform{loc\text{\_}lat2}{29267245}{36332852}\; ;\\
  \suniform{loc\text{\_}long2}{41483216}{46405563}\\
\end{array}
}
\end{displaymath}

$$ \squerydefk \; is\text{\_}target\text{\_}close \;
: \;target\text{\_}location\text{\_}lat\;target\text{\_}location\text{\_}long\;dist \; \ra \;
is\text{\_}close$$
\begin{displaymath}{\small
\begin{array}{l}
  \sassign{is\text{\_}close}{0}\; ;\\
  \sassign{dist\text{\_}lat}{\ebinop{-}{loc\text{\_}lat1}{target\text{\_}location\text{\_}lat}}\;
  ;\\
  \sassign{dist\text{\_}long}{\ebinop{-}{loc\text{\_}long1}{target\text{\_}location\text{\_}long}}\;
  ;\\
  \sifk \; \lreln{<}{dist\text{\_}lat}{0} \; \sthenk\\
  \;\;\;\sassign{dist\text{\_}lat}{\ebinop{\times}{-1}{dist\text{\_}lat}}\;
  ;\\
  \sifk \; \lreln{<}{dist\text{\_}long}{0} \; \sthenk\\
  \;\;\;\sassign{dist\text{\_}long}{\ebinop{\times}{-1}{dist\text{\_}long}}\;
  ;\\
  \sifk \; \lreln{\leq}{\ebinop{+}{dist\text{\_}lat}{dist\text{\_}long}}{dist} \; \sthenk\\
  \;\;\;\sassign{is\text{\_}close}{1}\\
\end{array}
}
\end{displaymath}

$$ \squerydefk \; who\text{\_}is\text{\_}closer \;
: \;target\text{\_}location\text{\_}lat\;target\text{\_}location\text{\_}long \; \ra \;
who\text{\_}closer$$
\begin{displaymath}{\small
\begin{array}{l}
  \sassign{diff\text{\_}lat1}{\ebinop{-}{loc\text{\_}lat1}{target\text{\_}location\text{\_}lat}}\;
  ;\\
  \sassign{diff\text{\_}long1}{\ebinop{-}{loc\text{\_}long1}{target\text{\_}location\text{\_}long}}\;
  ;\\
  \sassign{diff\text{\_}lat2}{\ebinop{-}{loc\text{\_}lat2}{target\text{\_}location\text{\_}lat}}\;
  ;\\
  \sassign{diff\text{\_}long2}{\ebinop{-}{loc\text{\_}long2}{target\text{\_}location\text{\_}long}}\;
  ;\\
  \sifk \; \lreln{<}{diff\text{\_}lat1}{0} \; \sthenk\\
  \;\;\;\sassign{diff\text{\_}lat1}{\ebinop{\times}{-1}{diff\text{\_}lat1}}\;
  ;\\
  \sifk \; \lreln{<}{diff\text{\_}long1}{0} \; \sthenk\\
  \;\;\;\sassign{diff\text{\_}long1}{\ebinop{\times}{-1}{diff\text{\_}long1}}\;
  ;\\
  \sifk \; \lreln{<}{diff\text{\_}lat2}{0} \; \sthenk\\
  \;\;\;\sassign{diff\text{\_}lat2}{\ebinop{\times}{-1}{diff\text{\_}lat2}}\;
  ;\\
  \sifk \; \lreln{<}{diff\text{\_}long2}{0} \; \sthenk\\
  \;\;\;\sassign{diff\text{\_}long2}{\ebinop{\times}{-1}{diff\text{\_}long2}}\;
  ;\\
  \sassign{dist1}{\ebinop{+}{diff\text{\_}long1}{diff\text{\_}lat1}}\;
  ;\\
  \sassign{dist2}{\ebinop{+}{diff\text{\_}long2}{diff\text{\_}lat2}}\;
  ;\\
  \sifk \; \lreln{\leq}{dist1}{dist2} \; \sthenk\\
  \;\;\;\sassign{who\text{\_}closer}{0}\\
  \selsek\\
  \;\;\;\sassign{who\text{\_}closer}{1}\\
\end{array}
}
\end{displaymath}


\section{Benchmark Results}

Table~\ref{fig:bench_table} tabulates performance results for all of
the benchmark programs, for each possible base domain
(intervals, octagons, and polyhedra labeled $ \Box $, $ \Diamond $,
and $ \pentagon $ respectively).  Each column is the maximum size
of the permitted powerset, whereas each grouping of rows contains,
respectively, the wall clock time in seconds (median of 20 runs), the
running time's semi-interquartile range (SIQR) with the number of
outliers in parentheses (which are defined to be the points $
3\times\text{SIQR} $ below the first quartile or above the third), and
the max belief computed (smaller being more accurate).

\pagebreak
\begin{table*}[h!]
\centering
{\tiny \begin{tabular}{||p{1.35cm}|p{0.50cm}p{0.50cm}p{0.50cm}p{0.50cm}p{0.50cm}p{0.50cm}p{0.50cm}p{0.50cm}p{0.50cm}p{0.50cm}p{0.50cm}p{0.50cm}p{0.50cm}p{0.50cm}p{0.50cm}p{0.50cm}p{0.50cm}c||}
\hline \hline {\small time [s]} \par {\scriptsize max belief}& \multicolumn{18}{c||}{\textbf{prob-poly set size bound}} \\
\hline \hline \textbf{query} & \textbf{1} & \textbf{2} & \textbf{3} & \textbf{4} & \textbf{5} & \textbf{6} & \textbf{7} & \textbf{8} & \textbf{9} & \textbf{10} & \textbf{15} & \textbf{20} & \textbf{25} & \textbf{30} & \textbf{35} & \textbf{40} & \textbf{$ \infty $} & \\
\hline \hline bday \par 1 box & {\small 0.0}\par{\scriptsize\parbox{1.0cm}{1}} \par{\scriptsize 0} & {\small 0.0}\par{\scriptsize\parbox{1.0cm}{0.00386}} \par{\scriptsize 0} & {\small 0.0}\par{\scriptsize\parbox{1.0cm}{0.00386}} \par{\scriptsize 0} & {\small 0.0}\par{\scriptsize\parbox{1.0cm}{0.00386}} \par{\scriptsize 0} & {\small 0.0}\par{\scriptsize\parbox{1.0cm}{0.00386}} \par{\scriptsize 0} & {\small 0.0}\par{\scriptsize\parbox{1.0cm}{0.00386}} \par{\scriptsize 0} & {\small 0.0}\par{\scriptsize\parbox{1.0cm}{0.00386}} \par{\scriptsize 0} & {\small 0.0}\par{\scriptsize\parbox{1.0cm}{0.00386}} \par{\scriptsize 0} & {\small 0.0}\par{\scriptsize\parbox{1.0cm}{0.00386}} \par{\scriptsize 0} & {\small 0.0}\par{\scriptsize\parbox{1.0cm}{0.00386}} \par{\scriptsize 0} & {\small 0.0}\par{\scriptsize\parbox{1.0cm}{0.00386}} \par{\scriptsize 0} & {\small 0.0}\par{\scriptsize\parbox{1.0cm}{0.00386}} \par{\scriptsize 0} & {\small 0.0}\par{\scriptsize\parbox{1.0cm}{0.00386}} \par{\scriptsize 0} & {\small 0.0}\par{\scriptsize\parbox{1.0cm}{0.00386}} \par{\scriptsize 0} & {\small 0.0}\par{\scriptsize\parbox{1.0cm}{0.00386}} \par{\scriptsize 0} & {\small 0.0}\par{\scriptsize\parbox{1.0cm}{0.00386}} \par{\scriptsize 0} & {\small 0.0}\par{\scriptsize\parbox{1.0cm}{0.00386}} \par{\scriptsize 0} & \\
\hline bday \par 1 octalatte & {\small 0.6}\par{\scriptsize\parbox{1.0cm}{1}} \par{\scriptsize 8} & {\small 0.6}\par{\scriptsize\parbox{1.0cm}{0.00386}} \par{\scriptsize 8} & {\small 0.7}\par{\scriptsize\parbox{1.0cm}{0.00386}} \par{\scriptsize 8} & {\small 0.7}\par{\scriptsize\parbox{1.0cm}{0.00386}} \par{\scriptsize 8} & {\small 0.7}\par{\scriptsize\parbox{1.0cm}{0.00386}} \par{\scriptsize 8} & {\small 0.7}\par{\scriptsize\parbox{1.0cm}{0.00386}} \par{\scriptsize 8} & {\small 0.7}\par{\scriptsize\parbox{1.0cm}{0.00386}} \par{\scriptsize 8} & {\small 0.8}\par{\scriptsize\parbox{1.0cm}{0.00386}} \par{\scriptsize 8} & {\small 0.7}\par{\scriptsize\parbox{1.0cm}{0.00386}} \par{\scriptsize 8} & {\small 0.7}\par{\scriptsize\parbox{1.0cm}{0.00386}} \par{\scriptsize 8} & {\small 0.9}\par{\scriptsize\parbox{1.0cm}{0.00386}} \par{\scriptsize 8} & {\small 0.7}\par{\scriptsize\parbox{1.0cm}{0.00386}} \par{\scriptsize 8} & {\small 0.7}\par{\scriptsize\parbox{1.0cm}{0.00386}} \par{\scriptsize 8} & {\small 0.8}\par{\scriptsize\parbox{1.0cm}{0.00386}} \par{\scriptsize 8} & {\small 0.8}\par{\scriptsize\parbox{1.0cm}{0.00386}} \par{\scriptsize 8} & {\small 0.7}\par{\scriptsize\parbox{1.0cm}{0.00386}} \par{\scriptsize 8} & {\small 0.7}\par{\scriptsize\parbox{1.0cm}{0.00386}} \par{\scriptsize 8} & \\
\hline bday \par 1 poly & {\small 0.7}\par{\scriptsize\parbox{1.0cm}{1}} \par{\scriptsize 8} & {\small 0.7}\par{\scriptsize\parbox{1.0cm}{0.00386}} \par{\scriptsize 8} & {\small 0.8}\par{\scriptsize\parbox{1.0cm}{0.00386}} \par{\scriptsize 8} & {\small 0.9}\par{\scriptsize\parbox{1.0cm}{0.00386}} \par{\scriptsize 8} & {\small 0.8}\par{\scriptsize\parbox{1.0cm}{0.00386}} \par{\scriptsize 8} & {\small 0.8}\par{\scriptsize\parbox{1.0cm}{0.00386}} \par{\scriptsize 8} & {\small 0.8}\par{\scriptsize\parbox{1.0cm}{0.00386}} \par{\scriptsize 8} & {\small 0.8}\par{\scriptsize\parbox{1.0cm}{0.00386}} \par{\scriptsize 8} & {\small 0.8}\par{\scriptsize\parbox{1.0cm}{0.00386}} \par{\scriptsize 8} & {\small 0.8}\par{\scriptsize\parbox{1.0cm}{0.00386}} \par{\scriptsize 8} & {\small 0.8}\par{\scriptsize\parbox{1.0cm}{0.00386}} \par{\scriptsize 8} & {\small 0.9}\par{\scriptsize\parbox{1.0cm}{0.00386}} \par{\scriptsize 8} & {\small 0.9}\par{\scriptsize\parbox{1.0cm}{0.00386}} \par{\scriptsize 8} & {\small 0.8}\par{\scriptsize\parbox{1.0cm}{0.00386}} \par{\scriptsize 8} & {\small 0.8}\par{\scriptsize\parbox{1.0cm}{0.00386}} \par{\scriptsize 8} & {\small 0.8}\par{\scriptsize\parbox{1.0cm}{0.00386}} \par{\scriptsize 8} & {\small 0.8}\par{\scriptsize\parbox{1.0cm}{0.00386}} \par{\scriptsize 8} & \\
\hline bday \par 1+2 box & {\small 0.0}\par{\scriptsize\parbox{1.0cm}{1}} \par{\scriptsize 0} & {\small 0.0}\par{\scriptsize\parbox{1.0cm}{1}} \par{\scriptsize 0} & {\small 0.0}\par{\scriptsize\parbox{1.0cm}{0.02703}} \par{\scriptsize 0} & {\small 0.0}\par{\scriptsize\parbox{1.0cm}{0.02703}} \par{\scriptsize 0} & {\small 0.0}\par{\scriptsize\parbox{1.0cm}{0.02703}} \par{\scriptsize 0} & {\small 0.0}\par{\scriptsize\parbox{1.0cm}{0.02703}} \par{\scriptsize 0} & {\small 0.0}\par{\scriptsize\parbox{1.0cm}{0.02703}} \par{\scriptsize 0} & {\small 0.0}\par{\scriptsize\parbox{1.0cm}{0.02703}} \par{\scriptsize 0} & {\small 0.0}\par{\scriptsize\parbox{1.0cm}{0.02703}} \par{\scriptsize 0} & {\small 0.0}\par{\scriptsize\parbox{1.0cm}{0.02703}} \par{\scriptsize 0} & {\small 0.0}\par{\scriptsize\parbox{1.0cm}{0.02703}} \par{\scriptsize 0} & {\small 0.0}\par{\scriptsize\parbox{1.0cm}{0.02703}} \par{\scriptsize 0} & {\small 0.0}\par{\scriptsize\parbox{1.0cm}{0.02703}} \par{\scriptsize 0} & {\small 0.0}\par{\scriptsize\parbox{1.0cm}{0.02703}} \par{\scriptsize 0} & {\small 0.0}\par{\scriptsize\parbox{1.0cm}{0.02703}} \par{\scriptsize 0} & {\small 0.0}\par{\scriptsize\parbox{1.0cm}{0.02703}} \par{\scriptsize 0} & {\small 0.0}\par{\scriptsize\parbox{1.0cm}{0.02703}} \par{\scriptsize 0} & \\
\hline bday \par 1+2 octalatte & {\small 0.8}\par{\scriptsize\parbox{1.0cm}{1}} \par{\scriptsize 6} & {\small 1.3}\par{\scriptsize\parbox{1.0cm}{1}} \par{\scriptsize 8} & {\small 1.4}\par{\scriptsize\parbox{1.0cm}{0.02703}} \par{\scriptsize 8} & {\small 1.4}\par{\scriptsize\parbox{1.0cm}{0.02703}} \par{\scriptsize 8} & {\small 1.5}\par{\scriptsize\parbox{1.0cm}{0.02703}} \par{\scriptsize 8} & {\small 1.5}\par{\scriptsize\parbox{1.0cm}{0.02703}} \par{\scriptsize 8} & {\small 1.5}\par{\scriptsize\parbox{1.0cm}{0.02703}} \par{\scriptsize 8} & {\small 2.1}\par{\scriptsize\parbox{1.0cm}{0.02703}} \par{\scriptsize 8} & {\small 1.5}\par{\scriptsize\parbox{1.0cm}{0.02703}} \par{\scriptsize 8} & {\small 2.4}\par{\scriptsize\parbox{1.0cm}{0.02703}} \par{\scriptsize 8} & {\small 1.7}\par{\scriptsize\parbox{1.0cm}{0.02703}} \par{\scriptsize 8} & {\small 1.5}\par{\scriptsize\parbox{1.0cm}{0.02703}} \par{\scriptsize 8} & {\small 1.5}\par{\scriptsize\parbox{1.0cm}{0.02703}} \par{\scriptsize 8} & {\small 1.5}\par{\scriptsize\parbox{1.0cm}{0.02703}} \par{\scriptsize 8} & {\small 1.5}\par{\scriptsize\parbox{1.0cm}{0.02703}} \par{\scriptsize 8} & {\small 1.4}\par{\scriptsize\parbox{1.0cm}{0.02703}} \par{\scriptsize 8} & {\small 1.5}\par{\scriptsize\parbox{1.0cm}{0.02703}} \par{\scriptsize 8} & \\
\hline bday \par 1+2 poly & {\small 1.1}\par{\scriptsize\parbox{1.0cm}{1}} \par{\scriptsize 8} & {\small 1.3}\par{\scriptsize\parbox{1.0cm}{1}} \par{\scriptsize 8} & {\small 1.6}\par{\scriptsize\parbox{1.0cm}{0.02703}} \par{\scriptsize 8} & {\small 2.6}\par{\scriptsize\parbox{1.0cm}{0.02703}} \par{\scriptsize 8} & {\small 1.6}\par{\scriptsize\parbox{1.0cm}{0.02703}} \par{\scriptsize 8} & {\small 1.6}\par{\scriptsize\parbox{1.0cm}{0.02703}} \par{\scriptsize 8} & {\small 1.6}\par{\scriptsize\parbox{1.0cm}{0.02703}} \par{\scriptsize 8} & {\small 1.6}\par{\scriptsize\parbox{1.0cm}{0.02703}} \par{\scriptsize 8} & {\small 1.6}\par{\scriptsize\parbox{1.0cm}{0.02703}} \par{\scriptsize 8} & {\small 1.6}\par{\scriptsize\parbox{1.0cm}{0.02703}} \par{\scriptsize 8} & {\small 1.6}\par{\scriptsize\parbox{1.0cm}{0.02703}} \par{\scriptsize 8} & {\small 1.6}\par{\scriptsize\parbox{1.0cm}{0.02703}} \par{\scriptsize 8} & {\small 1.6}\par{\scriptsize\parbox{1.0cm}{0.02703}} \par{\scriptsize 8} & {\small 1.7}\par{\scriptsize\parbox{1.0cm}{0.02703}} \par{\scriptsize 8} & {\small 1.6}\par{\scriptsize\parbox{1.0cm}{0.02703}} \par{\scriptsize 8} & {\small 1.6}\par{\scriptsize\parbox{1.0cm}{0.02703}} \par{\scriptsize 8} & {\small 1.6}\par{\scriptsize\parbox{1.0cm}{0.02703}} \par{\scriptsize 8} & \\
\hline bday \par 1+2+special box & {\small 0.5}\par{\scriptsize\parbox{1.0cm}{1}} \par{\scriptsize 0} & {\small 0.5}\par{\scriptsize\parbox{1.0cm}{1}} \par{\scriptsize 0} & {\small 0.4}\par{\scriptsize\parbox{1.0cm}{4.22e-4}} \par{\scriptsize 0} & {\small 0.4}\par{\scriptsize\parbox{1.0cm}{4.22e-4}} \par{\scriptsize 0} & {\small 0.4}\par{\scriptsize\parbox{1.0cm}{4.22e-4}} \par{\scriptsize 0} & {\small 0.4}\par{\scriptsize\parbox{1.0cm}{4.22e-4}} \par{\scriptsize 0} & {\small 0.4}\par{\scriptsize\parbox{1.0cm}{4.22e-4}} \par{\scriptsize 0} & {\small 0.4}\par{\scriptsize\parbox{1.0cm}{8.06e-4}} \par{\scriptsize 0} & {\small 0.4}\par{\scriptsize\parbox{1.0cm}{8.06e-4}} \par{\scriptsize 0} & {\small 0.5}\par{\scriptsize\parbox{1.0cm}{8.06e-4}} \par{\scriptsize 0} & {\small 0.5}\par{\scriptsize\parbox{1.0cm}{4.60e-4}} \par{\scriptsize 0} & {\small 0.5}\par{\scriptsize\parbox{1.0cm}{0.00186}} \par{\scriptsize 0} & {\small 0.5}\par{\scriptsize\parbox{1.0cm}{0.00163}} \par{\scriptsize 0} & {\small 0.5}\par{\scriptsize\parbox{1.0cm}{0.00145}} \par{\scriptsize 0} & {\small 0.5}\par{\scriptsize\parbox{1.0cm}{4.60e-4}} \par{\scriptsize 0} & {\small 0.5}\par{\scriptsize\parbox{1.0cm}{3.84e-4}} \par{\scriptsize 0} & {\small 0.5}\par{\scriptsize\parbox{1.0cm}{3.84e-4}} \par{\scriptsize 0} & \\
\hline bday \par 1+2+special octalatte & {\small 2.7}\par{\scriptsize\parbox{1.0cm}{1}} \par{\scriptsize 10} & {\small 3.7}\par{\scriptsize\parbox{1.0cm}{1}} \par{\scriptsize 12} & {\small 8.0}\par{\scriptsize\parbox{1.0cm}{4.22e-4}} \par{\scriptsize 12} & {\small 5.0}\par{\scriptsize\parbox{1.0cm}{4.22e-4}} \par{\scriptsize 10} & {\small 7.2}\par{\scriptsize\parbox{1.0cm}{4.22e-4}} \par{\scriptsize 11} & {\small 5.9}\par{\scriptsize\parbox{1.0cm}{4.22e-4}} \par{\scriptsize 11} & {\small 6.7}\par{\scriptsize\parbox{1.0cm}{4.22e-4}} \par{\scriptsize 11} & {\small 8.0}\par{\scriptsize\parbox{1.0cm}{8.06e-4}} \par{\scriptsize 11} & {\small 7.0}\par{\scriptsize\parbox{1.0cm}{8.06e-4}} \par{\scriptsize 11} & {\small 10.7}\par{\scriptsize\parbox{1.0cm}{8.06e-4}} \par{\scriptsize 11} & {\small 12.4}\par{\scriptsize\parbox{1.0cm}{4.60e-4}} \par{\scriptsize 11} & {\small 14.7}\par{\scriptsize\parbox{1.0cm}{0.00132}} \par{\scriptsize 16} & {\small 14.9}\par{\scriptsize\parbox{1.0cm}{0.00116}} \par{\scriptsize 16} & {\small 15.9}\par{\scriptsize\parbox{1.0cm}{0.00103}} \par{\scriptsize 16} & {\small 15.3}\par{\scriptsize\parbox{1.0cm}{4.60e-4}} \par{\scriptsize 14} & {\small 17.2}\par{\scriptsize\parbox{1.0cm}{3.84e-4}} \par{\scriptsize 10} & {\small 16.1}\par{\scriptsize\parbox{1.0cm}{3.84e-4}} \par{\scriptsize 10} & \\
\hline bday \par 1+2+special poly & {\small 3.4}\par{\scriptsize\parbox{1.0cm}{1}} \par{\scriptsize 10} & {\small 5.7}\par{\scriptsize\parbox{1.0cm}{1}} \par{\scriptsize 12} & {\small 5.0}\par{\scriptsize\parbox{1.0cm}{4.22e-4}} \par{\scriptsize 12} & {\small 6.3}\par{\scriptsize\parbox{1.0cm}{4.22e-4}} \par{\scriptsize 10} & {\small 8.7}\par{\scriptsize\parbox{1.0cm}{4.22e-4}} \par{\scriptsize 11} & {\small 6.0}\par{\scriptsize\parbox{1.0cm}{4.22e-4}} \par{\scriptsize 11} & {\small 6.8}\par{\scriptsize\parbox{1.0cm}{4.22e-4}} \par{\scriptsize 11} & {\small 10.8}\par{\scriptsize\parbox{1.0cm}{8.06e-4}} \par{\scriptsize 11} & {\small 8.7}\par{\scriptsize\parbox{1.0cm}{8.06e-4}} \par{\scriptsize 11} & {\small 7.4}\par{\scriptsize\parbox{1.0cm}{8.06e-4}} \par{\scriptsize 11} & {\small 10.9}\par{\scriptsize\parbox{1.0cm}{4.60e-4}} \par{\scriptsize 11} & {\small 14.2}\par{\scriptsize\parbox{1.0cm}{4.60e-4}} \par{\scriptsize 14} & {\small 16.9}\par{\scriptsize\parbox{1.0cm}{4.60e-4}} \par{\scriptsize 14} & {\small 18.2}\par{\scriptsize\parbox{1.0cm}{4.60e-4}} \par{\scriptsize 14} & {\small 18.5}\par{\scriptsize\parbox{1.0cm}{4.22e-4}} \par{\scriptsize 12} & {\small 18.0}\par{\scriptsize\parbox{1.0cm}{3.84e-4}} \par{\scriptsize 10} & {\small 18.2}\par{\scriptsize\parbox{1.0cm}{3.84e-4}} \par{\scriptsize 10} & \\
\hline \hline bday large \par 1 box & {\small 0.0}\par{\scriptsize\parbox{1.0cm}{1}} \par{\scriptsize 0} & {\small 0.0}\par{\scriptsize\parbox{1.0cm}{0.00141}} \par{\scriptsize 0} & {\small 0.0}\par{\scriptsize\parbox{1.0cm}{0.00141}} \par{\scriptsize 0} & {\small 0.0}\par{\scriptsize\parbox{1.0cm}{0.00141}} \par{\scriptsize 0} & {\small 0.0}\par{\scriptsize\parbox{1.0cm}{0.00141}} \par{\scriptsize 0} & {\small 0.0}\par{\scriptsize\parbox{1.0cm}{0.00141}} \par{\scriptsize 0} & {\small 0.0}\par{\scriptsize\parbox{1.0cm}{0.00141}} \par{\scriptsize 0} & {\small 0.0}\par{\scriptsize\parbox{1.0cm}{0.00141}} \par{\scriptsize 0} & {\small 0.0}\par{\scriptsize\parbox{1.0cm}{0.00141}} \par{\scriptsize 0} & {\small 0.0}\par{\scriptsize\parbox{1.0cm}{0.00141}} \par{\scriptsize 0} & {\small 0.0}\par{\scriptsize\parbox{1.0cm}{0.00141}} \par{\scriptsize 0} & {\small 0.0}\par{\scriptsize\parbox{1.0cm}{0.00141}} \par{\scriptsize 0} & {\small 0.0}\par{\scriptsize\parbox{1.0cm}{0.00141}} \par{\scriptsize 0} & {\small 0.0}\par{\scriptsize\parbox{1.0cm}{0.00141}} \par{\scriptsize 0} & {\small 0.0}\par{\scriptsize\parbox{1.0cm}{0.00141}} \par{\scriptsize 0} & {\small 0.0}\par{\scriptsize\parbox{1.0cm}{0.00141}} \par{\scriptsize 0} & {\small 0.0}\par{\scriptsize\parbox{1.0cm}{0.00141}} \par{\scriptsize 0} & \\
\hline bday large \par 1 octalatte & {\small 0.6}\par{\scriptsize\parbox{1.0cm}{1}} \par{\scriptsize 8} & {\small 0.6}\par{\scriptsize\parbox{1.0cm}{0.00141}} \par{\scriptsize 8} & {\small 0.8}\par{\scriptsize\parbox{1.0cm}{0.00141}} \par{\scriptsize 8} & {\small 0.8}\par{\scriptsize\parbox{1.0cm}{0.00141}} \par{\scriptsize 8} & {\small 0.7}\par{\scriptsize\parbox{1.0cm}{0.00141}} \par{\scriptsize 8} & {\small 0.8}\par{\scriptsize\parbox{1.0cm}{0.00141}} \par{\scriptsize 8} & {\small 0.7}\par{\scriptsize\parbox{1.0cm}{0.00141}} \par{\scriptsize 8} & {\small 0.7}\par{\scriptsize\parbox{1.0cm}{0.00141}} \par{\scriptsize 8} & {\small 0.7}\par{\scriptsize\parbox{1.0cm}{0.00141}} \par{\scriptsize 8} & {\small 0.7}\par{\scriptsize\parbox{1.0cm}{0.00141}} \par{\scriptsize 8} & {\small 1.7}\par{\scriptsize\parbox{1.0cm}{0.00141}} \par{\scriptsize 8} & {\small 0.7}\par{\scriptsize\parbox{1.0cm}{0.00141}} \par{\scriptsize 8} & {\small 0.7}\par{\scriptsize\parbox{1.0cm}{0.00141}} \par{\scriptsize 8} & {\small 0.7}\par{\scriptsize\parbox{1.0cm}{0.00141}} \par{\scriptsize 8} & {\small 0.7}\par{\scriptsize\parbox{1.0cm}{0.00141}} \par{\scriptsize 8} & {\small 0.7}\par{\scriptsize\parbox{1.0cm}{0.00141}} \par{\scriptsize 8} & {\small 0.8}\par{\scriptsize\parbox{1.0cm}{0.00141}} \par{\scriptsize 8} & \\
\hline bday large \par 1 poly & {\small 0.7}\par{\scriptsize\parbox{1.0cm}{1}} \par{\scriptsize 8} & {\small 0.7}\par{\scriptsize\parbox{1.0cm}{0.00141}} \par{\scriptsize 8} & {\small 0.8}\par{\scriptsize\parbox{1.0cm}{0.00141}} \par{\scriptsize 8} & {\small 0.9}\par{\scriptsize\parbox{1.0cm}{0.00141}} \par{\scriptsize 8} & {\small 0.8}\par{\scriptsize\parbox{1.0cm}{0.00141}} \par{\scriptsize 8} & {\small 0.8}\par{\scriptsize\parbox{1.0cm}{0.00141}} \par{\scriptsize 8} & {\small 0.9}\par{\scriptsize\parbox{1.0cm}{0.00141}} \par{\scriptsize 8} & {\small 0.9}\par{\scriptsize\parbox{1.0cm}{0.00141}} \par{\scriptsize 8} & {\small 1.5}\par{\scriptsize\parbox{1.0cm}{0.00141}} \par{\scriptsize 8} & {\small 0.9}\par{\scriptsize\parbox{1.0cm}{0.00141}} \par{\scriptsize 8} & {\small 0.8}\par{\scriptsize\parbox{1.0cm}{0.00141}} \par{\scriptsize 8} & {\small 0.8}\par{\scriptsize\parbox{1.0cm}{0.00141}} \par{\scriptsize 8} & {\small 0.9}\par{\scriptsize\parbox{1.0cm}{0.00141}} \par{\scriptsize 8} & {\small 0.8}\par{\scriptsize\parbox{1.0cm}{0.00141}} \par{\scriptsize 8} & {\small 0.9}\par{\scriptsize\parbox{1.0cm}{0.00141}} \par{\scriptsize 8} & {\small 0.8}\par{\scriptsize\parbox{1.0cm}{0.00141}} \par{\scriptsize 8} & {\small 0.8}\par{\scriptsize\parbox{1.0cm}{0.00141}} \par{\scriptsize 8} & \\
\hline bday large \par 1+2 box & {\small 0.0}\par{\scriptsize\parbox{1.0cm}{1}} \par{\scriptsize 0} & {\small 0.0}\par{\scriptsize\parbox{1.0cm}{1}} \par{\scriptsize 0} & {\small 0.0}\par{\scriptsize\parbox{1.0cm}{0.00990}} \par{\scriptsize 0} & {\small 0.0}\par{\scriptsize\parbox{1.0cm}{0.00990}} \par{\scriptsize 0} & {\small 0.0}\par{\scriptsize\parbox{1.0cm}{0.00990}} \par{\scriptsize 0} & {\small 0.0}\par{\scriptsize\parbox{1.0cm}{0.00990}} \par{\scriptsize 0} & {\small 0.0}\par{\scriptsize\parbox{1.0cm}{0.00990}} \par{\scriptsize 0} & {\small 0.0}\par{\scriptsize\parbox{1.0cm}{0.00990}} \par{\scriptsize 0} & {\small 0.0}\par{\scriptsize\parbox{1.0cm}{0.00990}} \par{\scriptsize 0} & {\small 0.0}\par{\scriptsize\parbox{1.0cm}{0.00990}} \par{\scriptsize 0} & {\small 0.0}\par{\scriptsize\parbox{1.0cm}{0.00990}} \par{\scriptsize 0} & {\small 0.0}\par{\scriptsize\parbox{1.0cm}{0.00990}} \par{\scriptsize 0} & {\small 0.0}\par{\scriptsize\parbox{1.0cm}{0.00990}} \par{\scriptsize 0} & {\small 0.0}\par{\scriptsize\parbox{1.0cm}{0.00990}} \par{\scriptsize 0} & {\small 0.0}\par{\scriptsize\parbox{1.0cm}{0.00990}} \par{\scriptsize 0} & {\small 0.0}\par{\scriptsize\parbox{1.0cm}{0.00990}} \par{\scriptsize 0} & {\small 0.0}\par{\scriptsize\parbox{1.0cm}{0.00990}} \par{\scriptsize 0} & \\
\hline bday large \par 1+2 octalatte & {\small 0.8}\par{\scriptsize\parbox{1.0cm}{1}} \par{\scriptsize 6} & {\small 1.3}\par{\scriptsize\parbox{1.0cm}{1}} \par{\scriptsize 8} & {\small 1.5}\par{\scriptsize\parbox{1.0cm}{0.00990}} \par{\scriptsize 8} & {\small 1.5}\par{\scriptsize\parbox{1.0cm}{0.00990}} \par{\scriptsize 8} & {\small 1.4}\par{\scriptsize\parbox{1.0cm}{0.00990}} \par{\scriptsize 8} & {\small 1.5}\par{\scriptsize\parbox{1.0cm}{0.00990}} \par{\scriptsize 8} & {\small 1.5}\par{\scriptsize\parbox{1.0cm}{0.00990}} \par{\scriptsize 8} & {\small 1.5}\par{\scriptsize\parbox{1.0cm}{0.00990}} \par{\scriptsize 8} & {\small 1.5}\par{\scriptsize\parbox{1.0cm}{0.00990}} \par{\scriptsize 8} & {\small 1.4}\par{\scriptsize\parbox{1.0cm}{0.00990}} \par{\scriptsize 8} & {\small 3.0}\par{\scriptsize\parbox{1.0cm}{0.00990}} \par{\scriptsize 8} & {\small 1.5}\par{\scriptsize\parbox{1.0cm}{0.00990}} \par{\scriptsize 8} & {\small 1.4}\par{\scriptsize\parbox{1.0cm}{0.00990}} \par{\scriptsize 8} & {\small 1.4}\par{\scriptsize\parbox{1.0cm}{0.00990}} \par{\scriptsize 8} & {\small 1.4}\par{\scriptsize\parbox{1.0cm}{0.00990}} \par{\scriptsize 8} & {\small 1.4}\par{\scriptsize\parbox{1.0cm}{0.00990}} \par{\scriptsize 8} & {\small 1.5}\par{\scriptsize\parbox{1.0cm}{0.00990}} \par{\scriptsize 8} & \\
\hline bday large \par 1+2 poly & {\small 1.1}\par{\scriptsize\parbox{1.0cm}{1}} \par{\scriptsize 8} & {\small 1.3}\par{\scriptsize\parbox{1.0cm}{1}} \par{\scriptsize 8} & {\small 1.6}\par{\scriptsize\parbox{1.0cm}{0.00990}} \par{\scriptsize 8} & {\small 1.6}\par{\scriptsize\parbox{1.0cm}{0.00990}} \par{\scriptsize 8} & {\small 1.6}\par{\scriptsize\parbox{1.0cm}{0.00990}} \par{\scriptsize 8} & {\small 1.6}\par{\scriptsize\parbox{1.0cm}{0.00990}} \par{\scriptsize 8} & {\small 1.6}\par{\scriptsize\parbox{1.0cm}{0.00990}} \par{\scriptsize 8} & {\small 1.6}\par{\scriptsize\parbox{1.0cm}{0.00990}} \par{\scriptsize 8} & {\small 3.1}\par{\scriptsize\parbox{1.0cm}{0.00990}} \par{\scriptsize 8} & {\small 1.7}\par{\scriptsize\parbox{1.0cm}{0.00990}} \par{\scriptsize 8} & {\small 1.6}\par{\scriptsize\parbox{1.0cm}{0.00990}} \par{\scriptsize 8} & {\small 1.6}\par{\scriptsize\parbox{1.0cm}{0.00990}} \par{\scriptsize 8} & {\small 2.6}\par{\scriptsize\parbox{1.0cm}{0.00990}} \par{\scriptsize 8} & {\small 1.6}\par{\scriptsize\parbox{1.0cm}{0.00990}} \par{\scriptsize 8} & {\small 1.6}\par{\scriptsize\parbox{1.0cm}{0.00990}} \par{\scriptsize 8} & {\small 1.6}\par{\scriptsize\parbox{1.0cm}{0.00990}} \par{\scriptsize 8} & {\small 1.6}\par{\scriptsize\parbox{1.0cm}{0.00990}} \par{\scriptsize 8} & \\
\hline bday large \par 1+2+special box & {\small 0.5}\par{\scriptsize\parbox{1.0cm}{1}} \par{\scriptsize 0} & {\small 0.5}\par{\scriptsize\parbox{1.0cm}{1}} \par{\scriptsize 0} & {\small 0.4}\par{\scriptsize\parbox{1.0cm}{1.61e-4}} \par{\scriptsize 0} & {\small 0.4}\par{\scriptsize\parbox{1.0cm}{1.61e-4}} \par{\scriptsize 0} & {\small 0.4}\par{\scriptsize\parbox{1.0cm}{1.61e-4}} \par{\scriptsize 0} & {\small 0.4}\par{\scriptsize\parbox{1.0cm}{1.61e-4}} \par{\scriptsize 0} & {\small 0.5}\par{\scriptsize\parbox{1.0cm}{1.61e-4}} \par{\scriptsize 0} & {\small 0.5}\par{\scriptsize\parbox{1.0cm}{3.08e-4}} \par{\scriptsize 0} & {\small 0.5}\par{\scriptsize\parbox{1.0cm}{3.08e-4}} \par{\scriptsize 0} & {\small 0.5}\par{\scriptsize\parbox{1.0cm}{3.08e-4}} \par{\scriptsize 0} & {\small 0.5}\par{\scriptsize\parbox{1.0cm}{1.76e-4}} \par{\scriptsize 0} & {\small 0.5}\par{\scriptsize\parbox{1.0cm}{3.08e-4}} \par{\scriptsize 0} & {\small 0.5}\par{\scriptsize\parbox{1.0cm}{3.08e-4}} \par{\scriptsize 0} & {\small 0.5}\par{\scriptsize\parbox{1.0cm}{3.08e-4}} \par{\scriptsize 0} & {\small 0.5}\par{\scriptsize\parbox{1.0cm}{1.76e-4}} \par{\scriptsize 0} & {\small 0.5}\par{\scriptsize\parbox{1.0cm}{3.08e-4}} \par{\scriptsize 0} & {\small 0.5}\par{\scriptsize\parbox{1.0cm}{1.47e-4}} \par{\scriptsize 0} & \\
\hline bday large \par 1+2+special octalatte & {\small 3.8}\par{\scriptsize\parbox{1.0cm}{1}} \par{\scriptsize 10} & {\small 6.3}\par{\scriptsize\parbox{1.0cm}{1}} \par{\scriptsize 12} & {\small 8.5}\par{\scriptsize\parbox{1.0cm}{1.61e-4}} \par{\scriptsize 12} & {\small 8.5}\par{\scriptsize\parbox{1.0cm}{1.61e-4}} \par{\scriptsize 10} & {\small 7.6}\par{\scriptsize\parbox{1.0cm}{1.61e-4}} \par{\scriptsize 11} & {\small 11.6}\par{\scriptsize\parbox{1.0cm}{1.61e-4}} \par{\scriptsize 11} & {\small 8.7}\par{\scriptsize\parbox{1.0cm}{1.61e-4}} \par{\scriptsize 11} & {\small 12.3}\par{\scriptsize\parbox{1.0cm}{3.08e-4}} \par{\scriptsize 11} & {\small 10.7}\par{\scriptsize\parbox{1.0cm}{3.08e-4}} \par{\scriptsize 11} & {\small 10.9}\par{\scriptsize\parbox{1.0cm}{3.08e-4}} \par{\scriptsize 11} & {\small 18.5}\par{\scriptsize\parbox{1.0cm}{1.76e-4}} \par{\scriptsize 11} & {\small 18.4}\par{\scriptsize\parbox{1.0cm}{3.08e-4}} \par{\scriptsize 11} & {\small 21.7}\par{\scriptsize\parbox{1.0cm}{3.08e-4}} \par{\scriptsize 11} & {\small 24.3}\par{\scriptsize\parbox{1.0cm}{3.08e-4}} \par{\scriptsize 11} & {\small 28.8}\par{\scriptsize\parbox{1.0cm}{1.76e-4}} \par{\scriptsize 11} & {\small 32.2}\par{\scriptsize\parbox{1.0cm}{3.08e-4}} \par{\scriptsize 11} & {\small 49.3}\par{\scriptsize\parbox{1.0cm}{1.47e-4}} \par{\scriptsize 10} & \\
\hline bday large \par 1+2+special poly & {\small 4.0}\par{\scriptsize\parbox{1.0cm}{1}} \par{\scriptsize 10} & {\small 4.8}\par{\scriptsize\parbox{1.0cm}{1}} \par{\scriptsize 12} & {\small 7.0}\par{\scriptsize\parbox{1.0cm}{1.61e-4}} \par{\scriptsize 12} & {\small 7.1}\par{\scriptsize\parbox{1.0cm}{1.61e-4}} \par{\scriptsize 10} & {\small 9.3}\par{\scriptsize\parbox{1.0cm}{1.61e-4}} \par{\scriptsize 11} & {\small 7.9}\par{\scriptsize\parbox{1.0cm}{1.61e-4}} \par{\scriptsize 11} & {\small 15.5}\par{\scriptsize\parbox{1.0cm}{1.61e-4}} \par{\scriptsize 11} & {\small 8.6}\par{\scriptsize\parbox{1.0cm}{3.08e-4}} \par{\scriptsize 11} & {\small 12.3}\par{\scriptsize\parbox{1.0cm}{3.08e-4}} \par{\scriptsize 11} & {\small 9.5}\par{\scriptsize\parbox{1.0cm}{3.08e-4}} \par{\scriptsize 11} & {\small 16.1}\par{\scriptsize\parbox{1.0cm}{1.76e-4}} \par{\scriptsize 11} & {\small 18.8}\par{\scriptsize\parbox{1.0cm}{3.08e-4}} \par{\scriptsize 11} & {\small 23.1}\par{\scriptsize\parbox{1.0cm}{3.08e-4}} \par{\scriptsize 11} & {\small 25.7}\par{\scriptsize\parbox{1.0cm}{3.08e-4}} \par{\scriptsize 11} & {\small 33.6}\par{\scriptsize\parbox{1.0cm}{1.76e-4}} \par{\scriptsize 11} & {\small 33.7}\par{\scriptsize\parbox{1.0cm}{3.08e-4}} \par{\scriptsize 11} & {\small 50.2}\par{\scriptsize\parbox{1.0cm}{1.47e-4}} \par{\scriptsize 10} & \\
\hline \hline pizza box & {\small 0.1}\par{\scriptsize\parbox{1.0cm}{1}} \par{\scriptsize 0} & {\small 0.1}\par{\scriptsize\parbox{1.0cm}{1}} \par{\scriptsize 0} & {\small 0.1}\par{\scriptsize\parbox{1.0cm}{1}} \par{\scriptsize 0} & {\small 0.1}\par{\scriptsize\parbox{1.0cm}{1}} \par{\scriptsize 0} & {\small 0.1}\par{\scriptsize\parbox{1.0cm}{1}} \par{\scriptsize 0} & {\small 0.2}\par{\scriptsize\parbox{1.0cm}{1}} \par{\scriptsize 0} & {\small 0.2}\par{\scriptsize\parbox{1.0cm}{1}} \par{\scriptsize 0} & {\small 0.2}\par{\scriptsize\parbox{1.0cm}{1}} \par{\scriptsize 0} & {\small 0.2}\par{\scriptsize\parbox{1.0cm}{1}} \par{\scriptsize 0} & {\small 0.2}\par{\scriptsize\parbox{1.0cm}{1}} \par{\scriptsize 0} & {\small 0.2}\par{\scriptsize\parbox{1.0cm}{1}} \par{\scriptsize 0} & {\small 0.3}\par{\scriptsize\parbox{1.0cm}{1}} \par{\scriptsize 0} & {\small 0.3}\par{\scriptsize\parbox{1.0cm}{1}} \par{\scriptsize 0} & {\small 0.3}\par{\scriptsize\parbox{1.0cm}{1}} \par{\scriptsize 0} & {\small 0.4}\par{\scriptsize\parbox{1.0cm}{1}} \par{\scriptsize 0} & {\small 0.4}\par{\scriptsize\parbox{1.0cm}{1}} \par{\scriptsize 0} & {\small 0.4}\par{\scriptsize\parbox{1.0cm}{1}} \par{\scriptsize 0} & \\
\hline pizza octalatte & ? \par ? & {\small 35.2}\par{\scriptsize\parbox{1.0cm}{1}} \par{\scriptsize 29} & {\small 45.6}\par{\scriptsize\parbox{1.0cm}{1}} \par{\scriptsize 30} & {\small 42.2}\par{\scriptsize\parbox{1.0cm}{1.63e-9}} \par{\scriptsize 30} & {\small 73.9}\par{\scriptsize\parbox{1.0cm}{1}} \par{\scriptsize 33} & {\small 59.5}\par{\scriptsize\parbox{1.0cm}{4.60e-10}} \par{\scriptsize 30} & {\small 88.4}\par{\scriptsize\parbox{1.0cm}{4.60e-10}} \par{\scriptsize 36} & {\small 79.3}\par{\scriptsize\parbox{1.0cm}{2.14e-10}} \par{\scriptsize 30} & {\small 74.0}\par{\scriptsize\parbox{1.0cm}{2.14e-10}} \par{\scriptsize 30} & {\small 80.2}\par{\scriptsize\parbox{1.0cm}{2.14e-10}} \par{\scriptsize 30} & {\small 98.0}\par{\scriptsize\parbox{1.0cm}{1.08e-10}} \par{\scriptsize 30} & {\small 120.1}\par{\scriptsize\parbox{1.0cm}{6.00e-11}} \par{\scriptsize 27} & {\small 147.3}\par{\scriptsize\parbox{1.0cm}{6.00e-11}} \par{\scriptsize 29} & {\small 165.6}\par{\scriptsize\parbox{1.0cm}{6.00e-11}} \par{\scriptsize 27} & {\small 185.2}\par{\scriptsize\parbox{1.0cm}{6.00e-11}} \par{\scriptsize 26} & {\small 209.0}\par{\scriptsize\parbox{1.0cm}{6.00e-11}} \par{\scriptsize 26} & {\small 209.0}\par{\scriptsize\parbox{1.0cm}{6.00e-11}} \par{\scriptsize 26} & \\
\hline pizza poly & {\small 204.4}\par{\scriptsize\parbox{1.0cm}{1}} \par{\scriptsize 45} & {\small 89.1}\par{\scriptsize\parbox{1.0cm}{1}} \par{\scriptsize 40} & {\small 84.5}\par{\scriptsize\parbox{1.0cm}{1}} \par{\scriptsize 38} & {\small 78.5}\par{\scriptsize\parbox{1.0cm}{8.66e-10}} \par{\scriptsize 40} & {\small 2161.9}\par{\scriptsize\parbox{1.0cm}{4.95e-10}} \par{\scriptsize 80} & {\small 71.3}\par{\scriptsize\parbox{1.0cm}{1.50e-10}} \par{\scriptsize 36} & {\small 91.7}\par{\scriptsize\parbox{1.0cm}{1.50e-10}} \par{\scriptsize 33} & {\small 72.7}\par{\scriptsize\parbox{1.0cm}{1.37e-10}} \par{\scriptsize 27} & {\small 453.6}\par{\scriptsize\parbox{1.0cm}{1.37e-10}} \par{\scriptsize 66} & {\small 79.2}\par{\scriptsize\parbox{1.0cm}{1.37e-10}} \par{\scriptsize 27} & {\small 110.1}\par{\scriptsize\parbox{1.0cm}{6.00e-11}} \par{\scriptsize 29} & {\small 120.4}\par{\scriptsize\parbox{1.0cm}{6.00e-11}} \par{\scriptsize 26} & {\small 145.0}\par{\scriptsize\parbox{1.0cm}{6.00e-11}} \par{\scriptsize 26} & {\small 168.3}\par{\scriptsize\parbox{1.0cm}{6.00e-11}} \par{\scriptsize 26} & {\small 190.9}\par{\scriptsize\parbox{1.0cm}{6.00e-11}} \par{\scriptsize 26} & {\small 217.8}\par{\scriptsize\parbox{1.0cm}{6.00e-11}} \par{\scriptsize 26} & {\small 212.8}\par{\scriptsize\parbox{1.0cm}{6.00e-11}} \par{\scriptsize 26} & \\
\hline \hline travel box & {\small 0.2}\par{\scriptsize\parbox{1.0cm}{1}} \par{\scriptsize 0} & {\small 0.2}\par{\scriptsize\parbox{1.0cm}{1}} \par{\scriptsize 0} & {\small 0.2}\par{\scriptsize\parbox{1.0cm}{1}} \par{\scriptsize 0} & {\small 0.2}\par{\scriptsize\parbox{1.0cm}{1}} \par{\scriptsize 0} & {\small 0.2}\par{\scriptsize\parbox{1.0cm}{1}} \par{\scriptsize 0} & {\small 0.2}\par{\scriptsize\parbox{1.0cm}{1}} \par{\scriptsize 0} & {\small 0.2}\par{\scriptsize\parbox{1.0cm}{0.02778}} \par{\scriptsize 0} & {\small 0.2}\par{\scriptsize\parbox{1.0cm}{0.01389}} \par{\scriptsize 0} & {\small 0.3}\par{\scriptsize\parbox{1.0cm}{0.01389}} \par{\scriptsize 0} & {\small 0.3}\par{\scriptsize\parbox{1.0cm}{0.00417}} \par{\scriptsize 0} & {\small 0.4}\par{\scriptsize\parbox{1.0cm}{0.00370}} \par{\scriptsize 0} & {\small 0.4}\par{\scriptsize\parbox{1.0cm}{0.00253}} \par{\scriptsize 0} & {\small 0.4}\par{\scriptsize\parbox{1.0cm}{5.05e-4}} \par{\scriptsize 0} & {\small 0.5}\par{\scriptsize\parbox{1.0cm}{5.05e-4}} \par{\scriptsize 0} & {\small 0.4}\par{\scriptsize\parbox{1.0cm}{5.05e-4}} \par{\scriptsize 0} & {\small 0.5}\par{\scriptsize\parbox{1.0cm}{5.05e-4}} \par{\scriptsize 0} & {\small 0.4}\par{\scriptsize\parbox{1.0cm}{5.05e-4}} \par{\scriptsize 0} & \\
\hline travel octalatte & {\small 9.9}\par{\scriptsize\parbox{1.0cm}{1}} \par{\scriptsize 20} & {\small 5.9}\par{\scriptsize\parbox{1.0cm}{1}} \par{\scriptsize 17} & {\small 10.3}\par{\scriptsize\parbox{1.0cm}{1}} \par{\scriptsize 21} & {\small 8.3}\par{\scriptsize\parbox{1.0cm}{1}} \par{\scriptsize 16} & {\small 14.0}\par{\scriptsize\parbox{1.0cm}{1}} \par{\scriptsize 21} & {\small 12.7}\par{\scriptsize\parbox{1.0cm}{1}} \par{\scriptsize 16} & {\small 16.1}\par{\scriptsize\parbox{1.0cm}{0.01667}} \par{\scriptsize 17} & {\small 17.0}\par{\scriptsize\parbox{1.0cm}{0.00833}} \par{\scriptsize 16} & {\small 22.0}\par{\scriptsize\parbox{1.0cm}{0.00833}} \par{\scriptsize 16} & {\small 23.3}\par{\scriptsize\parbox{1.0cm}{0.00417}} \par{\scriptsize 17} & {\small 38.2}\par{\scriptsize\parbox{1.0cm}{0.00247}} \par{\scriptsize 20} & {\small 45.8}\par{\scriptsize\parbox{1.0cm}{0.00152}} \par{\scriptsize 17} & {\small 51.4}\par{\scriptsize\parbox{1.0cm}{5.05e-4}} \par{\scriptsize 18} & {\small 62.7}\par{\scriptsize\parbox{1.0cm}{5.05e-4}} \par{\scriptsize 20} & {\small 68.8}\par{\scriptsize\parbox{1.0cm}{5.05e-4}} \par{\scriptsize 19} & {\small 76.0}\par{\scriptsize\parbox{1.0cm}{5.05e-4}} \par{\scriptsize 18} & {\small 120.7}\par{\scriptsize\parbox{1.0cm}{5.05e-4}} \par{\scriptsize 16} & \\
\hline travel poly & {\small 210.0}\par{\scriptsize\parbox{1.0cm}{1}} \par{\scriptsize 34} & {\small 9.8}\par{\scriptsize\parbox{1.0cm}{1}} \par{\scriptsize 19} & {\small 15.2}\par{\scriptsize\parbox{1.0cm}{1}} \par{\scriptsize 23} & {\small 11.9}\par{\scriptsize\parbox{1.0cm}{1}} \par{\scriptsize 18} & {\small 23.0}\par{\scriptsize\parbox{1.0cm}{1}} \par{\scriptsize 25} & {\small 18.9}\par{\scriptsize\parbox{1.0cm}{1}} \par{\scriptsize 22} & {\small 31.7}\par{\scriptsize\parbox{1.0cm}{0.00556}} \par{\scriptsize 28} & {\small 21.8}\par{\scriptsize\parbox{1.0cm}{0.00278}} \par{\scriptsize 22} & {\small 37.1}\par{\scriptsize\parbox{1.0cm}{0.00278}} \par{\scriptsize 28} & {\small 29.2}\par{\scriptsize\parbox{1.0cm}{0.00278}} \par{\scriptsize 22} & {\small 57.5}\par{\scriptsize\parbox{1.0cm}{0.00123}} \par{\scriptsize 22} & {\small 60.5}\par{\scriptsize\parbox{1.0cm}{0.00101}} \par{\scriptsize 22} & {\small 66.7}\par{\scriptsize\parbox{1.0cm}{5.05e-4}} \par{\scriptsize 19} & {\small 71.3}\par{\scriptsize\parbox{1.0cm}{5.05e-4}} \par{\scriptsize 25} & {\small 81.0}\par{\scriptsize\parbox{1.0cm}{5.05e-4}} \par{\scriptsize 20} & {\small 83.0}\par{\scriptsize\parbox{1.0cm}{5.05e-4}} \par{\scriptsize 20} & {\small 121.0}\par{\scriptsize\parbox{1.0cm}{5.05e-4}} \par{\scriptsize 16} & \\
\hline \hline
\end{tabular}
}
\addtocounter{table}{-1}
\caption{Query evaluation benchmarks}
\label{fig:bench_table}
\end{table*}
