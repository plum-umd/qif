\todo{Description of dynamic security check.  How we implement it.
  Proof that it does not reveal information.}

Security policies based on belief are very suitable at bounding the
probabilities of certain bad outcomes occurring; if a rational
adversary believes your credit card number is what your actual real
number is with a probability $ p $, then, being rational, have an
expected probability $ p $ of guessing the correct number.

\todo{more about this, alleviating the problem of determining policy parameters}

\paragraph{Relative entropy}

\paragraph{Min-entropy}

\paragraph{Soundness}

Knowledge-based policies need not consider the exact belief to be
sound. Policies based on relative entropy or min-entropy both admit
means of sound knowledge estimation. A method that overestimates the
probability of any hypothesis, and can maintain this even after belief
revision, will be sound -- a policy decision will be rejected if the
policy would have rejected on the exact belief.

\todo{demonstrate this point}

\paragraph{Entropy}
