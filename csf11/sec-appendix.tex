\appendices

\vspace*{.1in}
\section{Concrete probabilistic semantics}
\label{appendix:concrete}

Here we briefly explain the concrete probabilistic semantics given in
Figure~\ref{fig-sem-nondet2-core}.  More details can be found in
Clarkson et al.~\cite{clarkson09quantifying}.

The semantics of $\sskip$ is straightforward: it is the identity on
distributions.  The semantics of sequences $\sseq{\stmt_1}{\stmt_2}$
is also straightforward: the distribution that results from executing
$\stmt_1$ with $\delta$ is given as input to $\stmt_2$ to produce
the result.

The semantics of assignment is $\delta \bparen{x \ra \aexp}$, which is
defined as follows: 
$$ \delta \bparen{x \ra \aexp} \defeq \lambda \sigma \lsep \sum_{\tau \; | \; \tau
  \bparen{x \ra \eeval{\aexp}{\tau}} = \sigma} \delta (\tau) $$ In
words, the result of substituting an expression $\aexp$ for $x$ is a
distribution where state $\sigma$ is given a probability that is the
sum of the probabilities of all states $\tau$ that are equal to
$\sigma$ when $x$ is mapped to the distribution on $\aexp$ in $\tau$.
For implementation purposes, it will be useful to consider separately the
case where assignment is invertible.

When $x \ra \aexp$ is an invertible transformation, the formula for assignment can be simplified to the following, where $x \ra \aexp'$ is the inverse of $x \ra \aexp$.
\[
\delta \bparen{x \ra \aexp} \defeq \lambda \sigma \lsep \delta (\sigma\bparen{x \ra \eeval{\aexp'}{\sigma}})
\]

When $x \ra \aexp$ is not invertible, the original definition is equivalent to a projection followed by an assignment.  Let $V' = \dom{\delta} - \{x\}$ and let $\delta' = \project{\delta}{V'}$.  Then we have the following for a non-invertible assignment.
\[\delta \bparen{x \ra \aexp} \defeq \lambda \sigma \lsep \aif \sigma(x) = \eeval{E}{\sigma}\athen
\delta'(\project{\sigma}{V'}) \aelse 0 \]
In the appendix, we show that this definition by cases is
equivalent to the original definition (Theorem \ref{thm:pp:assign:eqdef}).

The semantics for conditionals makes use of two operators on
distributions which we now define.  First, given distributions
$\delta_1$ and $\delta_2$ we define the \emph{distribution sum} as
follows:
$$ \delta_1 + \delta_2 \defeq \lambda \sigma \lsep \delta_1(\sigma) +
\delta_2(\sigma) $$ In words, the probability mass for a given state
$\sigma$ of the summed distribution is just the sum of the masses from
the input distributions for $\sigma$.  Second, given a distribution
$\delta$ and a boolean expression $\bexp$, we define the
\emph{distribution conditioned on $\bexp$} to be
$$ \dcond{\delta}{\bexp} \defeq \lambda \sigma \lsep \aif \eeval{\bexp}{\sigma} \athen
\delta(\sigma) \aelse 0 $$
In short, the resulting distribution retains only the probability mass
from $\delta$ for states $\sigma$ in which $\bexp$
holds.

With these two operators, the semantics of conditionals can be stated
simply: the resulting distribution is the sum of the distributions of
the two branches, where the first branch's distribution is conditioned
on $\bexp$ being true, while the second branch's distribution is
conditioned on $\bexp$ being false.

The semantics for probabilistic conditionals like that of conditionals
but makes use of \emph{distribution scaling}, which is defined as
follows: given $\delta$ and some scalar $p$ in $[0,1]$, we have
$$ p \cdot \delta \defeq \lambda \sigma \lsep p \cdot \delta(\sigma) $$
In short, the probability ascribed to each state is just the
probability ascribed to that state by $\delta$ but multiplied by $p$.
For probabilistic conditionals, we sum the distributions of the two
branches, scaling them according to the odds $q$ and $1 - q$.

The semantics of a single iteration of a while loop is
essentially that of $\sif{B}{S}{\sskip}$ and the semantics of the
entire loop is the fixed point of a function that composes the
distributions produced by each iteration.  That such a fixed point exists
is proved by Clarkson et al.~\cite{clarkson09quantifying}.

Finally, the semantics of $\suniform{x}{n_1}{n_2}$, introduced in
Section~\ref{sec:absinterp} is given as
$$
\begin{array}{rcl}
\pevalp{\suniform{x}{n_1}{n_2}}{\delta} & = &
\paren{\project{\delta}{V - \set{x}}} \times \delta'
\end{array}
$$
Where $ V $ is the set of variables of $ \delta $, and $ \delta' $ is
defined as follows.
$$ \delta ' = \lambda \sigma \lsep \aif n_1 \leq \sigma(x) \leq n_2 \athen
\frac{1}{n_2-n_1+1} \aelse 0 $$

\vspace*{.1in}
\section{Alternative (flawed) threshold security policy}
\label{appendix:flawed}

As an alternative to Definition~\ref{def:threshold}, suppose we used the
following instead:
$$\forall \sigma_L \in
\nzset{\project{\delta'}{L}}.\,
(\normal{\project{\dcond{\delta'}{\sigma_L}}{H}})(\sigma_H) \leq t$$
Here is an example that illustrates why this definition is not safe, as it
could underestimate the information a querier can learn.

Suppose Bob's threshold for his birth year $\var{byear}$ is $t =
0.05$.  He models a social networking site $X$ as believing his age is
more likely between 20 and 40 than between 40 and 60, e.g., $1971 \leq
\var{byear} < 1991$ with probability $0.6$ (thus, $0.03$ per
possibility) and $1951 \leq \var{byear} < 1971$ with probability $0.4$
(thus, $0.02$ per possibility).  If user Bob was born in 1965, then
$X$'s believes his is actual birth year not as likely a more recent
year, say 1975; in any case $X$ does not currently believe any
possibility above Bob's threshold.  Now suppose $X$ submits program
$S$ that determines whether Bob's birth year is even.  The revised
belief will include only even (when $\var{output} = \strue$) or odd
(when $\var{output} = \sfalse$) birthdays, increasing the likelihood
of years in the range $[1971,1991)$ to be $0.06$ per point, and the
likelihood of years in the range $[1951,1971)$ to be $0.04$ per point.
Bob's birthday is 1965, and its probability $0.04$ is less than $t$,
so according to the flawed definition the agent would respond to this
query.  But if this query result is returned, $X$ will see that there
are ten possibilities of birth year that are above Bob's threshold.
$X$ can deduce that none of these possibilities is Bob's actual birth
year, or else the query would have been rejected.  Excluding these
possibilities, he knows that Bob's birth year is one of ten
possibilities between 1951 and 1971 ascribing to each a probability
$0.1$ which exceeds Bob's threshold of $0.05$.

\vspace*{.1in}
\section{Example queries} \label{appendix:queries}

%rendering latex
$$ \ssecretk: $$
\begin{displaymath}{\small
\begin{array}{l}
  \sassign{country}{1}\; ;\\
  \sassign{birth\text{\_}year}{1983}\; ;\\
  \sassign{completed\text{\_}school\text{\_}type}{4}\; ;\\
  \sassign{language}{5}\\
\end{array}
}
\end{displaymath}
$$ \sbeliefk: $$
\begin{displaymath}{\small
\begin{array}{l}
  \suniform{country}{1}{200}\; ;\\
  \suniform{birth\text{\_}year}{1900}{2011}\; ;\\
  \suniform{language}{1}{50}\; ;\\
  \suniform{completed\text{\_}school\text{\_}type}{0}{5}\\
\end{array}
}
\end{displaymath}

$$ \squerydefk \; travel \; : \; \; \ra \; output$$
\begin{displaymath}{\small
\begin{array}{l}
  \sifk \; country \; = \; 1 \; \vee \; country \; = \; 3 \; \vee \; country \; = \; 8 \; \vee \; country \; = \; 10 \; \vee \; country \; = \; 18 \; \sthenk\\
  \;\;\;\sassign{main\text{\_}country}{1}\\
  \selsek\\
  \;\;\;\sassign{main\text{\_}country}{0}\; ;\\
  \sifk \; country \; = \; 169 \; \vee \; country \; = \; 197 \; \vee \; country \; = \; 194 \; \vee \; country \; = \; 170 \; \vee \; country \; = \; 206 \; \vee \; country \; = \; 183 \; \vee \; country \; = \; 188 \; \sthenk\\
  \;\;\;\sassign{island}{1}\\
  \selsek\\
  \;\;\;\sassign{island}{0}\; ;\\
  \sassign{age}{\ebinop{-}{2010}{birth\text{\_}year}}\; ;\\
  \sifk \; language \; = \; 1 \; \wedge \; main\text{\_}country \; = \; 1 \; \vee \; island \; = \; 1 \; \wedge \; age \; \geq \; 21 \; \wedge \; completed\text{\_}school\text{\_}type \; \geq \; 4 \; \sthenk\\
  \;\;\;\sassign{output}{1}\\
  \selsek\\
  \;\;\;\sassign{output}{0}\\
\end{array}
}
\end{displaymath}


$$ \squeryk \; travel \; : $$
\begin{displaymath}{\small
\begin{array}{l}
  \sskip\\
\end{array}
}
\end{displaymath}

no errors


We provide here the queries and prebeliefs we used for the experiments
in Section~\ref{sec:impl}. The queries are described as functions
from some set of inputs to some set of outputs. The exact syntax is as
follows.
$$
\begin{array}{l}
\squerydefk \; queryname \; in_1 \cdots in_n \ra out_1 \cdots out_m \; : \\
\;\; querybody \\
\end{array}
$$

To specify a query invocation we use the following syntax.
$$
\begin{array}{l}
\squeryk \; queryname \; : \\
\;\; in_1 \; := \; val_1; \\
\;\; \cdots \\
\;\; in_n \; := \; val_n
\end{array}
$$

Each experiment must also specify the values of the secrets being
queried, and the querier's prebelief.  Each specification is a merely
a program that sets the values of these variables.  For the actual
secret values this program begins with the declaration $ \ssecretk$;
the resulting state of executing program is taken to be the secret
state.  The program to set the prebelief begins $ \sbeliefk$ and has
the same format; note that this program will use $\spifk$ or
$\suniform{x}{n_1}{n_2}$ to give secrets different possible values
with different probabilities.  

We now give the content of the queries used in the experiments.

\subsubsection{Birthday}
For the small stateset size birthday experiments we used the following
secret and prebelief.
$$ \ssecretk: $$
\begin{displaymath}{\small
\begin{array}{l}
  \sassign{s\text{\_}bday}{270}\; ;\\
  \sassign{s\text{\_}byear}{1980}\\
\end{array}
}\end{displaymath}
$$ \sbeliefk: $$
\begin{displaymath}{\small
\begin{array}{l}
  \suniform{s\text{\_}bday}{0}{364}\; ;\\
  \suniform{s\text{\_}byear}{1956}{1992}\\
\end{array}
}\end{displaymath}

The two queries used were as follows.

$$ \squerydefk \; bday \; : \;c\text{\_}day \; \ra \; output$$
\begin{displaymath}{\small
\begin{array}{l}
  \sifk \; \ebinop{\wedge}{\ebinop{\geq}{s\text{\_}bday}{c\text{\_}day}}{\ebinop{>}{\ebinop{+}{c\text{\_}day}{7}}{s\text{\_}bday}} \; \sthenk\\
  \;\;\;\sassign{output}{1}\\
  \selsek\\
  \;\;\;\sassign{output}{0}\\
\end{array}
}\end{displaymath}

$$ \squerydefk \; spec \; : \;c\text{\_}year \; \ra \; output$$
\begin{displaymath}{\small
\begin{array}{l}
  \sassign{age}{\ebinop{-}{c\text{\_}year}{s\text{\_}byear}}\; ;\\
  \sifk \; \ebinop{\vee}{\ebinop{\vee}{\ebinop{\vee}{\ebinop{\vee}{\ebinop{=}{age}{10}}{\ebinop{=}{age}{20}}}{\ebinop{=}{age}{30}}}{\ebinop{=}{age}{40}}}{\ebinop{=}{age}{50}} \; \sthenk\\
  \;\;\;\sassign{output\text{\_}temp}{1}\\
  \selsek\\
  \;\;\;\sassign{output\text{\_}temp}{0}\; ;\\
  \spifk\; 1/10 \; \sthenk\\
  \;\;\;\sassign{output}{1}\\
  \selsek\\
  \;\;\;\sassign{output}{output\text{\_}temp}\\
\end{array}
}\end{displaymath}

The statistics shown include the time spent processing this initial
setup as well as the following sequences of queries.

\begin{itemize}
\item{} A single bday query alone.
$$ \squeryk \; bday \; : $$
\begin{displaymath}{\small
\begin{array}{l}
  \sassign{c\text{\_}day}{260}\\
\end{array}
}\end{displaymath}

\item{} Two bday queries.
$$ \squeryk \; bday \; : $$
\begin{displaymath}{\small
\begin{array}{l}
  \sassign{c\text{\_}day}{261}\\
\end{array}
}\end{displaymath}

\item{} Two bday queries followed by a spec query.
$$ \squeryk \; spec \; : $$
\begin{displaymath}{\small
\begin{array}{l}
  \sassign{c\text{\_}year}{2011}\\
\end{array}
}\end{displaymath}

\end{itemize}

\subsubsection{Birthday (large)}

For the larger statespace birthday example we used the following
secret and prebelief generators.
$$ \ssecretk: $$
\begin{displaymath}{\small
\begin{array}{l}
  \sassign{s\text{\_}bday}{270}\; ;\\
  \sassign{s\text{\_}byear}{1980}\\
\end{array}
}\end{displaymath}
$$ \sbeliefk: $$
\begin{displaymath}{\small
\begin{array}{l}
  \suniform{s\text{\_}bday}{0}{364}\; ;\\
  \suniform{s\text{\_}byear}{1910}{2010}\\
\end{array}
}\end{displaymath}
The queries used were identical to the ones for the smaller statespace
birthday example.

\subsubsection{Pizza}

The pizza example is slightly more complicated, especially in the
construction of the prebelief.  This example models a targeted
Facebook advertisement for a local pizza shop.  There are four
relevant secret values.  The level of school currently being attended
by the Facebook user
is given by \verb|s_in_school_type|, which is an integer
ranging from 0 (not in school) to 6 (Ph.D. program).  Birth year
is as before and \verb|s_address_lat| and \verb|s_address_long|
give the latitude and longitude of the user's home address (represented as
decimal degrees scaled by a factor of
$10^6$ and converted to an integer).

The initial belief models the fact that each subsequent level of
education is less likely and also captures the correlation between current
educational level and age.  For example, a user is given an approximately 0.05 chance of
currently being an undergraduate in college, and college attendees are assumed to be born
no later than 1985 (whereas elementary school students may be born as
late as 2002).

rendering latex
$$ \ssecretk: $$
\begin{displaymath}{\small
\begin{array}{l}
  \sassign{s\text{\_}in\text{\_}school\text{\_}type}{4}\; ;\\
  \sassign{s\text{\_}birth\text{\_}year}{1983}\; ;\\
  \sassign{s\text{\_}address\text{\_}lat}{39003178}\; ;\\
  \sassign{s\text{\_}address\text{\_}long}{-76958199}\\
\end{array}
}
\end{displaymath}
$$ \sbeliefk: $$
\begin{displaymath}{\small
\begin{array}{l}
  \spifk\; 4/24 \; \sthenk\\
  \;\;\;\suniform{s\text{\_}in\text{\_}school\text{\_}type}{1}{1}\; ;\\
  \;\;\;\suniform{s\text{\_}birth\text{\_}year}{1998}{2002}\\
  \selsek\\
  \;\;\;\spifk\; 3/19 \; \sthenk\\
  \;\;\;\;\;\;\suniform{s\text{\_}in\text{\_}school\text{\_}type}{2}{2}\; ;\\
  \;\;\;\;\;\;\suniform{s\text{\_}birth\text{\_}year}{1990}{1998}\\
  \;\;\;\selsek\\
  \;\;\;\;\;\;\spifk\; 2/15 \; \sthenk\\
  \;\;\;\;\;\;\;\;\;\suniform{s\text{\_}in\text{\_}school\text{\_}type}{3}{3}\; ;\\
  \;\;\;\;\;\;\;\;\;\suniform{s\text{\_}birth\text{\_}year}{1985}{1992}\\
  \;\;\;\;\;\;\selsek\\
  \;\;\;\;\;\;\;\;\;\spifk\; 1/12 \; \sthenk\\
  \;\;\;\;\;\;\;\;\;\;\;\;\suniform{s\text{\_}in\text{\_}school\text{\_}type}{4}{4}\; ;\\
  \;\;\;\;\;\;\;\;\;\;\;\;\suniform{s\text{\_}birth\text{\_}year}{1980}{1985}\\
  \;\;\;\;\;\;\;\;\;\selsek\\
  \;\;\;\;\;\;\;\;\;\;\;\;\suniform{s\text{\_}in\text{\_}school\text{\_}type}{0}{0}\; ;\\
  \;\;\;\;\;\;\;\;\;\;\;\;\suniform{s\text{\_}birth\text{\_}year}{1900}{1985}\; ;\\
  \suniform{s\text{\_}address\text{\_}lat}{38867884}{39103178}\; ;\\
  \suniform{s\text{\_}address\text{\_}long}{-77058199}{-76825926}\\
\end{array}
}
\end{displaymath}

The query itself targets the pizza advertisement
at users who are either 1) in college, 2) aged 18 to 28, or 3) close to the pizza
shop (within a square region that is 2.5 miles on each side and centered on the pizza
shop).  If any of these conditions are satisfied, then the query returns 1,
indicating that the ad can be displayed.  The full text of the query is given below.

$$ \squerydefk \; pizza \; : \; \; \ra \; output$$
\begin{displaymath}{\small
\begin{array}{l}
  \sifk \; s\text{\_}in\text{\_}school\text{\_}type \; \geq \; 4 \; \sthenk\\
  \;\;\;\sassign{in\text{\_}school}{1}\\
  \selsek\\
  \;\;\;\sassign{in\text{\_}school}{0}\; ;\\
  \sassign{age}{\ebinop{-}{2010}{s\text{\_}birth\text{\_}year}}\; ;\\
  \sifk \; age \; \geq \; 18 \; \wedge \; age \; \leq \; 28 \; \sthenk\\
  \;\;\;\sassign{age\text{\_}criteria}{1}\\
  \selsek\\
  \;\;\;\sassign{age\text{\_}criteria}{0}\; ;\\
  \sassign{lr\text{\_}lat}{38967884}\; ;\\
  \sassign{ul\text{\_}lat}{39003178}\; ;\\
  \sassign{lr\text{\_}long}{-76958199}\; ;\\
  \sassign{ul\text{\_}long}{-76925926}\; ;\\
  \sifk \; s\text{\_}address\text{\_}lat \; \leq \;
  ul\text{\_}lat \; \wedge \\
  \;\;\;\; s\text{\_}address\text{\_}lat \; \geq \;
  lr\text{\_}lat \; \wedge \\
  \;\;\;\; s\text{\_}address\text{\_}long \; \geq \;
  lr\text{\_}long \; \wedge \\
  \;\;\;\; s\text{\_}address\text{\_}long \; \leq \; ul\text{\_}long \; \sthenk\\
  \;\;\;\sassign{in\text{\_}box}{1}\\
  \selsek\\
  \;\;\;\sassign{in\text{\_}box}{0}\; ;\\
  \sifk \; \paren{in\text{\_}school \; = \; 1 \; \vee \;
  age\text{\_}criteria \; = \; 1} \; \wedge \\
  \;\;\;\; in\text{\_}box \; = \; 1 \; \sthenk\\
  \;\;\;\sassign{output}{1}\\
  \selsek\\
  \;\;\;\sassign{output}{0}\\
\end{array}
}
\end{displaymath}

\subsubsection{Photo}

The photo query is a direct encoding of a case study that Facebook
includes on their advertising information
page~\cite{wedding-case-study}.  The advertisement was for CM
Photographics, and targets offers for wedding photography packages at
women between the ages of 24 and 30 who list in their profiles that
they are engaged.  The secret state consists of birth year, as before,
gender (0 indicates male, 1 indicates female), and ``relationship
status,'' which can take on a value from 0 to 9.  Each of these
relationship status values indicates one of the status choices
permitted by the Facebook software.  The example below involves only
four of these values, which are given below.
\begin{description}
\item[0] No answer
\item[1] Single
\item[2] In a relationship
\item[3] Engaged
\end{description}
The secret state and prebelief are as follows.
$$ \ssecretk: $$
\begin{displaymath}{\small
\begin{array}{l}
  \sassign{s\text{\_}birth\text{\_}year}{1983}\; ;\\
  \sassign{s\text{\_}gender}{0}\; ;\\
  \sassign{s\text{\_}relationship\text{\_}status}{0}\\
\end{array}
}
\end{displaymath}
$$ \sbeliefk: $$
\begin{displaymath}{\small
\begin{array}{l}
  \suniform{s\text{\_}birth\text{\_}year}{1900}{2010}\; ;\\
  \suniform{s\text{\_}gender}{0}{1}\; ;\\
  \suniform{s\text{\_}relationship\text{\_}status}{0}{3}\\
\end{array}
}
\end{displaymath}

The query itself is the following.
$$ \squerydefk \; cm\text{\_}advert \; : \; \; \ra \; output$$
\begin{displaymath}{\small
\begin{array}{l}
  \sassign{age}{\ebinop{-}{2010}{s\text{\_}birth\text{\_}year}}\; ;\\
  \sifk \; age \; \geq \; 24 \; \wedge \; age \; \leq \; 30 \; \sthenk\\
  \;\;\;\sassign{age\text{\_}sat}{1}\\
  \selsek\\
  \;\;\;\sassign{age\text{\_}sat}{0}\; ;\\
  \sifk \; s\text{\_}gender \; = \; 1 \; \wedge \\
  \;\;\;\; s\text{\_}relationship\text{\_}status \; = \;
  3 \; \wedge \\
  \;\;\;\; age\text{\_}sat \; = \; 1 \; \sthenk\\
  \;\;\;\sassign{output}{1}\\
  \selsek\\
  \;\;\;\sassign{output}{0}\\
\end{array}
}
\end{displaymath}
\vspace{5mm} % !!! formatting

\subsubsection{Travel}

This example is another Facebook advertising case
study~\cite{visitbritain-case-study}.  It is based on an ad campaign
run by Britain's national tourism agency, VisitBritain.  The campaign
targeted English-speaking Facebook users currently residing in
countries with strong ties to the United Kingdom.  They further filtered by
showing the advertisement only to college graduates who were at least 21 years of age.

We modeled this using four secret values: country, birth year, highest
completed education level, and primary language.  As with other
categorical data, we represent language and country using an
enumeration.  We ranked countries by number of Facebook users as
reported by socialbakers.com.  This resulted in the US being country
number 1 and the UK being country 3.  To populate the list of
countries with ``strong connections'' to the UK, we took a list of
former British colonies.  For the language attribute, we consider a
50-element enumeration where 0 indicates ``no answer'' and 1 indicates
``English'' (other values appear in the prebelief but are not used in
the query).

$$ \ssecretk: $$
\begin{displaymath}{\small
\begin{array}{l}
  \sassign{country}{1}\; ;\\
  \sassign{birth\text{\_}year}{1983}\; ;\\
  \sassign{completed\text{\_}school\text{\_}type}{4}\; ;\\
  \sassign{language}{5}\\
\end{array}
}
\end{displaymath}

$$ \sbeliefk: $$
\begin{displaymath}{\small
\begin{array}{l}
  \suniform{country}{1}{200}\; ;\\
  \suniform{birth\text{\_}year}{1900}{2011}\; ;\\
  \suniform{language}{1}{50}\; ;\\
  \suniform{completed\text{\_}school\text{\_}type}{0}{5}\\
\end{array}
}
\end{displaymath}

$$ \squerydefk \; travel \; : \; \; \ra \; output$$
\begin{displaymath}{\small
\begin{array}{l}
  \sifk \; country \; = \; 1 \; \vee \; country \; = \; 3 \; \vee \\
  \;\;\;\; country \; = \; 8 \; \vee \; country \; = \; 10 \; \vee \\
  \;\;\;\; country \; = \; 18 \; \sthenk\\
  \;\;\;\sassign{main\text{\_}country}{1}\\
  \selsek\\
  \;\;\;\sassign{main\text{\_}country}{0}\; ;\\
  \sifk \; country \; = \; 169 \; \vee \; country \; = \;
  197 \; \vee \\
  \;\;\;\; country \; = \; 194 \; \vee \; country \; = \;
  170 \; \vee \\
  \;\;\;\; country \; = \; 206 \; \vee \; country \; = \;
  183 \; \vee \\
  \;\;\;\; country \; = \; 188 \; \sthenk\\
  \;\;\;\sassign{island}{1}\\
  \selsek\\
  \;\;\;\sassign{island}{0}\; ;\\
  \sassign{age}{\ebinop{-}{2010}{birth\text{\_}year}}\; ;\\
  \sifk \; language \; = \; 1 \; \wedge \\
  \;\;\;\; \paren{main\text{\_}country \; = \; 1 \; \vee \; island \;
  = \; 1} \; \wedge \\
  \;\;\;\; age \; \geq \; 21 \; \wedge \\
  \;\;\;\; completed\text{\_}school\text{\_}type \; \geq \; 4 \; \sthenk\\
  \;\;\;\sassign{output}{1}\\
  \selsek\\
  \;\;\;\sassign{output}{0}\\
\end{array}
}
\end{displaymath}

%\vspace*{.1in}
\section{Soundness proofs for $ \ppolys $} \label{appendix:proof1}

%\begin{DIFnomarkup}

\subsection{Projection}

The proof of projection relies heavily on splitting up the support of a
distribution into equivalence classes based on the states they project
to. We will have $ \sigma, \sigma' \in \support{\delta} $ belonging to
the same equivalence class iff $ \project{\sigma}{V}
= \project{\sigma'}{V} $. The details are formalized in the following definition.

\begin{definition} \emph{Equivalence classes under projection.}
\begin{itemize}
\item{} $ \eqclass{\sigma_V}{\delta}{V} $ is an equivalence class of
elements of 
$ \support{\delta} $ that project to $ \sigma_V $ (when projected to
variables $ V $). Formally,
$ \eqclass{\sigma_V}{\delta}{V} \defeq \set{\sigma \in \support{\delta} \given \project{\sigma}{V}
= \sigma_V} $.

\item{} $ \eqclassc{\sigma_V}{\delta}{V} $ is a subset of
$ \support{\delta} $ that project to anything but $ \sigma_V $ or formally
$ \eqclassc{\sigma_V}{\delta}{V} \defeq \set{\sigma \in \support{\delta} \given
\project{\sigma}{V} \neq \sigma_V} $. 

\item{} $ \eqclass{\sigma_V}{\getpoly{}}{V} $ is a subset of
$ \pconc{C} $ that project to $ \sigma_V $ (when projected to
variables $ V $). Formally,
$ \eqclass{\sigma_V}{\getpoly{}}{V} \defeq \set{\sigma \in \pconc{\getpoly{}} \given \project{\sigma}{V}
= \sigma_V} $.

\item{} $ \eqclassc{\sigma_V}{\getpoly{}}{V} $ is a subset of
$ \pconc{\getpoly{}} $ that project to anything but $ \sigma_V $ or formally
$ \eqclassc{\sigma_V}{\getpoly{}}{V} \defeq \set{\sigma \in \pconc{\getpoly{}} \given
\project{\sigma}{V} \neq \sigma_V} $.
\end{itemize}
\end{definition}

\begin{remark} \label{thm:pp:project:eqclass} Let $ V \subseteq V' \subseteq \fv{\delta} $, 
$ \sigma \in \support{\delta} $, and
$ \sigma_V, \sigma'_V \in \support{\project{\delta}{V}} $.

\begin{subtheorem} \label{thm:pp:project:eqclass:project}
$ \project{\paren{\project{\sigma}{V'}}}{V} = \project{\sigma}{V} $
\end{subtheorem}

The sets of $ \set{\eqclass{\sigma_V}{\delta}{V}}_{\sigma_V \in \support{\project{\delta}{V}}} $ form a partition of
$ \support{\delta} $, equivalently the following two claims.

\begin{subtheorem} \label{thm:pp:project:eqclass:cover}
$ \support{\delta}
= \bigcup_{\sigma_V \in \support{\project{\delta}{V}}} \eqclass{\sigma_V}{\delta}{V}
$
\end{subtheorem}

\begin{subtheorem} \label{thm:pp:project:eqclass:disjoint}
$ \eqclass{\sigma_V}{\delta}{V} \cap \eqclass{\sigma'_V}{\delta}{V}
= \emptyset $ whenever $ \sigma_V \neq \sigma'_V $
\end{subtheorem}

Likewise, for any $ \sigma_{V} \in \support{\project{\delta}{V}} $,
the sets of
$ \set{\eqclass{\sigma_{V'}}{\delta}{V'}}_{\sigma_{V'} \in \eqclass{\sigma_V}{\project{\delta}{V'}}{V}}
$ form a partition of $ \eqclass{\sigma_{V}}{\delta}{V} $, implying
also the following claim.

\begin{subtheorem} \label{thm:pp:project:eqclass:order}
$ \eqclass{\sigma_V}{\delta}{V}
= \bigcup_{\sigma_{V'} \in \eqclass{\sigma_V}{\project{\delta}{V'}}{V}} \eqclass{\sigma_{V'}}{\delta}{V'}
$ 
\end{subtheorem}

The equivalence classes in terms of the concrete support sets as
related to the abstract support sets are expressed in the following
manner.

\begin{subtheorem} \label{thm:pp:project:eqclass:subsets}
$ \eqclass{\sigma_V}{\delta}{V} \subseteq \eqclass{\sigma_V}{\getpoly{}}{V}
$ and 
$ \eqclassc{\sigma_V}{\delta}{V} \subseteq \eqclassc{\sigma_V}{\getpoly{}}{V}
$ whenever $ \support{\delta} \subseteq \pconc{\getpoly{}} $
\end{subtheorem}

Finally, the concrete projection operation can be rewritten in terms
of the equivalence classes, a fact we will repeatedly use in the
proofs to follow without explicitly stating it.

\begin{subtheorem} \label{thm:pp:project:eqclass:projection}
$ \project{\delta}{V} = \lambda \sigma_V \lsep \sum_{\sigma \in \eqclass{\sigma_V}{\delta}{V}} \delta
(\sigma) $
\end{subtheorem}

\end{remark}

\begin{proof} All of these are merely expansions of the various
definitions involved. Note that the two parts
of \rref{thm:pp:project:eqclass:subsets} are not contradictory as
$ \eqclass{\sigma_V}{\delta}{V} $ and $ \eqclassc{\sigma_V}{\delta}{V}
$ are not set complements of each other when viewed as subsets of
$ \pconc{\poly{}} $, though they are complements when viewed as
subsets of $ \support{\delta} $.
\end{proof}

\begin{lemma}[Conservation of Mass]\label{thm:pp:project:mass} If $ V \subseteq \fv{\delta} $ then $ \pmass{\delta} = \pmass{\project{\delta}{V}} $.
\end{lemma}

\begin{proof} Let us consider the terms of the projected mass sum.
\begin{align*}
\pmass{\project{\delta}{V}} & = \sum_{\sigma_V \in \support{\project{\delta}{V}}} \paren{\sum_{\sigma \in \eqclass{\sigma_V}{\delta}{V}}\delta(\sigma)} \\
 & = \sum_{\sigma \in \support{\delta}} \delta (\sigma) \\ &
 = \pmass{\delta}
\end{align*}

The terms in the double sum are the same as those in the single sum as
all terms of the first are accounted for in the second due
to \rref{thm:pp:project:eqclass:cover} and none are double counted due
to \rref{thm:pp:project:eqclass:disjoint}.
\end{proof}

\begin{definition} \emph{Concrete forget} can be defined in terms of a projection to
all but one variable. That is, $ \forget{x}{\delta}
\defeq \project{\delta}{\fv{\delta} - \set{x}} $. Also,
$ \forget{x_1, \cdots, x_n}{\delta} \defeq \forget{x_2, \cdots,
x_n}{\forget{x_1}{\delta}} $.
\end{definition}

The correspondence between repeated concrete forget and a projection
involving removal of more than one variable will be demonstrated shortly.

\begin{lemma}[Order of Projection]\label{thm:pp:project:order} If $ V \subseteq
V' \subseteq \fv{\delta} $ then
$ \project{\paren{\project{\delta}{V'}}}{V} = \project{\delta}{V} $ .
\end{lemma}

\begin{proof} Let $ \sigma_V \in \project{\delta}{V} $.
\begin{align}
\paren{\project{\paren{\project{\delta}{V'}}}{V}}(\sigma_V)
& = \sum_{\sigma_{V'} \in \eqclass{\sigma_V}{ \project{\delta}{V'}
}{V}} \paren{ \sum_{\sigma \in \eqclass{\sigma_{V'}}{\delta}{V'}
} \delta (\sigma) } \\ &
= \sum_{\sigma \in \bigcup_{\sigma_{V'} \in \eqclass{\sigma_V}{ \project{\delta}{V'}
}{V} }} \delta(\sigma) \\ &
= \sum_{\sigma \in \eqclass{\sigma_V}{\delta}{V} } \delta (\sigma) \\
& = \paren{\project{\sigma}{V}}(\sigma_V)
\end{align}

The collapse of the double sums on (1) to (2) is due to the
correspondence between the terms of the double sum and the single sum
due to \rref{thm:pp:project:eqclass:cover} and \rref{thm:pp:project:eqclass:disjoint}. The equality of the
union of equivalence classes, (2) to (3) is due
to \rref{thm:pp:project:eqclass:order}.
\end{proof}

\begin{corollary}\label{thm:pp:project:forgetseq} $ \project{\delta}{V} = \forget{x_1, \cdots,
x_n}{\delta}$ where $ \fv{\delta} - V = \set{x_1, \cdots, x_n} $.
\end{corollary}

\begin{proof} Let us show this by induction on the size of
$ \vcomp \defeq \fv{\delta} - V $. When $ \setsize{\vcomp} = 0 $ or
$ \setsize{\vcomp} = 1 $, the claim holds vacuously or by
definition of concrete forget, respectively.

Let us assume the claim for $ \setsize{\vcomp} = m - 1 < n $ and consider
the case when $ \setsize{\vcomp} = m \leq n $.

\begin{align*}
\project{\delta}{V}
& = \project{\paren{\project{\delta}{V
\cup \set{x_1}}}}{V} & \mathstep{by \lref{thm:pp:project:order}}\\
& = \project{\forget{x_1}{\delta}}{V} \\
& = \forget{x_2, \cdots,
x_m}{\forget{x_1}{\delta}} & \mathstep{by induction} \\
& = \forget{x_1, \cdots, x_m}{\delta}
\end{align*}

Thus, by induction, the claim holds for $ m = n $.
\end{proof}

\begin{remark}[Counting Variations]\label{thm:pp:project:counting}
Two simple counting arguments are required for the further proofs.

\begin{itemize}
\item[(i)] If $ m $ objects are distributed fully into two bins, with one
of the bins having space for no more than $ a $ objects, then the
other must have at least $ m - a $ objects in it.
\item[(ii)] If $ m $ objects are to be packed into bins of sizes $
a_1, \cdots, a_n $, with $ \sum_{i} a_i \geq m $, the least number of
bins that can be used to fit all $ m $ objects is greater or equal to
$ \ceil{m / a^*} $ where $ a^* \geq \max_i a_i $.
\end{itemize}
\end{remark}

\begin{proof} Part (i) is immediate. For part (ii), consider some
optimal set of bins used to pack the $ m $ objects. This set of
bins would also let one pack $ m $ items assuming each bin had space
for exactly $ a^* $ objects as this is an upper bound on the size of
each bin. Thus the space of solutions to the original packing problem
is a subset of the space of solutions to the altered packing problem
where all bins are increased to fit $ a^* $ items. Thus the solution
for the original cannot use fewer bins than the optimal solution for
the altered problem. For this alternate problem, the minimum number of bins used
to pack all $ m $ items is exactly $ \ceil{m / a^*} $ by a
generalization of the pigeonhole principle.
\end{proof}

\begin{replemma}{lem:pp:forget}[Soundness of Forget]
If $\delta \in \ppconc{\pp{}}$ then $\forget{y}{\delta} \in \ppconc{\forget{y}{\pp{}}}$.
\end{replemma}

\begin{proof} Let $ \delta \in \ppconc{P} $, $ V = \fv{\delta}
- \set{y} $, and $ \delta_2 = \project{\delta}{V}$. By assumption
  $ \delta $ has the following properties.

\begin{align}
\label{thm:pp:forget:given1a} & \support{\delta} \subseteq \pconc{C} \\
\label{thm:pp:forget:given2a}& \smin{} \leq \setsize{\support{\delta}} \leq \smax{} \\
\label{thm:pp:forget:given3a}& \mmin{} \leq \pmass{\delta} \leq \mmax{} \\
\label{thm:pp:forget:given4a}& \forall \sigma \in \support{\delta} \;
. \; \pmin{} \leq \delta(\sigma) \leq \pmax{}
\end{align}

Let $ \pp{2} = \forget{y}{\pp{}} $. $ \pp{2} $ thus has the following
properties.

\begin{align}
& \getpoly{2} = \forget{y}{C} \\
& \pmin{2} = \pmin{} \cdot \maxparen{\minh{y} - \paren{\psize{C}
- \smin{}}, 1} \\
& \pmax{2} = \pmax{} \cdot \minparen{\maxh{y}, \smax{}} \\
& \smin{2} = \ceil{\smin{} / \maxh{y}} \\
& \smax{2} = \minparen{\psize{\getpoly{2}}, \smax{}} \\
& \mmin{2} = \mmin{} \\
& \mmax{2} = \mmax{}
\end{align}

The quantities $ \minh{y} $ and $ \maxh{y} $ are defined to exhibit
the following properties.

\begin{align}
\label{thm:pp:forget:given1b} & \minh{y} \leq \min_{\sigma_V \in \pconc{\getpoly{2}}} \setsize{\eqclass{\sigma_V}{C}{V}} \\
\label{thm:pp:forget:given2b} & \maxh{y} \geq \max_{\sigma_V \in \pconc{\getpoly{2}}} \setsize{\eqclass{\sigma_V}{C}{V}}
\end{align}

To show that $ \delta_2 \in \ppconc{\forget{y}{\pp{}}} $ we need to show the following.

\begin{align}
\label{thm:pp:forget:cond1} & \support{\delta_2} \subseteq \pconc{\getpoly{2}} \\
\label{thm:pp:forget:cond2} & \smin{2} \leq \setsize{\support{\delta_2}} \leq \smax{2} \\
\label{thm:pp:forget:cond3} & \mmin{2} \leq \pmass{\delta_2} \leq \mmax{2} \\
\label{thm:pp:forget:cond4} & \forall \sigma_V \in \support{\delta_2} \qsep \pmin{2} \leq \delta_2(\sigma_V) \leq \pmax{2}
\end{align}

Let us show each of these in turn.

\begin{subproof}{\clref{thm:pp:forget:cond1} -- Support} Let $ \sigma_V \in \support{\delta_2}
$. Thus $ \delta_2(\sigma_V)
= \sum_{\sigma \in \eqclass{\sigma_V}{\delta}{V}} \delta(\sigma) > 0 $ so
there exists $ \sigma \in \eqclass{\sigma_V}{\delta}{V} $ with
$ \delta(\sigma) > 0 $. So $ \sigma \in \support{\delta} $. Therefore,
by \cnumref{thm:pp:forget:given1a}, $ \sigma \in \pconc{\getpoly{}}
$, therefore $ \sigma_V \in \pconc{\getpoly{2}} $ by definition of
polyhedron forget. Therefore
$ \support{\delta_2} \subseteq \pconc{\getpoly{2}} $.
\end{subproof}

\begin{subproof}{\clref{thm:pp:forget:cond2} Support points} First let us
show the following claim.
\begin{align}
\label{thm:pp:forget:cond1b} & \max_{\sigma_V \in \support{\delta_2}} \setsize{\eqclass{\sigma_V}{\delta}{V}} \leq \maxh{y}
\end{align}

By construction of $ \maxh{y} $, we have $ \maxh{y} \geq \max_{\sigma_V \in \pconc{C_2}} \setsize{\eqclass{\sigma_V}{C}{V}}
$. Now, $ \support{\delta_2} \subseteq \pconc{\getpoly{2}} $
by \cnumref{thm:pp:forget:cond1}. Also for any
$ \sigma_V \in \support{\delta_2} $, we have
$ \eqclass{\sigma_V}{\delta}{V} \subseteq \eqclass{\sigma_V}{\getpoly{}}{V}
$ by \rref{thm:pp:project:eqclass:subsets}. Therefore
$ \max_{\sigma_V \in \pconc{C_2}} \setsize{\eqclass{\sigma_V}{C}{V}} \geq \max_{\sigma_V \in \support{\delta_2}}\setsize{\eqclass{\sigma_V}{\getpoly{}}{V}} \geq \max_{\sigma_V \in \support{\delta}} \setsize{\eqclass{\sigma_V}{\delta}{V}}
$. Thus concluding 
$ \maxh{y} \geq \max_{\sigma_V \in \support{\delta_2}} \setsize{\eqclass{\sigma_V}{\delta}{V}}
$.

Consider the elements of $ \support{\delta} $ as they map via state
projection to elements of $ \support{\delta_2} $. Let us view the
elements of the later as bins, with the elements of the former as
objects to pack into the bins. By \cnumref{thm:pp:forget:cond1b}, we
know no bin has more than $ \maxh{y} $ objects, thus we can
apply \rref{thm:pp:project:counting} to conclude there are at least
$ \ceil{\setsize{\support{\delta}} / \maxh{y}} $ non-empty bins, or in
other words,
$ \setsize{\support{\delta_2}} \geq \ceil{\setsize{\support{\delta}}
/\maxh{y}} $. This is itself at least as large as $ \ceil{\smin{}
/ \maxh{y}} = \smin{2} $ by \cnumref{thm:pp:forget:given2a}. Therefore
$ \setsize{\support{\delta_2}} \geq \smin{2} $.

For the other side of the inequality, note that the number of bins
used, or $ \setsize{\support{\delta_2}} $ cannot exceed
$ \setsize{\support{\delta}} \leq \smax{} $ itself. It also cannot
exceed $ \setsize{\pconc{\getpoly{2}}} = \psize{\getpoly{2}} $
given \cnumref{thm:pp:forget:cond1}. Therefore
$ \support{\delta_2} \leq \minparen{\psize{\getpoly{2}}, \smax{}} $,
concluding requirement \cnumref{thm:pp:forget:cond2}.
\end{subproof}

\begin{subproof}{\clref{thm:pp:forget:cond3} Mass} This requirement holds
trivially due to \lref{thm:pp:project:mass} and
assumption \cnumref{thm:pp:forget:given3a}.
\end{subproof}

\begin{subproof}{\clref{thm:pp:forget:cond4} Probability} Let us first show the following claim.
\begin{align}
\label{thm:pp:forget:cond2b}
& \min_{\sigma_V \in \support{\delta_2}} \setsize{\eqclass{\sigma_V}{\delta}{V}} \geq \minh{y}
+ \smin{} - \psize{\getpoly{}}
\end{align}

Let $ \sigma_V \in \support{\delta_2} $. Let us consider the size of
$ \eqclassc{\sigma_V}{\delta}{V} $.

\begin{align*}
\setsize{\eqclassc{\sigma_V}{\delta}{V}}
& \leq \setsize{\eqclassc{\sigma_V}{\getpoly{}}{V}}
& \mathstep{by \rref{thm:pp:project:eqclass:subsets}}\\
& = \psize{\getpoly{}} - \setsize{\eqclass{\sigma_V}{\getpoly{}}{V}} \\
& \leq \psize{\getpoly{}}
- \min_{\tau_V \in \pconc{\getpoly{2}}} \setsize{\eqclass{\tau_V}{\getpoly{}}{V}} \\
& \leq \psize{\getpoly{}} - \minh{y} & \mathstep{by \cnumref{thm:pp:forget:given1b}}
\end{align*}

Let us view now the elements of $ \support{\delta} $ as mapping (via
projection) into two bins, $ \eqclass{\sigma_V}{\delta}{V} $ and
$ \eqclassc{\sigma_V}{\delta}{V} $. By the argument above, we know the
second bin cannot hold more than $ \psize{\getpoly{}} - \minh{y} $ elements,
thus, by \rref{thm:pp:project:counting} (i), it must be the case that the
first bin contains at least $ \setsize{\support{\delta}}
- \paren{\psize{\getpoly{}} - \minh{y}} $ elements. This itself is no
smaller than $ \smin{} - \psize{\getpoly{}} + \minh{y} $
by \cnumref{thm:pp:forget:given2a}. Therefore
$ \setsize{\eqclass{\sigma_V}{\delta}{V}} \geq \smin{} - \psize{\getpoly{}} + \minh{y} $
and thus claim \cnumref{thm:pp:forget:cond2b} holds.

Consider now $ \sigma_V \in \support{\delta_2}
$. By \cnumref{thm:pp:forget:given4a} and the concrete projection
definition, it must be the case that $ \delta_2(\sigma_V)
= \sum_{\sigma \in \eqclass{\sigma_V}{\delta}{V}} \delta(\sigma) \geq \pmin{}
$.
Also,

\begin{align*}
\delta_2(\sigma_V)
& = \sum_{\sigma \in \eqclass{\sigma_V}{\delta}{V}} \delta(\sigma) \\
& \geq \sum_{\sigma \in \eqclass{\sigma_V}{\delta}{V}} \pmin{}
& \mathstep{by \cnumref{thm:pp:forget:given4a}}\\
& = \setsize{\eqclass{\sigma_V}{\delta}{V}} \cdot \pmin{} \\
& \geq \paren{\minh{y} + \smin{} - \psize{\getpoly{}}} \cdot \pmin{}
& \mathstep{by \cnumref{thm:pp:forget:cond2b}}
\end{align*}

Therefore,
$ \delta_2(\sigma_V) \geq \pmin{} \cdot \minparen{1, \minh{y}
+ \smin{} - \psize{\getpoly{}}} $ $ = \pmin{2} $, concluding one
inequality of the last condition.

For the other inequality, let us once more consider a general
$ \sigma_V \in \support{\delta_2} $. 

\begin{align*}
\delta_2(\sigma_V)
& = \sum_{\sigma \in \eqclass{\sigma_V}{\delta}{V}} \delta(\sigma) \\
& \leq \sum_{\sigma \in \eqclass{\sigma_V}{\delta}{V}} \pmax{}
& \mathstep{by \cnumref{thm:pp:forget:given4a}} \\
& = \setsize{\eqclass{\sigma_V}{\delta}{V}} \cdot \pmax{} \\
& \leq \maxh{y} \cdot \pmax{} & \mathstep{by \cnumref{thm:pp:forget:cond1b}}
\end{align*}

Since $ \eqclass{\sigma_V}{\delta}{V} \subseteq \support{\delta} $, we have
$ \setsize{\eqclass{\sigma_V}{\delta}{V}} \leq \setsize{\support{\delta}}
 \leq \smax{} $ (by \cnumref{thm:pp:forget:given2a}). Thus we can also bound
 $ \delta_2 (\sigma_V) $ by $ \smax{} \cdot \pmax{} $. Therefore,
 $ \delta_2(\sigma_V) \leq \pmax{} \cdot \minparen{\maxh{y}, \smax{}} $,
 completing the last claim.
\end{subproof}
\end{proof}

\begin{lemma}[Soundness of Projection] \label{lem:pp:project}
If $\delta \in \ppconc{\pp{}}$ then
$\project{\delta}{V} \in \ppconc{\project{\pp{}}{V}}$.
\end{lemma}

\begin{proof} Let us show this by induction on the size of
$ \vcomp \defeq \fv{\delta} - V $. When $ \setsize{\vcomp} = 0 $ there
is no projection to be done, when $ \setsize{\vcomp} = 1 $, the claim
holds by \lref{lem:pp:forget}. Let us assume the claim holds for
$ \setsize{\vcomp} = n - 1 $ and look at the case where
$ \setsize{\vcomp} = n $.

Let us write $ \vcomp = \set{x_1, \cdots,
x_n} $. Thus $ \project{\delta}{V} = \forget{x_1, \cdots, x_n}{\delta}
$ by Corollary~\ref{thm:pp:project:forgetseq}. By definition of
forget, we also have $ \forget{x_1, \cdots, x_n}{\delta}
= \forget{x_2, \cdots, x_n}{\forget{x_1}{\delta}} $ and
$ \forget{x_1, \cdots, x_n}{\pp{}} = \forget{x_2, \cdots,
x_n}{\forget{x_1}{\pp{}}} $. By \lref{lem:pp:forget}, we know that
$ \forget{x_1}{\delta} \in \ppconc{\forget{x_1}{\pp{}}} $, therefore,
by induction, $ \project{\delta}{V} = \forget{x_2, \cdots,
x_n}{\forget{x_1}{\delta}} \in \forget{x_2, \cdots,
x_n}{\forget{x_1}{\pp{}}} = \project{\pp{}}{V} $.
\end{proof}

\subsection{Assignment}
We begin with some useful notation. 
\begin{notation} Let $ \sigma $ be a state, $ E $
be an expression, $ x $ be a variable, $ S \subseteq \states $, $
V \subseteq \vars $.

\begin{itemize}
\item{}
$ \stassign{\sigma}{x}{E} \defeq \sigma \bparen{x \ra \eeval{E}{\sigma}}
$
\item{} $ \stsassign{S}{x}{E}
\defeq \set{\stassign{\sigma}{x}{E} \given \sigma \in S} $.
\item{} $ \project{S}{V} \defeq \set{\project{\sigma}{V} \given \sigma \in S} $
\end{itemize}
\end{notation}

\begin{definition} 
A state $ \sigma $ is \emph{feasible} for $ \bassign{x}{E} $ iff
$ \sigma \in \stsassign{\states}{x}{E} $. We will say that $ \sigma $
is merely feasible if the assignment is clear from the context.
\end{definition}

\begin{definition} $ \funassign{x}{E} $ is the function from $ \states
$ to feasible states (for $ \bassign{x}{E} $) defined by
$ \funassign{x}{E} : \sigma \mapsto \stsassign{\sigma}{x}{E} $
\end{definition}

\begin{definition} The inverted equivalence class for $ \sigma $ under assignment $ \bassign{x}{E} $
is the set of states that map to $ \sigma $. We define two varieties,
one over all possible states and one for just the states in the
support of a distribution.
\begin{itemize} 
\item{} $ \eqassign{\sigma}{x}{E} \defeq \set{\tau \given \stassign{\tau}{x}{E}
= \sigma} $
\item{} $ \eqassignsup{\sigma}{x}{E}{\delta}
\defeq \set{\tau \in \support{\delta} \given \stassign{\tau}{x}{E}
= \sigma} $
\end{itemize}

\end{definition}

Note that $ \sigma $ is feasible iff $ \eqassign{\sigma}{x}{E} \neq \emptyset $.

\begin{definition} An assignment $ \bassign{x}{E} $ is invertible iff
$ \funassign{x}{E} $ is invertible. We will denote
$ \funinvassign{x}{E} $ as the inverse of $ \funassign{x}{E} $, if it
is invertible. The invertability of $ \funassign{x}{E} $ is
characterized by the existence of the inverse, having the property,
that for every $ \sigma \in \states $, we have
$ \funinvassign{x}{E} \paren{\funassign{x}{E} \paren{\sigma}} = \sigma
$. Equivalently, for every feasible state $ \sigma $,
$ \funassign{x}{E}\paren{\funinvassign{x}{E}\paren{\sigma}} = \sigma
$.

We can also characterize invertability via inverted equivalence
classes. $ \bassign{x}{E} $ is invertible iff for every feasible
$ \sigma $, $ \setsize{\eqassign{\sigma}{x}{E}} = 1 $.

We will say $ E $ is invertible if the variable is clear from the
context.

\end{definition}

Note that since $ \funassign{x}{E} $ only changes the $ x $ component
of a state, the inverse, $ \funinvassign{x}{E} $, also only changes
the $ x $ component, if the inverse exists. This doesn't mean,
however, that the inverse can be represented by an assignment of some
$ E' $ to $ x $. Furthermore, since our language for expressions lacks
division and non-integer constants, no assignment's inverse can be
represented by an assignment.

\begin{definition} The expression $ E $ is \emph{integer linear} iff $
E = n_1 \times x_1 + \cdots + n_m \times x_m $, where $ n_i $ are
integer constants, and $ x_i $ are variables. We assume that all the
variables in a given context are present. We will generally use $ x_i
$ and $ n_i $ to refer to the contents of a integer linear expression.
\end{definition}

From now on, we will assume all expressions $ E $ are integer
linear. Programs containing non-linear expressions are just not
handled by our system at this stage and linear expressions not fully
specified are equivalent to integer linear expressions with $ n_i = 0
$ for variables unused in the original expression.

\begin{lemma} \label{thm:pp:assign:zero} $ \bassign{x_1}{E} $ is non-invertible iff $ n_1 = 0
$. In other words, $ \bassign{x_1}{E} $ is non-invertible iff $ E $ doesn't depend on $
x_1 $.
\end{lemma}
\begin{proof}
\textbf{($ \Rightarrow $)}
Assume otherwise. Thus $ E $ is non-invertible but $ n_1 \neq 0 $. So
we have a feasible state $ \sigma $ with
$ \setsize{\eqassign{\sigma}{x_1}{E}} \neq 1 $. Since feasible states
have non empty inverted equivalence sets, it must be that
$ \setsize{\eqassign{\sigma}{x_1}{E}} \geq 2 $. So let
$ \tau, \tau' \in \eqassign{\sigma}{x_1}{E} $ with $ \tau \neq \tau'
$. So $ \stassign{\tau}{x_1}{E} = \stassign{\tau'}{x_1}{E} = \sigma
$. Since assignment to $ x_1 $ doesn't change the state other than in
its value of $ x_1 $, $ \tau $ and $ \tau' $ can only differ in their
value for $ x_1 $.

But since $ \tau $ and $ \tau' $ are identical after the assignment,
we have,

\begin{align*}
\stassign{\tau}{x_1}{E}(x_1) & = n_1 \tau (x_1) + n_2 \tau
(x_2) + \cdots + n_m \tau(x_m) \\ & = n_1 \tau (x_1) + n_2 \tau' (x_2)
+ \cdots + n_m \tau' (x_m) \\
& = n_1 \tau' (x_1) + n_2 \tau' (x_2) + \cdots + n_m \tau' (x_m) \\
& = \stassign{\tau'}{x_1}{E}(x_1) 
\end{align*}

Canceling out the common $ \tau'(x_i) $ terms, we have $
n_1 \tau(x_1) = n_1 \tau'(x_1) $ and since $ n_1 \neq 0 $, we conclude
$ \tau(x_1) = \tau'(x_1) $, contradicting $ \tau \neq \tau' $.

\textbf{($ \Leftarrow $)} Let $ \sigma $ be a feasible state and let
$ \tau \in \eqassign{\sigma}{x_1}{E} $. Let $ \tau'
= \stassign{\tau}{x_1}{\tau(x_1) + 1} $. Since $ E $ doesn't depend
on $ x_1 $, we have $ \eeval{E}{\tau} = \eeval{E}{\tau'} $ and
therefore $ \stassign{\tau'}{x_1}{E} = \stassign{\tau}{x_1}{E}
= \sigma $ and so we have $ \tau, \tau' \in \eqassign{\sigma}{x_1}{E}
$ with $ \tau \neq \tau' $, therefore $ E $ is non-invertible.
\end{proof}

\begin{lemma} \label{thm:pp:assign:nifeasible} Assume $ \bassign{x}{E} $ is non-invertible. $ \sigma $
is feasible iff $ \stassign{\sigma}{x}{E} = \sigma $.
\end{lemma}

\begin{proof} \textbf{($\Rightarrow$)} Let $ \sigma $ be
feasible. Thus $ \sigma = \stassign{\tau}{x}{E} $ for some
$ \tau \in \states $. Since $ E $ doesn't depend on $ x $
by \lref{thm:pp:assign:zero}, we have
$ \stassign{\paren{\stassign{\tau}{x}{E}}}{x}{E} = \stassign{\tau}{x}{E}
= \sigma $. So $ \stassign{\sigma}{x}{E} = \sigma $.

\textbf{($\Leftarrow$)} Assume $ \stassign{\sigma}{x}{E} = \sigma
$. Thus $ \sigma \in \stsassign{\states}{x}{E} $ by definition.
\end{proof}

\begin{lemma} \label{thm:pp:assign:eqproject} Assume $ \bassign{x}{E} $ is non-invertible. Let
$ \delta $ be a distribution with $ x \in \fv{\delta} $ and let $ V =
\fv{\delta} - \set{x} $. If $ \sigma
$ is feasible, then $ \eqassignsup{\sigma}{x}{E}{\delta}
= \eqclass{\project{\sigma}{V}}{\delta}{V} $.

%\begin{subtheorem} For every
%$ \sigma_{V} \in \support{\forget{x}{\delta}} $, there exists a
%unique feasible $ \tau \in \support{\delta} $ with
%$ \project{\tau}{V} = \sigma_{V} $.
%\end{subtheorem}
\end{lemma}

\begin{proof} Let $ \tau \in \eqassignsup{\sigma}{x}{E}{\delta} $. So
$ \stassign{\tau}{x}{E} = \sigma $ and $ \tau \in \support{\delta}
$. But the assignment only changes $ x $, thus $ \project{\tau}{V}
= \project{\sigma}{V} $, therefore
$ \tau \in \eqclass{\project{\sigma}{V}}{\delta}{V} $. Thus
$ \eqassignsup{\sigma}{x}{E}{\delta} \subseteq \eqclass{\project{\sigma}{V}}{\delta}{V} $.

Let $ \tau \in \eqclass{\project{\sigma}{V}}{\delta}{V} $. So
$ \tau \in \support{\delta} $ and $ \project{\tau}{V}
= \project{\sigma}{V} $. Since $ E $ doesn't depend on $ x $, we have
$ \stassign{\tau}{x}{E} = \stassign{\sigma}{x}{E} = \sigma $; the second
equality follows from \lref{thm:pp:assign:nifeasible} as $ \sigma $ is
feasible by assumption. So
$ \tau \in \eqassignsup{\sigma}{x}{E}{\delta} $. Therefore
$ \eqclass{\project{\sigma}{V}}{\delta}{V} \subseteq \eqassignsup{\sigma}{x}{E}{\delta}
$.

%\textbf{(ii)} Since $ E $ does not depend on $ x $, we can evaluate it
%on $ \sigma_{V} $. Let $ \tau = \sigma_{V} \cup \set{x
%= \eeval{E}{\sigma_{V}}} $, a state extended to include $ x $ whose
%value is set to $ \eeval{E}{\sigma_{V}} $.  !!! todo or remove !!!
\end{proof}

%\begin{lemma} If $ \bassign{x}{E} $ is invertible, then
%$ \funinvassign{x}{E} $ can be defined via $ \funassign{x}{E}
%: \sigma \mapsto \stassign{\sigma}{x}{f(\sigma)} $ for some function $
%f $. That is, the inverse of an assignment to $ x $ only modifies the
%value of $ x $.
%\end{lemma}
%
%\begin{proof} Without the loss of generality, let us set $ x = x_1
%$. We have $ E = n_1 x_1 + \cdots + n_m x_m $. Let us define $
%f $ via $ f : \sigma \mapsto \frac{1}{n_1}\bparen{\sigma(x_1)
%- \sum_{i=2}^{m} n_i \sigma(x_i)} $. Note that $
%n_1 \neq 0 $ by \lref{thm:pp:assign:zero}.
%
%Let $ \sigma $ be a state. Let us show that
%$ \funinvassign{x_1}{E}\paren{\funassign{x_1}{E}\paren{\sigma}}
%= \sigma $. Let $ \sigma' = \funassign{x_1}{E}\paren{\sigma} $. Since
%both $ \funassign{x_1}{E} $ and $ \funinvassign{x_1}{E} $ only change the
%values of $ x_1 $, we know that the non-$x_1$ components of
%$ \sigma $ , $ \sigma' $, and $ \funinvassign{x_1}{E} $ are
%identical. So let us look at the $ x_1 $ component.
%$$ \sigma'(x_1) = \sum_{i=1}^{m} n_1 \sigma(x_i) $$
%So,
%
%\begin{align*}
%& \paren{\funinvassign{x_1}{E}\sigma'}(x_1) \\
%& \; = \frac{1}{n_1} \bparen{\sigma'(x_1)
%- \sum_{i=2}^{m}n_i \sigma'(x_i) } \\
%& \; = \frac{1}{n_1} \bparen{n_1 \sigma(x_1)
%+ \sum_{i=2}^{m}n_i \sigma(x_i) - \sum_{i=2}^{m}n_i \sigma'(x_i)} \\
%& \; = \frac{1}{n_1} \bparen{n_1 \sigma(x_1)
%+ \sum_{i=2}^{m}n_i \sigma(x_i) - \sum_{i=2}^{m}n_i \sigma(x_i)} \\
%& \; = \frac{1}{n_1} \bparen{n_1 \sigma(x_1)} \\
%& \; = \sigma(x_1)
%\end{align*}
%
%Thus the $x_1$ component of $ \funinvassign{x_1}{E}(\sigma') $ is also
%identical to that of $ \sigma $, hence
%$ \funinvassign{x_1}{E}(\sigma') = \sigma $.
%\end{proof}

\begin{remark} \label{thm:pp:assign:one-inverse} Assume $ \bassign{x}{E} $ is invertible. For every
feasible $ \sigma $, we have $ \eqassign{\sigma}{x}{E}
= \set{\funinvassign{x}{E}(\sigma)} $.
\end{remark}

\begin{proof} Invertability tells us that $ \eqassign{\sigma}{x}{E} $
has only one element. The function $ \funinvassign{x}{E} $, given
the feasible $ \sigma $, produces an element of
$ \eqassign{\sigma}{x}{E} $, as
$ \stassign{\paren{\funinvassign{x}{E}(\sigma)}}{x}{E} = \sigma $.
\end{proof}

\begin{definition} We define an alternate means of assignment,
$ \deassign{\delta}{x}{E} $. Let $ V = \fv{\delta} - \set{x} $.

\begin{itemize}
\item{} If $ \bassign{x}{E} $ is invertible, then
\begin{align*}
\deassign{\delta}{x}{E} = \lambda \sigma \lsep \aif & \sigma \text{ is
feasible} \\
& \athen \delta \paren{\funinvassign{x}{E}(\sigma)} \\
& \aelse 0
\end{align*}
\item{} If $ \bassign{x}{E} $ is not invertible, then
\begin{align*}
\deassign{\delta}{x}{E} = \lambda \sigma \lsep \aif & \sigma \text{ is
feasible} \\
& \athen \project{\delta}{V} \paren{\project{\sigma}{V}} \\
& \aelse 0
\end{align*}
\end{itemize}

\end{definition}

\begin{lemma} \label{thm:pp:assign:eqdef} For any $ \delta $, $ \dassign{\delta}{x}{E}
= \deassign{\delta}{x}{E} $.
\end{lemma}

\begin{proof} Let $ \delta' = \dassign{\delta}{x}{E} $ and $ \delta''
= \deassign{\delta}{x}{E} $.
\begin{align*}
\delta'(\sigma)
& = \sum_{\tau \given \stassign{\tau}{x}{E} = \sigma} \delta (\tau) \\
& = \sum_{\tau \in \eqassignsup{\sigma}{x}{E}{\delta}} \delta (\tau)
\end{align*}

\textbf{Case 1: $ \bassign{x}{E} $ is invertible}

If $ \sigma $ is feasible, $ \eqassign{\sigma}{x}{E} $ has only one
element, $ \sigma^{-1} = \funinvassign{x}{E}(\sigma) $,
by \rref{thm:pp:assign:one-inverse}. So $ \delta'(\sigma)
= \delta(\sigma^{-1}) = \delta''(\sigma) $. Note that when
$ \sigma^{-1} $ is not in $ \support{\delta} $ then $ \delta'(\sigma)
= 0 = \delta''(\sigma) $.

If $ \sigma $ is not feasible then $ \eqassign{\sigma}{x}{E}
= \emptyset $ so $ \delta'(\sigma) = 0 = \delta''(\sigma) $.

\textbf{Case 2: $ \bassign{x}{E} $ is non-invertible}

If $ \sigma $ is feasible, then by \lref{thm:pp:assign:eqproject} we
have $ \eqassignsup{\sigma}{x}{E}{\delta}
= \eqclass{\project{\sigma}{V}}{\delta}{V} $.

\begin{align*}
\delta'(\sigma)
& = \sum_{\tau \in \eqassignsup{\sigma}{x}{E}{\delta}} \delta (\tau) \\
& = \sum_{\tau \in \eqclass{\project{\sigma}{V}}{\delta}{V}} \delta
(\tau)  \\
& = \paren{\project{\delta}{V}}(\project{\sigma}{V}) \\
& = \delta''(\sigma)
\end{align*}

If $ \sigma $ is not feasible then
$ \eqassignsup{\sigma}{x}{E}{\delta} = \emptyset $ so
$ \delta'(\sigma) = 0 = \delta''(\sigma) $.

\end{proof}

\begin{lemma} \label{thm:pp:assign:support} Assume $ \bassign{x}{E} $ is invertible, then $ \support{\delta}
= \set{\funinvassign{x}{E}(\sigma) \given \sigma \in \support{\deassign{\delta}{x}{E}}} $.
\end{lemma}

\begin{proof} Let $ \delta_2 = \deassign{\delta}{x}{E} $. Let $ \tau \in \support{\delta} $. So
$ \sigma \defeq \stassign{\tau}{x}{E} \in \support{\delta_2} $ and
$ \funinvassign{x}{E}(\sigma) = \tau $. So
$ \tau \in \set{\funinvassign{x}{E}(\sigma) \given \sigma \in
\support{\delta_2}} $.

Let
$ \tau \in \set{\funinvassign{x}{E}(\sigma) \given \sigma \in \support{\delta_2}}
$. So $ \tau = \funinvassign{x}{E}(\sigma) $ for some
$ \sigma \in \support{\delta_2} $. So there exists $ \tau' \in
\support{\delta} $ such that $ \stassign{\tau'}{x}{E} = \sigma $. But
$ \funinvassign{x}{E}(\sigma)
= \funinvassign{x}{E}(\funassign{x}{E}(\tau')) = \tau' $ so $ \tau'
= \tau $ as $ \tau = \funinvassign{x}{E}(\sigma) $. So $ \tau \in
\support{\delta} $.
\end{proof}

\begin{replemma}{lem:pp:assign}[Soundness of Assignment]
If $\delta \in \ppconc{\pp{}}$ then $\dassign{\delta}{x}{E} \in \ppconc{\dassign{\pp{}}{x}{E}}$.
\end{replemma}

\begin{proof} Let $ V = \fv{\delta} - \set{x} $. By assumption, we
have the following.
\begin{align}
\label{thm:pp:assign:given1a} & \support{\delta} \subseteq \pconc{\getpoly{}} \\
\label{thm:pp:assign:given2a}& \smin{} \leq \setsize{\support{\delta}} \leq \smax{} \\
\label{thm:pp:assign:given3a}& \mmin{} \leq \pmass{\delta} \leq \mmax{} \\
\label{thm:pp:assign:given4a}& \forall \sigma \in \support{\delta} \;
. \; \pmin{} \leq \delta(\sigma) \leq \pmax{}
\end{align}

Let $ \pp{2} = \dassign{\pp{}}{x}{E} $ and $ \delta_2
= \dassign{\delta}{x}{E} = \deassign{\delta}{x}{E}
$. \lref{thm:pp:assign:eqdef} lets us use
$ \dassign{\delta}{x}{E} $ or $ \deassign{\delta}{x}{E} $ interchangeably.

We consider two cases.
\textbf{Case 1: $ \bassign{x}{E} $ is invertible}

In this case, $ \pp{2} $ is defined with $ \getpoly{2} = \dassign{C}{x}{E} $ and all
other parameters as in $ \pp{} $. Thus we need to show the following.
\begin{align}
\label{thm:pp:assign:cond1a}& \support{\delta_2} \subseteq \pconc{\getpoly{2}} \\
\label{thm:pp:assign:cond2a}& \smin{}
= \smin{2} \leq \setsize{\support{\delta_2}} \leq \smax{2} = \smax{} \\
\label{thm:pp:assign:cond3a}& \mmin{}
= \mmin{2} \leq \pmass{\delta_2} \leq \mmax{2} = \mmax{} \\
\label{thm:pp:assign:cond4a}& \forall \sigma \in \support{\delta_2} \;
. \; \pmin{} = \pmin{2} \leq \delta_2(\sigma) \leq \pmax{2} = \pmax{}
\end{align}

\begin{subproof}{\clref{thm:pp:assign:cond1a} Support} By definition,
$ \pconc{\getpoly{2}}
= \set{\stassign{\sigma}{x}{E} \given \sigma \in \pconc{\getpoly{}}}
$. Let $ \tau \in \support{\delta_2} $, so we have $ \sigma \in
\support{\delta} \subseteq \pconc{\getpoly{}} $ with $ \stassign{\sigma}{x}{E} = \tau $. So
$ \tau \in \pconc{\getpoly{2}} $. So $ \tau \in \pconc{\getpoly{2}} $
and thus $ \support{\delta_2} \subseteq \pconc{\getpoly{2}} $.
\end{subproof}

\begin{subproof}{\clref{thm:pp:assign:cond2a} Support points}
By \lref{thm:pp:assign:support} we have $ \support{\delta}
= \set{\funinvassign{x}{E}(\sigma) \given \sigma \in \support{\delta_2}}
$. Inverse functions are necessarily injective over their domain, and
since $ \support{\delta_2} $ are all feasible (thus in the domain of
the inverse), we have
$ \setsize{\set{\funinvassign{x}{E}(\sigma) \given \sigma \in \support{\delta_2}}}
= \setsize{\support{\delta_2}} $. So $ \setsize{\support{\delta}}
= \setsize{\support{\delta_2}} $. This, together
with \cnumref{thm:pp:assign:given2a}, completes the claim.
\end{subproof}

\begin{subproof}{\clref{thm:pp:assign:cond3a} Mass} Note again that
$ \support{\delta_2} \subseteq \stsassign{\states}{x}{E} $. That is,
all possible states are feasible. So we can write:
\begin{align*}
\pmass{\delta_2} & = \sum_{\sigma \in \support{\delta_2}} \delta_2
(\sigma) \\
 & = \sum_{\sigma \in \support{\delta_2}} \delta
 ( \funinvassign{x}{E}(\sigma) ) & \mathstep{by defn. of $ \delta_2
 $} \\
 & = \sum_{\tau \in \support{\delta}} \delta (\tau) & \mathstep{by \lref{thm:pp:assign:support}} \\
 & = \pmass{\delta}
\end{align*}

The above, together with \cnumref{thm:pp:assign:given3a}, completes
this claim.
\end{subproof}

\begin{subproof}{\clref{thm:pp:assign:cond4a} Probability} Since
$ \support{\delta_2} $ are feasible, we have, for every
$ \sigma \in \support{\delta_2} $, $ \delta_2(\sigma)
= \delta(\funinvassign{x}{E}(\sigma)) $. But also,
$ \funinvassign{x}{E}(\sigma) \in \support{\delta} $. Taking this,
and \cnumref{thm:pp:assign:given4a}, completes this claim, and
soundness in the invertible case.
\end{subproof}

\textbf{Case 2: $ \bassign{x}{E} $ is non-invertible} In this case,
$ \pp{2} $ is defined via the forget operation. If $ \pp{1}
= \forget{x}{\pp{}} $ and $ \poly_1 = (\cons_1, V_1) $, then $ \pp{2}
= \dassign{\pp{}}{x}{E} $ has $ \getpoly{2} = (\cons_1 \cup \set{x =
E}, V_1 \cup \set{x}) $, and all other parameters as in $ \pp{1} $.

We need to show the following four claims.

\begin{align}
\label{thm:pp:assign:cond1b}& \support{\delta_2} \subseteq \pconc{\getpoly{2}} \\
\label{thm:pp:assign:cond2b}& \smin{1}
= \smin{2} \leq \setsize{\support{\delta_2}} \leq \smax{2} = \smax{1} \\
\label{thm:pp:assign:cond3b}& \mmin{1}
= \mmin{2} \leq \pmass{\delta_2} \leq \mmax{2} = \mmax{1} \\
\label{thm:pp:assign:cond4b}& \forall \sigma \in \support{\delta_2} \;
. \; \pmin{1} = \pmin{2} \leq \delta_2(\sigma) \leq \pmax{2} = \pmax{1}
\end{align}

Recall the definition of $ \delta_2 $:
\begin{align*}
\deassign{\delta}{x}{E} = \lambda \sigma \lsep \aif & \sigma \text{ is
feasible} \\
& \athen \project{\delta}{V} \paren{\project{\sigma}{V}} \\
& \aelse 0
\end{align*}

\begin{subproof}{\clref{thm:pp:assign:cond1b} Support} Let
$ \sigma \in \support{\delta_2} $. So
$ \project{\sigma}{V} \in \support{\project{\delta}{V}} $. so there
exists $ \tau \in \support{\delta} \subseteq \pconc{\getpoly{}} $ with $ \project{\tau}{V}
= \project{\sigma}{V} $. So
$ \project{\tau}{V} \in \pconc{\forget{x}{\getpoly{}}}
= \pconc{\getpoly{1}} $. So $ \tau \in \pconc{(\cons_1, V_1 \cup \set{x})}
$ as the add dimension operation leaves $ x $ unconstrained. The
non-constraint of $ x $ also tells us that
$ \sigma \in \pconc{(\cons_1, V_1 \cup \set{x})} $ as we have
$ \project{\sigma}{V} = \project{\tau}{V} $.

Since $ \sigma \in \support{\delta_2}$, $ \sigma $ is feasible so
$ \sigma $ satisfies the $ x = E $ constraint as $ \sigma
= \stassign{\tau}{x}{E} $ for some $ \tau $. Thus, overall, we have
$ \sigma \in \pconc{(\cons_1 \cup \set{x = E}, V_1 \cup \set{x})}
= \pconc{\getpoly{2}} $.
\end{subproof}

\begin{subproof}{\clref{thm:pp:assign:cond2b} Support points} Let $ \delta_1 =
\project{\delta}{V} = \forget{x}{\delta} $. By soundness of
forget (\lref{lem:pp:forget}), we have the following.

$$ \smin{1} = \smin{2} \leq \setsize{\support{\delta_1}} \leq \smax{2}
= \smax{1} $$

All we need to show, then, is the following.

\begin{align}
\label{thm:pp:assign:claim1} \setsize{\support{\delta_1}} = \setsize{\support{\delta_2}}
\end{align}

Let us show this by establishing a bijection $ f $ between the two sets. Let us define $ f
: \support{\delta_1} \ra \support{\delta_2} $ via $ f
: \sigma_V \mapsto \sigma_V \cup \set{x = \eeval{E}{\sigma_V}} $.

To show $ f $ is injective, let $ \sigma_V, \sigma_V'$ be such that $
f(\sigma_V) = f(\sigma_V') $. Since $ f $ does not change any part of
the state other than adding $ x $, it must be that $ \sigma_V
= \sigma_V' $.

To show that $ f $ is surjective, consider
$ \sigma \in \support{\delta_2} $. So $ \sigma $ is feasible, so
$ \stassign{\sigma}{x}{E} = \sigma $
by \lref{thm:pp:assign:nifeasible}. Also
$ \project{\sigma}{V} \in \support{\sigma_V} $, considering the
definition of $ \sigma_2 $. Since $ E $ doesn't depend on $ x $, we
can write $ \eeval{E}{\sigma} = \eeval{E}{\project{\sigma}{V}} $,
therefore $ f(\project{\sigma}{V}) = \project{\sigma}{V} \cup \set{x
= \eeval{E}{\project{\sigma}{V}}} = \stassign{\sigma}{x}{E} = \sigma
$.

Since $ f $ is injective and surjective, it is a bijection and thus
$ \setsize{\support{\delta_1}} = \setsize{\support{\delta_2}} $.
\end{subproof}

\begin{subproof}{\clref{thm:pp:assign:cond3b} Mass} Let $ \delta_1 =
\project{\delta}{V} = \forget{x}{\delta} $. Let us show the following claim.

\begin{align}
\label{thm:pp:assign:claim2} LHS = \support{\delta_1}
= \set{\project{\sigma}{V} \given \sigma \in \support{\delta_2}} =
RHS
\end{align}

Let $ \sigma_V \in \support{\delta_1} $. So there exists
$ \sigma \in \support{\delta} $ with $ \project{\sigma}{V} = \sigma_V
$. So $ \stassign{\sigma}{x}{E} \in \support{\delta_2} $. But the
assignment doesn't change anything but $ x $, so it must be that
$ \project{\paren{\stassign{\sigma}{x}{E}}}{V}
= \project{\sigma}{V} $, therefore
$ \sigma_V = \project{\sigma}{V} \in \set{\project{\tau}{V} \given \tau \in \support{\delta_2}}
$. Thus $ LHS \subseteq RHS $.

On the other side, let $ \sigma \in \support{\delta_2} $, so $ \sigma
= \stassign{\tau}{x}{E} $ for some $ \tau \in \support{\delta} $, by
the original definition of distribution assignment. So
$ \project{\tau}{V} \in \support{\delta_1} $. But
$ \project{\paren{\stassign{\tau}{x}{E}}}{V} = \project{\tau}{V} $ as
the assignment doesn't change anything but $ x $. So
$ \project{\sigma}{V}
= \project{\paren{\stassign{\tau}{x}{E}}}{V}
= \stassign{\tau}{x}{E} \in \support{\delta_1} $, concluding that $
RHS \subseteq LHS $, and thus $ LHS = RHS $.

Note that this, together with \cnumref{thm:pp:assign:claim1}, show
that not only are the sets equal, but also no two elements of
$ \support{\delta_2} $ can map, via projection to $ V $, to the same
element of $ \support{\delta_1} $.

By soundness of forget (\lref{lem:pp:forget}), we have the following.
$$ \mmin{1} = \mmin{2} \leq \pmass{\delta_1} \leq \mmax{2}
= \mmax{1} $$

Again, we proceed to show that $ \pmass{\delta_1} = \pmass{\delta_2}
$.

\begin{align*}
\pmass{\delta_1}
&= \sum_{\sigma_V \in \support{\delta_1}} \delta_1 (\sigma_V) \\
&= \sum_{\sigma \in \support{\delta_2}}  \delta_1
(\project{\sigma}{V}) & \mathstep{by \cnumref{thm:pp:assign:claim1} and
\cnumref{thm:pp:assign:claim2}} \\
&= \sum_{\sigma \in \support{\delta_2}} \delta_2 (\sigma)
& \mathstep{by defn. of $ \delta_2 $} \\
&= \pmass{\delta_2}
\end{align*}

\end{subproof}

\begin{subproof}{\clref{thm:pp:assign:cond4b} Probability} Let
$ \sigma \in \support{\delta_2} $. So $ \sigma $ is feasible, so
$ \delta_2(\sigma) = \paren{\project{\delta}{V}}(\project{\sigma}{V})
> 0 $. Therefore
$ \project{\sigma}{V} \in \support{\project{\delta}{V}} $. Thus, by
soundness of forget (\lref{lem:pp:forget}), we have
$ \pmin{2}
= \pmin{1} \leq \paren{\project{\delta}{V}}(\project{\sigma}{V}) \leq \pmax{1}
= \pmax{2}
$, concluding the claim and the lemma.
\end{subproof}
\end{proof}

%\begin{theorem}
%\label{thm:assignment-by-cases}
%Let $\delta \bparen{x \ra \aexp}$ be defined as follows.
%\begin{equation}
%\label{thm:assignment-by-cases:original-def}
%\delta \bparen{x \ra \aexp} \defeq \lambda \sigma \lsep \sum_{\tau \; | \; \tau
%  \bparen{x \ra \eeval{\aexp}{\tau}} = \sigma} \delta (\tau)
%\end{equation}
%Let $\delta \aparen{x \ra \aexp}$ be defined as follows, when $x \ra \aexp$ is invertible.
%\begin{equation}
%\label{thm:assignment-by-cases:invertible-def}
%\delta \aparen{x \ra \aexp} \defeq \lambda \sigma \lsep \delta (\sigma\bparen{x \ra \eeval{\aexp'}{\sigma}})
%\end{equation}
%Let $V' = \dom{\delta} - \{x\}$ and $\delta' = \project{\delta}{V'}$.
%Let $\delta \aparen{x \ra \aexp}$ be defined as follows when $x \ra \aexp$ is non-invertible.
%\begin{equation}
%\label{thm:assignment-by-cases:noninvertible-def}
%\delta \aparen{x \ra \aexp} \defeq \lambda \sigma \lsep \aif \sigma(x) = \eeval{E}{\sigma}\athen
%\delta'(\project{\sigma}{V'}) \aelse 0
%\end{equation}
%Then $\delta \bparen{x \ra \aexp} = \delta \aparen{x \ra \aexp}$.
%\end{theorem}
%
%\begin{proof}
%If $x \ra \aexp$ is invertible, then for each $\sigma$, there is exactly one $\tau$ such that $\tau \bparen{x \ra \eeval{\aexp}{\tau}} = \sigma$.  Furthermore, $\tau = \sigma \bparen{x \ra \eeval{\aexp'}{\sigma}}$ for some $\aexp'$.  Thus, the summation in (\ref{thm:assignment-by-cases:original-def}) reduces to a single case where $\tau = \sigma \bparen{x \ra \eeval{\aexp'}{\sigma}}$, which is equivalent to (\ref{thm:assignment-by-cases:invertible-def}).
%
%If $x \ra \aexp$ is non-invertible, then the value of $\aexp$ does not depend on $x$ (all linear updates with a non-zero coefficient for $x$ are invertible).  Thus we have $\sigma(x) = \eeval{\aexp}{\sigma}$ if and only if $\sigma(x) = \eeval{\aexp}{(\project{\sigma}{V'})}$.  Similarly, we have
%\[\tau \bparen{x \ra \eeval{\aexp}{\tau}} = \sigma\]
%if and only if
%\[(\project{\tau}{V'}) \bparen{x \ra \eeval{\aexp}{(\project{\tau}{V'})}} = \sigma\]
%for the $\tau$ from the summation in (\ref{thm:assignment-by-cases:original-def}).  This implies that $\tau$ and $\sigma$ agree on all variables except $x$, which gives us $\project{\tau}{V'} = \project{\sigma}{V'}$.  Thus, in order for the sum to be populated, we must have the following.
%\begin{equation}
%\label{thm:assignment-by-cases:sigma-in-sum}
%(\project{\sigma}{V'}) \bparen{x \ra \eeval{\aexp}{(\project{\sigma}{V'})}} = \sigma
%\end{equation}
%
%We now turn to the evaluation of (\ref{thm:assignment-by-cases:noninvertible-def}).  Equation (\ref{thm:assignment-by-cases:sigma-in-sum}) indicates that if $\sigma(x) \neq \eeval{E}{\sigma}$ then the sum evaluates to $0$, thus validating the else branch of (\ref{thm:assignment-by-cases:noninvertible-def}).
%
%We next show that if $\sigma(x) = \eeval{E}{\sigma}$ then the sum in (\ref{thm:assignment-by-cases:original-def}) is exactly $\delta'(\project{\sigma}{V'})$.  Expanding the definition of $\project{\delta}{V'}$, we have
%\[\delta' = \lambda \sigma_{V'} \in \states_{V'} \lsep \sum_{\sigma' \mid (\project{\sigma'}{V'} = \sigma_{V'})} \delta(\sigma')\]
%Thus $\delta'(\project{\sigma}{V'})$ is the following.
%\[\sum_{\sigma' \mid (\project{\sigma'}{V'} =\project{\sigma}{V'})} \delta(\sigma')\]
%To show that this sum is equivalent to that given in (\ref{thm:assignment-by-cases:original-def}), it suffices to show that
%\[\{\tau \mid (\project{\tau}{V'}) \bparen{x \ra \eeval{\aexp}{(\project{\tau}{V'})}} = \sigma\}\]
%is equivalent to
%\[\{\sigma' \mid (\project{\sigma'}{V'} =\project{\sigma}{V'})\}\]
%To show this, we must show
%\[\bigl((\project{\tau}{V'}) \bparen{x \ra \eeval{\aexp}{(\project{\tau}{V'})}} = \sigma\bigr)
%\Leftrightarrow
%\bigl(\project{\tau}{V'} =\project{\sigma}{V'}\bigr)\]
%
%The forward direction is straightforward as updating $\tau$ with $x \ra \eeval{\aexp}{\project{\tau}{V'}}$ changes only $x$.  Thus the fact that the update of $\tau$ is equal to $\sigma$ implies $\project{\tau}{V'} =\project{\sigma}{V'}$.
%
%For the backward implication, we have that $\project{\tau}{V'} =\project{\sigma}{V'}$.  Since $\aexp$ does not involve $x$, this implies that
%\[(\project{\tau}{V'}) \bparen{x \ra \eeval{\aexp}{(\project{\tau}{V'})}}\]
%is equivalent to
%\[(\project{\sigma}{V'}) \bparen{x \ra \eeval{\aexp}{(\project{\sigma}{V'})}}\]
%We have that $\sigma(x) = \eeval{E}{\sigma}$ and thus
%\[(\project{\sigma}{V'}) \bparen{x \ra \eeval{\aexp}{(\project{\sigma}{V'})}} = \sigma\]
%Combining these yields the following, which is our final goal.
%\[(\project{\tau}{V'}) \bparen{x \ra \eeval{\aexp}{(\project{\tau}{V'})}} = \sigma\]
%\end{proof}

\subsection{Plus}

\deleted{
\begin{reptheorem}{thm:pord}
If $\pp{1} \ppord \pp{2}$ then $\ppconc{\pp{1}} \subseteq \ppconc{\pp{2}}$.  If $\getpoly{1} \neq \bot$ then we also have $\ppconc{\pp{1}} \subseteq \ppconc{\pp{2}}$ implies $\pp{1} \ppord \pp{2}$.
\end{reptheorem}

\begin{proof}
We first note that $\ppconc{\pp{1}} \subseteq \ppconc{\pp{2}}$ is equivalent to
\[\forall \delta.\ \delta \in \ppconc{\pp{1}} \imp \delta \in \ppconc{\pp{2}}\]
which is, by expanding the definition of $\ppconcfun$, equivalent to showing the following for all $\delta$.
\begin{equation}
\label{thm:pord:rhs}
\begin{aligned}
 & \nzset{\delta} \subseteq \pconc{\getpoly{1}} \wedge {} \\
  & \smin{1} \leq \setsize{\nzset{\delta}} \leq \smax{1} \wedge {} \\
  & \mmin{1} \leq \pmass{\delta} \leq \mmax{1} \wedge \\
  & \forall \sigma \in \nzset{\delta}.\ \pmin{1} \leq \delta(\sigma) \leq \pmax{1}\\
\imp \\
  & \nzset{\delta} \subseteq \pconc{\getpoly{2}} \wedge {} \\
  & \smin{2} \leq \setsize{\nzset{\delta}} \leq \smax{2} \wedge {} \\
  & \mmin{2} \leq \pmass{\delta} \leq \mmax{2} \wedge \\
  & \forall \sigma \in \nzset{\delta}.\ \pmin{2} \leq \delta(\sigma) \leq \pmax{2}\\ 
\end{aligned}
\end{equation}

We next expand the definition of $\ppord$ (and then $\pord$) to conclude that $\pp{1} \pord \pp{2}$ is equivalent to
\begin{equation}
\label{thm:pord:lhs}
\begin{aligned}
&\pconc{\getpoly{1}} \subseteq \pconc{\getpoly{2}} \\
&{}\wedge \{\mathrm{s,p,m}\}_2^\textrm{min} \leq \{\mathrm{s,p,m}\}_1^\textrm{min} \\
&{}\wedge \{\mathrm{s,p,m}\}_1^\textrm{max} \leq \{\mathrm{s,p,m}\}_2^\textrm{max}
\end{aligned}
\end{equation}

That (\ref{thm:pord:lhs}) implies (\ref{thm:pord:rhs}) follows by transitivity of $\leq$ and $\subseteq$.  For example, we have $\pconc{\getpoly{1}} \subseteq \pconc{\getpoly{2}}$ from (\ref{thm:pord:lhs}).  We can assume that $\nzset{\delta} \subseteq \pconc{\getpoly{1}}$ and use transitivity of $\subseteq$ to obtain $\nzset{\delta} \subseteq \pconc{\getpoly{2}}$.  This process can be carried out for each conjunction in (\ref{thm:pord:rhs}).

To show that (\ref{thm:pord:rhs}) implies (\ref{thm:pord:lhs}), we use our assumption (\ref{thm:pord:rhs}) with a $\delta$ such that $\nzset{\delta} = \pconc{\getpoly{1}}$ (such a $\delta$ can always be found).  Then (\ref{thm:pord:rhs}) gives us that $\nzset{\delta} \subseteq \pconc{\getpoly{2}}$ and thus $\pconc{\getpoly{1}} \subseteq \pconc{\getpoly{2}}$.  A similar process can be carried out for the other conjuncts.

The only case that is slightly different involves the last conjuncts in the antecedent and conclusion of (\ref{thm:pord:rhs}), and this is where the condition that $\getpoly{1} \neq \bot$ is needed.  We choose $\delta$ such that $\delta(\sigma_0) = \pmin{1}$ for some $\sigma_0 \in \nzset{\delta}$ and that satisfies $\forall \sigma \in \nzset{\delta}.\ \pmin{1} \leq \delta(\sigma) \leq \pmax{1}$.  Such a $\delta$ can always be chosen because $\getpoly{1} \neq \bot$ which implies $\nzset{\delta} \neq \emptyset$.  We then apply (\ref{thm:pord:rhs}) to obtain $\forall \sigma \in \nzset{\delta}.\ \pmin{2} \leq \delta(\sigma) \leq \pmax{2}$, which we instantiate with $\sigma = \sigma_0$ to obtain $\pmin{2} \leq \delta(\sigma_0)$.  Combined with $\delta(\sigma_0) = \pmin{1}$, we then have $\pmin{2} \leq \pmin{1}$ as desired.
\end{proof}
}

\begin{definition}
Let $\overlap{\delta_1}{\delta_2} = \nzset{\delta_1} \cap \nzset{\delta_2}$.
\end{definition}

\begin{lemma}
\label{lem:pp:plus:overlap}
 If $\delta_1 \in \ppconc{\pp{1}}$ and $\delta_2 \in \ppconc{\pp{2}}$ then $\pessoverlap{\ppa}{\ppb} \leq \setsize{\overlap{\delta_1}{\delta_2}} \leq \optoverlap{\ppa}{\ppb}$.
\end{lemma}

\begin{proof}
We first note that for any sets $A, B$, it is the case that
$\setsize{A \cup B} = \setsize{A} + \setsize{B} - \setsize{A \cap B}$
(often called the ``inclusion-exclusion principle'').  Rearranging the
equation we also have $ \setsize{A \cap B} = \setsize{A} + \setsize{B}
- \setsize{A \cup B} $.

\deleted{
Utilizing deMorgan's laws for sets, we can use this property to compute the size
of intersections.  
Let $U = A \cup B$.  Note that $A \subseteq U$ and
$B \subseteq U$.  Let $\overline{A} = U - A$ and $\overline{B} = U -
B$.  More generally, let $\overline{e} = U - e$ for some set
expression $e$.  Then we have that $A \cap B
= \overline{\overline{A} \cup \overline{B}} = U -
(\overline{A} \cup \overline{B})$.  As
$\overline{A} \cup \overline{B} \subseteq U$, we have that
\[\setsize{A \cap B} = \setsize{U - (\overline{A} \cup \overline{B})} = \setsize{U} - \setsize{\overline{A} \cup \overline{B}}\]
Applying the rule above for the size of unions, we then obtain
\[\setsize{A \cap B} = \setsize{U} - (\setsize{\overline{A}} + \setsize{\overline{B}} - \setsize{\overline{A} \cap \overline{B}})\]
Distributing the first minus sign then yields the following.
\[\setsize{A \cap B} = \setsize{U} - \setsize{\overline{A}} - \setsize{\overline{B}} + \setsize{\overline{A} \cap \overline{B}})\]
Note that since $U = A \cup B$ we have that
$\overline{A} \cap \overline{B} = (U - A) \cap (U - B) = \emptyset$.
Simplifying and expanding our definition of $\overline{A}$ and
$\overline{B}$ then yields the following progression.
\begin{equation}
\label{lem:pp:plus:overlap:size-equality}
\begin{aligned}[b]
\setsize{A \cap B} &= \setsize{U} - \setsize{U - A} - \setsize{U - B} \\
&= \setsize{U} - \setsize{U} + \setsize{A} - \setsize{U} + \setsize{B} \\
&= \setsize{A} + \setsize{B} - \setsize{U}
\end{aligned}
\end{equation}
}

We will make use of this formula with $A = \nzset{\delta_1}$, $B
= \nzset{\delta_2}$. %, and $U = \nzset{\delta_1} \cup \nzset{\delta_2}$.

\paragraph{Lower Bound}

We first show the lower bound.  Expanding the definitions of $\pessoverlap{\ppa}{\ppb}$ and $\overlap{\delta_1}{\delta_2}$, this reduces to showing the following.
\begin{multline*}
\max((\smin{1} - n_1) + (\smin{2} - n_2) - n_3,\ 0)\\
{} \leq \setsize{\nzset{\delta_1} \cap \nzset{\delta_2}}
\end{multline*}
Clearly we have $0 \leq \setsize{\nzset{\delta_1} \cap \nzset{\delta_2}}$, so it remains to show that the following holds.
\begin{multline*}
(\smin{1} - n_1) + (\smin{2} - n_2) - n_3\\
{} \leq \setsize{\nzset{\delta_1} \cap \nzset{\delta_2}}
\end{multline*}

Expanding the definitions of $n_1,n_2$ from Definition \ref{def:abs-overlap}, we obtain
\begin{multline*}
(\smin{1} - (\psize{\getpoly{1}} - n_3)) + (\smin{2} - (\psize{\getpoly{2}} - n_3)) - n_3\\
{} \leq \setsize{\nzset{\delta_1} \cap \nzset{\delta_2}}
\end{multline*}
and rearranging yields the following.
\begin{multline*}
\smin{1} + \smin{2} - (\psize{\getpoly{1}} + \psize{\getpoly{2}} - n_3)\\
{} \leq \setsize{\nzset{\delta_1} \cap \nzset{\delta_2}}
\end{multline*}

This follows from the rearranged inclusion-exclusion principle provided we can show $\smin{1} \leq \setsize{\nzset{\delta_1}}$, $\smin{2} \leq \setsize{\nzset{\delta_2}}$, and $\psize{\getpoly{1}} + \psize{\getpoly{2}} - n_3 \geq \setsize{\nzset{\delta_1} \cup \nzset{\delta_2}}$.  The first two follow directly from our assumptions that $\delta_1 \in \ppconc{\pp{1}}$ and $\delta_2 \in \ppconc{\pp{2}}$.  For the third condition, we reason as follows.

We have from our assumptions that $\pconc{\getpoly{1}} \supseteq \nzset{\delta_1}$ and $\pconc{\getpoly{2}} \supseteq \nzset{\delta_2}$.  Thus, we have
\[\pconc{\getpoly{1}} \cup \pconc{\getpoly{2}} \supseteq \nzset{\delta_1} \cup \nzset{\delta_2}\]
and finally
\[\setsize{\pconc{\getpoly{1}} \cup \pconc{\getpoly{2}}} \geq \setsize{\nzset{\delta_1} \cup \nzset{\delta_2}}\]
Utilizing the inclusion-exclusion principle, we have
\begin{multline*}
\setsize{\pconc{\getpoly{1}}} + \setsize{\pconc{\getpoly{2}}} - \setsize{\pconc{\getpoly{1}} \cap \pconc{\getpoly{2}}}\\
{} \geq \setsize{\nzset{\delta_1} \cup \nzset{\delta_2}}
\end{multline*}
Since we have $\setsize{\pconc{\getpoly{}}} = \psize{\getpoly{}}$, we can rewrite this to the following.
\begin{multline*}
\psize{\getpoly{1}} + \psize{\getpoly{2}} - \setsize{\pconc{\getpoly{1}} \cap \pconc{\getpoly{2}}}\\
{} \geq \setsize{\nzset{\delta_1} \cup \nzset{\delta_2}}
\end{multline*}

It remains to show that
$\setsize{\pconc{\getpoly{1}} \cap \pconc{\getpoly{2}}} = n_3$.  We
have that $\pconc{\getpoly{1} \pmeet \getpoly{2}}
= \pconc{\getpoly{1}} \cap \pconc{\getpoly{2}}$ (that is, $\pmeet$ is
precise).  This allows us to complete the final step, concluding that
$n_3$, which is defined as $\psize{\getpoly{1} \pmeet \getpoly{2}}$ is
equal to $\setsize{\pconc{\getpoly{1}} \cap \pconc{\getpoly{2}}}$.

\paragraph{Upper Bound}

We next show that the upper bound holds.  Our goal is to show the
following.
\[ \optoverlap{\ppa}{\ppb} \geq \setsize{\overlap{\delta_1}{\delta_2}} \]
Expanding our definitions yields the following formula.
\[
\min(\smax{1}, \smax{2}, n_3)
{} \geq \setsize{\nzset{\delta_1} \cap \nzset{\delta_2}}
\]

We first note that the following holds.
\[\setsize{\nzset{\delta_1} \cap \nzset{\delta_2}} \leq \setsize{\nzset{\delta_1}} \leq \smax{1}\]
Thus $\smax{1}$ is a sound upper bound.  Similarly, we have
\[\setsize{\nzset{\delta_1} \cap \nzset{\delta_2}} \leq \setsize{\nzset{\delta_2}} \leq \smax{2}\]
which shows that $\smax{2}$ is a sound upper bound.  Finally, we note that our assumptions give us $\nzset{\delta_1} \subseteq \pconc{\getpoly{1}}$ and $\nzset{\delta_1} \subseteq \pconc{\getpoly{1}}$.  Thus we have the following.
\[\nzset{\delta_1} \cap \nzset{\delta_2} \subseteq \pconc{\getpoly{1}} \cap \pconc{\getpoly{2}}\]
We showed previously that $n_3 = \setsize{\pconc{\getpoly{1}} \cap \pconc{\getpoly{2}}}$.  Thus we have
\[\setsize{\nzset{\delta_1} \cap \nzset{\delta_2}} \leq n_3\]
which shows that $n_3$ is a sound upper bound.

Since all of $\smax{1}, \smax{2}$, and $n_3$ are sound upper bounds, their minimum is also a sound upper bound.
\end{proof}

\begin{lemma}
\label{lem:pp:plus:setsize}
\begin{multline*}
\setsize{\nzset{\delta_1 + \delta_2}} =\\
 \setsize{\nzset{\delta_1}} + \setsize{\nzset{\delta_2}} - \setsize{\overlap{\delta_1}{\delta_2}}
\end{multline*}
\end{lemma}

\begin{proof}
  First we note that $\nzset{\delta_1 + \delta_2} = \{\sigma \mid
  \delta_1(\sigma) + \delta_2(\sigma) > 0\}$.  Since the range of
  $\delta_1$ and $\delta_2$ is $[0,1]$, we have that $\delta_1(\sigma)
  + \delta_2(\sigma) > 0$ if and only if either $\delta_1(\sigma) > 0$
  or $\delta_2(\sigma) > 0$.  Thus, we have $\sigma \in
  \nzset{\delta_1 + \delta_2}$ if and only if $\sigma \in
  \nzset{\delta_1}$ or $\sigma \in \nzset{\delta_2}$, which implies
  $\nzset{\delta_1 + \delta_2} = \nzset{\delta_1} \cup
  \nzset{\delta_2}$.

  Next, we note that for any sets $A,B$ we have $\setsize{A \cup B} =
  \setsize{A} + \setsize{B} - \setsize{A \cap B}$.  Utilizing this
  statement with $A = \nzset{\delta_1}$ and $B = \nzset{\delta_2}$
  completes the proof.
\end{proof}

\begin{replemma}{lem:pp:plus}[Soundness of Plus]
If $\delta_1 \in \ppconc{\pp{1}}$ and $\delta_2 \in \ppconc{\pp{2}}$ then $\delta_1 + \delta_2 \in \ppconc{\pp{1} + \pp{2}}$.
\end{replemma}

\begin{proof}
Suppose $\delta_1 \in \ppconc{\pp{1}}$ and $\delta_2 \in \ppconc{\pp{2}}$.  Then we have the following.
\begin{gather}
\label{thm:pp:plus:cond1a}
\nzset{\delta_1} \subseteq \pconc{\getpoly{1}} \\
\label{thm:pp:plus:cond2a}
\smin{1} \leq \setsize{\nzset{\delta_1}} \leq \smax{1} \\
\label{thm:pp:plus:cond3a}
\mmin{1} \leq \pmass{\delta_1} \leq \mmax{1} \\
\label{thm:pp:plus:cond4a}
\forall \sigma \in \nzset{\delta_1}.\ \pmin{1} \leq \delta_1(\sigma) \leq \pmax{1}
\end{gather}
and
\begin{gather}
\label{thm:pp:plus:cond1b}
\nzset{\delta_2} \subseteq \pconc{\getpoly{2}} \\
\label{thm:pp:plus:cond2b}
\smin{2} \leq \setsize{\nzset{\delta_2}} \leq \smax{2} \\
\label{thm:pp:plus:cond3b}
\mmin{2} \leq \pmass{\delta_2} \leq \mmax{2} \\
\label{thm:pp:plus:cond4b}
\forall \sigma \in \nzset{\delta_2}.\ \pmin{2} \leq \delta_2(\sigma) \leq \pmax{2}
\end{gather}

The definition of abstract plus has special cases when either of the
arguments are zero, that is, if $ \iszero{\pp{1}} $ or
$ \iszero{\pp{2}} $. Without the loss of generality, let us assume
$ \iszero{\pp{2}} $ and thus by definition $ \pp{1} + \pp{2} = \pp{1}
$. Since $ \ppconc{\pp{2}} = \set{\distzero} $, where $ \distzero $ is
the distribution assigning probability of $ 0 $ to every state. Therefore $ \delta_2
= \distzero $ and thus $ \delta_1 + \delta_2 = \delta_1 $. But we
already have $ \delta_1 \in \ppconc{\pp{1}} $ by assumption, hence we
are done in this case.

In the case when not $ \iszero{\pp{1}} $ and not $ \iszero{\pp{2}} $
we must show the following.
\begin{gather}
\label{thm:pp:plus:cond1c}
\nzset{\delta_1 + \delta_2} \subseteq \pconc{\getpoly{1} \pjoin \getpoly{2}} \\
\label{thm:pp:plus:cond2c1}
\maxparen{\smin{1} + \smin{2} - \optoverlap{\ppa}{\ppb}, 0}\leq \setsize{\nzset{\delta_1 + \delta_2}}\\
\label{thm:pp:plus:cond2c2}
\setsize{\nzset{\delta_1 + \delta_2}} \leq \minparen{\smax{1}
+ \smax{2} - \pessoverlap{\ppa}{\ppb},  \psize{\getpoly{3}}} \\
\label{thm:pp:plus:cond3c}
\mmin{1} + \mmin{2}\leq \pmass{\delta_1 + \delta_2} \leq \mmax{1} + \mmax{2}
\end{gather}
\pxm{added the smin negativity check, doesn't really require anything
in the proof as $ 0 \leq $ the size of any set}
We also must show the conditions on $\pmin{}$ and $\pmax{}$ for the sum.


Condition (\ref{thm:pp:plus:cond1c}) follows from
(\ref{thm:pp:plus:cond1a}) and (\ref{thm:pp:plus:cond1b}) and the fact
that $\pjoin$ over-approximates union.  The key step is noting that
$\nzset{\delta_1 + \delta_2} = \nzset{\delta_1} \cup
\nzset{\delta_2}$.  To show this we consider some $\sigma \in
\nzset{\delta_1 + \delta_2}$.  We have that $(\delta_1 +
\delta_2)(\sigma) > 0$ which, expanding the definition of $+$, yields
$\delta_1(\sigma) + \delta_2(\sigma) > 0$.  Since the range of
$\delta_1$ and $\delta_2$ is $[0,1]$, this implies that either
$\delta_1(\sigma) > 0$ or $\delta_2(\sigma) > 0$ and thus $\sigma \in
\nzset{\delta_1}$ or $\sigma \in \nzset{\delta_2}$.

Conditions (\ref{thm:pp:plus:cond2c1}) and (\ref{thm:pp:plus:cond2c2})
follow from (\ref{thm:pp:plus:cond2a}) and (\ref{thm:pp:plus:cond2b})
and Lemmas \ref{lem:pp:plus:overlap} and \ref{lem:pp:plus:setsize}.
We have $\smin{1} \leq \setsize{\nzset{\delta_1}}$ from (\ref{thm:pp:plus:cond2a})
and $\smin{2} \leq \setsize{\nzset{\delta_2}}$ from (\ref{thm:pp:plus:cond2b}).
Monotonicity of addition then gives us
\[\smin{1} + \smin{2} \leq \setsize{\nzset{\delta_1}} + \setsize{\nzset{\delta_2}}\]
From Lemma \ref{lem:pp:plus:overlap} we have $\setsize{\overlap{\delta_1}{\delta_2}} \leq \optoverlap{\ppa}{\ppb}$ and thus
\[-\optoverlap{\ppa}{\ppb} \leq - \setsize{\overlap{\delta_1}{\delta_2}}\]
Combining with the above yields
\begin{multline*}
\smin{1} + \smin{2} - \optoverlap{\ppa}{\ppb} \leq {}\\
 \setsize{\nzset{\delta_1}} + \setsize{\nzset{\delta_2}} - \setsize{\overlap{\delta_1}{\delta_2}}
\end{multline*}
We can then rewrite the right-hand side according to Lemma \ref{lem:pp:plus:setsize} to obtain
\[\smin{1} + \smin{2} -\optoverlap{\ppa}{\ppb} \leq \setsize{\nzset{\delta_1 + \delta_2}}\]
which is condition (\ref{thm:pp:plus:cond2c1}).

Condition (\ref{thm:pp:plus:cond2c2}) follows the same reasoning.  We have $\setsize{\nzset{\delta_1}} + \setsize{\nzset{\delta_2}} \leq \smax{1} + \smax{2}$ by (\ref{thm:pp:plus:cond2a}) and (\ref{thm:pp:plus:cond2b}).  We then apply Lemma \ref{lem:pp:plus:overlap} and \ref{lem:pp:plus:setsize} to obtain condition (\ref{thm:pp:plus:cond2c2}).

For Condition (\ref{thm:pp:plus:cond3c}), note that
\[\pmass{\delta_1 + \delta_2} = \sum_\sigma\bigl(\delta_1(\sigma) + \delta_2(\sigma)\bigr) = \sum_\sigma \delta_1(\sigma) + \sum_\sigma \delta_2(\sigma)\]
This is then equivalent to $\pmass{\delta_1} + \pmass{\delta_2}$.
We have shown that $\pmass{\delta_1 + \delta_2} = \pmass{\delta_1} + \pmass{\delta_2}$.
Condition (\ref{thm:pp:plus:cond3c}) then follows from monotonicity of addition applied to (\ref{thm:pp:plus:cond3a}) and (\ref{thm:pp:plus:cond3b})

We now consider the $\pmin{}$ and $\pmax{}$ conditions.  Let $\pp{3} = \pp{1} + \pp{2}$ and $\delta_3 = \delta_1 + \delta_2$.  We must show.
\[\forall \sigma \in \nzset{\delta_3} \qsep \pmin{3} \leq \delta_3(\sigma) \leq \pmax{3}\]

The values $\pmin{3}$ and $\pmax{3}$ are defined by cases and we
consider these cases separately.  In one case, we have that $\pmin{}$
of the sum is $\min(\pmin{1},\pmin{2})$.  This is always a sound
choice.  To see why, suppose $\sigma \in \nzset{\delta_1 + \delta_2}$.
Then $\sigma \in \nzset{\delta_1}$ or $\sigma \in \nzset{\delta_2}$.
If $\sigma \in \nzset{\delta_1}$, then $(\delta_1+\delta_2)(\sigma) =
\delta_1(\sigma) + \delta_2(\sigma)$ is at least $\pmin{1}$.
Similarly, if $\sigma \in \nzset{\delta_2}$ then
$(\delta_1+\delta_2)(\sigma) \geq \delta_2(\sigma)$.

Similarly, the value $\pmax{1}+ \pmax{2}$ is always a sound choice for
$\pmax{3}$.  Consider $\sigma \in \nzset{\delta_3}$.  Then $\sigma \in
\nzset{\delta_1}$ or $\sigma \in \nzset{\delta_2}$.  If $\sigma \in
\nzset{\delta_1}$ and $\sigma \not\in \nzset{\delta_2}$, then we have
\[\delta_3(\sigma) = \delta_1(\sigma) + \delta_2(\sigma) = \delta_1(\sigma)\]
By (\ref{thm:pp:plus:cond4a}) we then have $\delta_3(\sigma) \leq
\pmax{1}$ and thus $\delta_3(\sigma) \leq \pmax{1} + \pmax{2}$ as
desired.

Similarly, if $\sigma \not\in \nzset{\delta_1}$ and $\sigma \in
\nzset{\delta_2}$ then by (\ref{thm:pp:plus:cond4b}) we have
\[\delta_3(\sigma) = \delta_2(\sigma) \leq \pmax{2} \leq \pmax{1} + \pmax{2}\]

Finally, if $\sigma \in \nzset{\delta_1}$ and $\sigma \in
\nzset{\delta_2}$ then by (\ref{thm:pp:plus:cond4a}) we have
$\delta_1(\sigma) \leq \pmax{1}$.  By (\ref{thm:pp:plus:cond4b}) we
have $\delta_2(\sigma) \leq \pmax{2}$.  Combining these we have
$\delta_1(\sigma) + \delta_2(\sigma) \leq \pmax{1} + \pmax{2}$ which
is equivalent to $\delta_3(\sigma) \leq \pmax{3}$ as desired.

Next we consider the $\pessoverlap{\ppa}{\ppb} = \psize{\getpoly{3}}$
case for $\pmin{3}$.  We must show that $\pmin{1} + \pmin{2}$ is a
sound lower bound on $\delta_3(\sigma)$ for
$\sigma \in \nzset{\delta_3}$.  We have by
Lemma \ref{lem:pp:plus:overlap} that
$\pessoverlap{\ppa}{\ppb} \leq \setsize{\overlap{\delta_1}{\delta_2}}$.
Since $\pessoverlap{\ppa}{\ppb} = \psize{\getpoly{3}}$ and
$\psize{\getpoly{3}} \geq \setsize{\overlap{\delta_1}{\delta_2}}$, we
have that $\psize{\getpoly{3}}
= \setsize{\overlap{\delta_1}{\delta_2}}$.  Expanding the definition
of $\overlap{\delta_1}{\delta_2}$ yields
\begin{equation}
\label{thm:pp:plus:cap-eq}
\setsize{\nzset{\delta_1} \cap \nzset{\delta_2}} = \psize{\getpoly{3}}
\end{equation}

We have from (\ref{thm:pp:plus:cond1c}) that $\nzset{\delta_1
+ \delta_2} \subseteq \pconc{\getpoly{3}}$ and from the proof of
(\ref{thm:pp:plus:cond1c}) we have that $\nzset{\delta_1 + \delta_2}
= \nzset{\delta_1} \cup \nzset{\delta_2}$.  Combining these yields
\[\setsize{\nzset{\delta_1} \cup \nzset{\delta_2}} \leq \psize{\getpoly{3}}\]
Combining this with (\ref{thm:pp:plus:cap-eq}) yields
\[\setsize{\nzset{\delta_1} \cup \nzset{\delta_2}} \leq \setsize{\nzset{\delta_1} \cap \nzset{\delta_2}}\]
For any sets $A, B$, we have that $\setsize{A \cup
B} \geq \setsize{A \cap B}$ and thus the above inequality implies the
following.
\[\setsize{\nzset{\delta_1} \cup \nzset{\delta_2}} = \setsize{\nzset{\delta_1} \cap \nzset{\delta_2}}\]
The fact that the size of the intersection and the size of the union
of $\nzset{\delta_1}$ and $\nzset{\delta_2}$ is identical implies that
$\nzset{\delta_1} = \nzset{\delta_2}$.  This implies that for all
$\sigma$, we have $\sigma \in \nzset{\delta_1}$ if and only if
$\sigma \in \nzset{\delta_2}$.

Now consider $\sigma \in \nzset{\delta_3}$.  We have
$\sigma \in \nzset{\delta_1}$ or $\sigma \in \nzset{\delta_2}$, as
before, but now we can strengthen this to
$\sigma \in \nzset{\delta_1}$ \emph{and}
$\sigma \in \nzset{\delta_2}$.  By (\ref{thm:pp:plus:cond4a}) we have
$\pmin{1} \leq \delta_1(\sigma)$ and by (\ref{thm:pp:plus:cond4b}) we
have $\pmin{2} \leq \delta_2(\sigma)$.  Thus we have
\[\pmin{1} + \pmin{2} \leq \delta_1(\sigma) + \delta_2(\sigma)\]
which was our goal.

Finally we consider the $\optoverlap{\ppa}{\ppb} = 0$ case for
$\pmax{3}$ (the ``otherwise'' case in Definition \ref{def:pplus}).
Consider a $\sigma \in \nzset{\delta_3}$.  We must show that
$\delta_3(\sigma) \leq \max(\pmax{1},\pmax{2})$.  We have that either
$\sigma \in \nzset{\delta_1}$ or $\sigma \in \nzset{\delta_2}$.  We
cannot have both since $\optoverlap{\ppa}{\ppb} = 0$ which, by
Lemma \ref{lem:pp:plus:overlap} implies that
$\setsize{\overlap{\delta_1}{\delta_2}} = 0$.  If
$\sigma \in \nzset{\delta_1}$ then by (\ref{thm:pp:plus:cond4a}) we
have $\delta_1(\sigma) \leq \pmax{1}$.  We have
$\sigma \not\in \nzset{\delta_2}$ and thus $\delta_2(\sigma) = 0$.
Thus we reason that
\[\delta_1(\sigma) + \delta_2(\sigma) = \delta_1(\sigma) \leq \pmax{1} \leq \max(\pmax{1},\pmax{2})\]
Similarly, if $\sigma \in \nzset{\delta_2}$ then we apply (\ref{thm:pp:plus:cond4b}) to obtain
\[\delta_1(\sigma) + \delta_2(\sigma) = \delta_2(\sigma) \leq \pmax{2} \leq \max(\pmax{1},\pmax{2})\]
\end{proof}

\subsection{Product}
\begin{replemma}{lem:pp:product}[Soundness of Product]
If $\delta_1 \in \ppconc{\pp{1}}$ and $\delta_2 \in \ppconc{\pp{2}}$
then $\delta_1 \times \delta_2 \in \ppconc{\pp{1} \times \pp{2}}$.
\end{replemma}

\begin{proof} By assumption, we have the following for $ i = 1,2 $.

\begin{align}
\label{thm:pp:prod:given1} & \support{\delta_i} \subseteq \pconc{\poly_i} \\
\label{thm:pp:prod:given2} & \smin{i} \leq \setsize{\support{\delta_i}} \leq \smax{i} \\
\label{thm:pp:prod:given3} & \mmin{i} \leq \pmass{\delta_i} \leq \mmax{i} \\
\label{thm:pp:prod:given4} & \forall \sigma \in \support{\delta_i} \;
. \; \pmin{i} \leq \delta(\sigma_i) \leq \pmax{i}
\end{align}

Let $ \delta_3 = \delta_1 \dprod \delta_2 $ and $ \pp{3}
= \pp{1} \pprod \pp{2} $. Recall the definition of $ \pp{3} $.
\[
\begin{array}{rcl@{\hspace{0.35cm}}|@{\hspace{0.35cm}}rcl}
\multicolumn{6}{c}{\poly_3\ =\ \poly_1 \pprod \poly_2} \bigstrut[b] \\
\pmin{3} &=& \pmin{1} \cdot \pmin{2} &
\pmax{3} &=& \pmax{1} \cdot \pmax{2} \bigstrut \\
\smin{3} &=& \smin{1} \cdot \smin{2} &
\smax{3} &=& \smax{1} \cdot \smax{2} \bigstrut \\
\mmin{3} &=& \mmin{1} \cdot \mmin{2} &
\mmax{3} &=& \mmax{1} \cdot \mmax{2} \bigstrut[t]
\end{array}
\]

We must show the following four claims.
\begin{align}
\label{thm:pp:prod:cond1}& \support{\delta_3} \subseteq \pconc{\poly_3} \\
\label{thm:pp:prod:cond2}& \smin{3} \leq \setsize{\support{\delta_3}} \leq \smax{3} \\
\label{thm:pp:prod:cond3}& \mmin{3} \leq \pmass{\delta_3} \leq \mmax{3} \\
\label{thm:pp:prod:cond4}& \forall \sigma \in \support{\delta_3} \;
. \; \pmin{3} \leq \delta_3(\sigma) \leq \pmax{3}
\end{align}

Also, recall the definition of concrete product.
$$ \delta_1 \dprod \delta_2 = \lambda
(\sigma_1, \sigma_2) \lsep \delta_1(\sigma_1) \cdot \delta_2(\sigma_2) $$

Let $ V_1 = \fv{\delta_1} $ and $ V_2 = \fv{\delta_2} $.

\begin{subproof}{\clref{thm:pp:prod:cond1} -- Support}
Let $ \sigma = (\sigma_1, \sigma_2) \in \support{\delta_3} $. Thus it must be that
$ \delta_1(\sigma_1) > 0 $ and
$ \delta_2(\sigma_2) > 0 $, thus, by \cnumref{thm:pp:prod:given1},
$ \sigma_1 \in \support{\delta_1} \subseteq \pconc{\poly_1}
$ and
$ \sigma_2 \in \support{\delta_2} \subseteq \pconc{\poly_2}
$, therefore $ \sigma \in \pconc{\delta_3} $.
\end{subproof}

\begin{subproof}{\clref{thm:pp:prod:cond2} -- Support points}
Using \cnumref{thm:pp:prod:given2} we get the following.
$$ \smin{1} \cdot \smin{2} \leq \setsize{\support{\delta_1}} \cdot \setsize{\support{\delta_2}} \leq \smax{1} \cdot \smax{2} $$

Likewise, the size of $ \support{\delta_3} $ can be equated as follows.
\begin{align*}
\setsize{\support{\delta_3}} & =
 \setsize{\set{(\sigma_1,
 \sigma_2) \left|
\begin{array}{l}
\sigma_1 \in \support{\delta_1}, \\
\sigma_2 \in \support{\delta_2}
\end{array} \right. } } \\
 & = \setsize{\support{\delta_1}} \cdot \setsize{\support{\delta_2}}
\end{align*}

This completes the claim as $ \smin{3} = \smin{1} \cdot \smin{2} $ and
$ \smax{3} = \smax{1} \cdot \smax{2} $.
\end{subproof}

\begin{subproof}{\clref{thm:pp:prod:cond3} -- Mass}
\begin{align*}
\pmass{\delta_3}
& = \sum_{\sigma \in \support{\delta_3}} \delta(\sigma) \\
& = \sum_{(\sigma_1, \sigma_2) \in \support{\delta_3}} \delta_1(\sigma_1) \cdot \delta_2(\sigma_2) \\
&
= \sum_{\sigma_1 \in \support{\delta_1}} \paren{\sum_{\sigma_2 \in \support{\delta_2}} \delta_1(\sigma_1) \cdot \delta_2(\sigma_2)} \\
&
= \sum_{\sigma_1 \in \support{\delta_1}} \delta_1(\sigma_1) \sum_{\sigma_2 \in \support{\delta_2}} \delta_2(\sigma_2) \\
&
= \sum_{\sigma_1 \in \support{\delta_1}} \delta_1(\sigma_1) \cdot \pmass{\delta_2} \\
& = \pmass{\delta_1} \cdot \pmass{\delta_2}
\end{align*}
\end{subproof}

Likewise, by \cnumref{thm:pp:prod:given3}, we have the following.
$$ \mmin{1} \cdot \mmin{2} \leq \pmass{\delta_1} \cdot \pmass{\delta_2} \leq \mmax{1} \cdot \mmax{2} $$

This completes the claim as $ \mmin{3} = \mmin{1} \cdot \mmin{2} $ and
$ \mmax{3} = \mmax{1} \cdot \mmax{2} $.

\begin{subproof}{\clref{thm:pp:prod:cond4} -- Probability}
Let $ \sigma = (\sigma_1, \sigma_2) \in \support{\delta_3} $. Thus
$ \sigma_1 \in \support{\delta_1} $ and
$ \sigma_2 \in \support{\delta_2} $. Also,
$ \delta_3\paren{\sigma}
= \delta_1(\sigma_1) \cdot \delta_2(\sigma_2)
$. By \cnumref{thm:pp:prod:given4}, we have
$ \pmin{1} \leq \delta_1(\sigma_1) \leq \pmax{1} $ and
$ \pmin{2} \leq \delta_2(\sigma_2) \leq \pmax{2} $. Therefore
$$ \pmin{3}
= \pmin{1} \cdot \pmin{2} \leq \delta_3(\sigma) \leq \pmax{1} \cdot \pmax{2}
= \pmax{3} $$

This completes the claim and the proof.
\end{subproof}

\end{proof}

\subsection{Conditioning}

\newcommand{\scond}[2]{#1 | #2}
\newcommand{\scondc}[2]{#1 | \overline{#2}}

\begin{definition} Given a set of states $ S $ and a boolean
expression $ \bexp $, let $ \scond{S}{\bexp} $ be the subset of $ S $
that satisfy the condition $ \bexp $ and $ \scondc{S}{\bexp} $ be the
subset of $ S $ that do not satisfy the condition. Formally,
\begin{align*}
\scond{S}{\bexp} & \defeq \set{\sigma \in S \given \eeval{\bexp}{\sigma}
= \atrue} \\ 
\scondc{S}{\bexp} & \defeq \set{\sigma \in S \given \eeval{\bexp}{\sigma}
= \afalse} 
\end{align*}

\end{definition}

\begin{replemma}{lem:pp:cond}[Soundness of Conditioning] If $ \delta \in \ppconc{\pp{}} $ then $
  \dcond{\delta}{B} \in \ppconc{\absdcond{\pp{}}{B}} $.
\end{replemma}

\newcommand{\supd}[0]{\support{\delta}}
\newcommand{\supin}[0]{\scond{\support{\delta}}{\bexp}}
\newcommand{\supout}[0]{\scondc{\support{\delta}}{\bexp}}

\begin{proof} Let $ \delta_2 = \dcond{\delta}{B} $. Recall the
definition of the conditional distribution:
\begin{align*}
\dcond{\delta}{\bexp} = \lambda \sigma \lsep \aif \eeval{\bexp}{\sigma} \athen
\delta(\sigma) \aelse 0
\end{align*}

Let $ \pp{2} = \absdcond{\pp{}}{B} $. The construction of $ \pp{2} $
produces the following parameters.
\[
\setlength{\arraycolsep}{1pt}
\begin{array}{lcl@{\hspace{0.35cm}}|@{\hspace{0.35cm}}lcl}
\pmin{2} &=& \pmin{} &
\smin{2}\ &=&\ \maxparen{\smin{} - \nn, 0}\\
\pmax{2} &=& \pmax{} \bigstrut[b] &
\smax{2} &=&\ \minparen{\smax{}, n} \bigstrut \\
\mmin{2} &=& \multicolumn{4}{l}{\maxparen{\pmin{2} \cdot
    \smin{2},\ \mmin{} - \pmax{} \cdot \minparen{\smax{}, \nn}}} \bigstrut\\
\mmax{2} &=& \multicolumn{4}{l}{\minparen{\pmax{2} \cdot
    \smax{2},\ \mmax{} - \pmin{} \cdot \maxparen{\smin{}-n, 0}}} \bigstrut\\
\getpoly{2} &=& \multicolumn{4}{l}{\abseeval{\bexp}{\getpoly{}}} \bigstrut[t] \\
\end{array}
\]
\vspace{20mm} % !!! formatting

The quantities $ n $ and $ \nn $ are defined in such a way that $ n $
over-approximates the number of support points of $ \delta $ that
satisfy $ \bexp $, whereas $ \nn $ over-approximates the number of support
points of $ \delta $ that do not satisfy $ \bexp $. Also,
$ \abseeval{\bexp}{\getpoly{}} $ is defined to contain at least 
the points in $ C $ that satisfy $ \bexp $. Making these properties
precise gives us the following.
\begin{align}
\label{thm:pp:cond:prop1} \setsize{\supin} & \leq n \\
\label{thm:pp:cond:prop2} \setsize{\supout} & \leq \nn \\
\label{thm:pp:cond:prop3} \scond{\pconc{\getpoly{}}}{\bexp} & \subseteq \pconc{\abseeval{\bexp}{\getpoly{}}}
\end{align}

By assumption we have the following.
\begin{align}
\label{thm:pp:cond:given1} & \support{\delta} \subseteq \pconc{C} \\
\label{thm:pp:cond:given2} & \smin{} \leq \setsize{\support{\delta}} \leq \smax{} \\
\label{thm:pp:cond:given3} & \mmin{} \leq \pmass{\delta} \leq \mmax{} \\
\label{thm:pp:cond:given4} & \forall \sigma \in \support{\delta} \;
. \; \pmin{} \leq \delta(\sigma) \leq \pmax{}
\end{align}

We need to show the following four claims.
\begin{align}
\label{thm:pp:cond:cond1}& \support{\delta_2} \subseteq \pconc{\getpoly{2}} \\
\label{thm:pp:cond:cond2}& \smin{2} \leq \setsize{\support{\delta_2}} \leq \smax{2} \\
\label{thm:pp:cond:cond3}& \mmin{2} \leq \pmass{\delta_2} \leq \mmax{2} \\
\label{thm:pp:cond:cond4}& \forall \sigma \in \support{\delta_2} \;
. \; \pmin{2} \leq \delta_2(\sigma) \leq \pmax{2}
\end{align}

\begin{subproof}{\clref{thm:pp:cond:cond1} -- Support} Let
$ \sigma \in \support{\delta_2} $. Thus it must be that
$ \sigma \in \support{\delta} $ and $ \eeval{\bexp}{\sigma} = \atrue
$. By $ \cnumref{thm:pp:cond:given1} $, we have
$ \sigma \in \pconc{\getpoly{}} $, therefore
$ \sigma \in \pconc{\getpoly{2}} $ as
$ \set{\sigma \in \pconc{\getpoly{}} \given \eeval{\bexp}{\sigma}
= \atrue} \subseteq \pconc{\getpoly{2}} $ by construction of
$ \getpoly{2} $.
\end{subproof}

\newcommand{\supdn}[0]{\support{\delta_2}}

\begin{subproof}{\clref{thm:pp:cond:cond2} -- Support points} Let us write
$ \supd $ as a union of two disjoint sets.
$$ \supd = \supin \cup \supout $$

Given the disjointness of the two, we also have the following.
$$ \setsize{\supd} = \setsize{\supin} + \setsize{\supout} $$

Now note that $ \supdn = \supin $. Thus we can write
$ \setsize{\supdn} = \setsize{\supd} - \setsize{\supout} $. We can
therefore estimate the size of the support of $ \delta_2 $ in the
following manner.
\begin{align*}
\setsize{\supdn} & = \setsize{\supd} - \setsize{\supout} \\
& \leq \setsize{\supd} \\
& \leq \smax{} & \mathstep{by \cnumref{thm:pp:cond:given2}}
\end{align*}

Therefore, using \cnumref{thm:pp:cond:prop1} and the above, we have
$ \setsize{\supdn} \leq \minparen{\smax{}, n} = \smax{2} $.

Going in the other direction, we can write as follows.
\begin{align*}
\setsize{\supdn} & = \setsize{\supd} - \setsize{\supout} \\
 & \geq \smin{} - \setsize{\supout}
 & \mathstep{by \cnumref{thm:pp:cond:given2}} \\
 & \geq \smin{} - \nn & \mathstep{by \cnumref{thm:pp:cond:prop2}}
\end{align*}

Since all sets are trivially of size at least $ 0 $, we have
$ \setsize{\supdn} \geq \maxparen{\smin{} - \nn, 0} = \smin{2} $.
\end{subproof}

\begin{subproof}{\clref{thm:pp:cond:cond4} -- Probability} Note that we will show
the probability claim before the mass as we will use the truth of the
probability claim in the mass arguments.

Let $ \sigma \in \supdn $. By definition of $ \delta_2 $, we have
$ \delta_2(\sigma) = \delta(\sigma) $. Thus $ \sigma \in \supd $ so
by \cnumref{thm:pp:cond:given4} we have:
$$ \pmin{2} = \pmin{} \leq \delta(\sigma) = \delta_2(\sigma) \leq \pmax{} = \pmax{2} $$
\end{subproof}

\begin{subproof}{\clref{thm:pp:cond:cond3} -- Mass} Let us first show the
following bound on the size of $ \supout $.
\begin{align}
\label{thm:pp:cond:prop1c} \maxparen{\smin{} - n,
0} \leq \setsize{\supout} \leq \minparen{\smax{}, \nn}
\end{align}

Since $ \setsize{\supd} = \setsize{\supin} + \setsize{\supout} $, we
can say $ \setsize{\supout} = \setsize{\supd} - \setsize{\supin} $ and
continue to the bound in the following manner.
\begin{align*}
\setsize{\supout} & = \setsize{\supd} - \setsize{\supin} \\
& \geq \smin{} - \setsize{\supin}
& \mathstep{by \cnumref{thm:pp:cond:given2}} \\
& \geq \smin{} - n & \mathstep{by \cnumref{thm:pp:cond:prop1}}
\end{align*}

Therefore $ \setsize{\supout} \geq \maxparen{\smin{} - n, 0} $ as
claimed. For the other end of the inequality, note that we have
$ \setsize{\supout} \leq \setsize{\supd} \leq \smax{} $
by \cnumref{thm:pp:cond:given2}. Also,
by \cnumref{thm:pp:cond:prop2}, $ \setsize{\supout} \leq \nn $. Therefore
$ \setsize{\supout} \leq \maxparen{\smax{}, \nn} $, completing our
bound.

Now, let us write $ \pmass{\delta} $ in two parts.

\begin{align*}
\pmass{\delta} & = \sum_{\sigma \in \support{\delta}} \delta(\sigma) \\
               & = \sum_{\sigma \in \supin} \delta(\sigma)
               + \sum_{\sigma \in \supout} \delta(\sigma) \\
               & = \pmass{\delta_2} + \sum_{\sigma \in \supout} \delta(\sigma)
\end{align*}

Therefore $ \pmass{\delta_2} = \pmass{\delta}
- \sum_{\sigma \in \supout} \delta(\sigma) $.

\begin{align*}
\pmass{\delta_2}
& = \pmass{\delta} - \sum_{\sigma \in \supout} \delta(\sigma) \\
& \leq \mmax{} - \sum_{\sigma \in \supout} \delta(\sigma)
& \mathstep{by \cnumref{thm:pp:cond:given3}} \\
& \leq \mmax{} - \sum_{\sigma \in \supout} \pmin{}
 & \mathstep{by \cnumref{thm:pp:cond:given4}} \\
& = \mmax{} - \setsize{\supout} \cdot \pmin{} \\
& \leq \mmax{} - \maxparen{\smin{} - n, 0} \cdot \pmin{}
& \mathstep{by \cnumref{thm:pp:cond:prop1c}}
\end{align*}

Also, we can bound the mass using our other already proven conditions.
\begin{align*}
\pmass{\delta_2} & = \sum_{\sigma \in \supdn} \delta_2(\sigma) \\
                 & \leq \sum_{\sigma \in \supdn} \pmax{2}
                 & \mathstep{by \cnumref{thm:pp:cond:cond4}} \\
                 & = \setsize{\supdn} \cdot \pmax{2} \\
                 & \leq \smax{2} \cdot \pmax{2} & \mathstep{by \cnumref{thm:pp:cond:cond2}}
\end{align*}

Combining the bounds, we have half of our probability condition.
\begin{align*}
\pmass{\delta_2} & \leq \mmax{2} \\
& = \minparen{\pmax{2} \cdot \smax{2}, \mmax{} - \pmin{} \cdot \maxparen{\smin{} - n, 0 }}
\end{align*}

For the other half, we proceed similarly.
\begin{align*}
\pmass{\delta_2}
& = \pmass{\delta} - \sum_{\sigma \in \supout} \delta(\sigma) \\
& \geq \mmin{} - \sum_{\sigma \in \supout} \delta(\sigma)
& \mathstep{by \cnumref{thm:pp:cond:given3}} \\
& \geq \mmin{} - \sum_{\sigma \in \supout} \pmax{}
 & \mathstep{by \cnumref{thm:pp:cond:given4}} \\
& = \mmin{} - \setsize{\supout} \cdot \pmax{} \\
& \geq \mmin{} - \minparen{\smax{}, \nn} \cdot \pmax{}
& \mathstep{by \cnumref{thm:pp:cond:prop1c}}
\end{align*}

And likewise another bound using our other conditions.
\begin{align*}
\pmass{\delta_2} & = \sum_{\sigma \in \supdn} \delta_2(\sigma) \\
                 & \geq \sum_{\sigma \in \supdn} \pmin{2}
                 & \mathstep{by \cnumref{thm:pp:cond:cond4}} \\
                 & = \setsize{\supdn} \cdot \pmin{2} \\
                 & \geq \smin{2} \cdot \pmin{2} & \mathstep{by \cnumref{thm:pp:cond:cond2}}
\end{align*}

Combining the two bounds, we have the final element of our proof.
\begin{align*}
\pmass{\delta_2} & \geq \mmin{2} \\
& = \maxparen{\pmin{2} \cdot \smin{2}, \mmin{} - \pmax{} \cdot \minparen{\smax{}, \nn}}
\end{align*}
\end{subproof}
\end{proof}

\subsection{Scalar product}
\begin{replemma}{lem:pp:scalar-prod}
If $\delta_1 \in \ppconc{\pp{1}}$ then $p \cdot \delta_1 \in \ppconc{p \cdot \pp{1}}$.
\end{replemma}

\begin{proof} By assumption we have the following.
\begin{align*}
%\label{thm:pp:scalar:given1} 
& \support{\delta_1} \subseteq \pconc{\poly_1} \\
%\label{thm:pp:scalar:given2} 
& \smin{1} \leq \setsize{\support{\delta_1}} \leq \smax{1} \\
%\label{thm:pp:scalar:given3} 
& \mmin{1} \leq \pmass{\delta_1} \leq \mmax{1} \\
%\label{thm:pp:scalar:given4} 
& \forall \sigma \in \support{\delta_1} \;
. \; \pmin{1} \leq \delta_1(\sigma) \leq \pmax{1}
\end{align*}

Let $ \delta_2 = p \cdot \delta_1 $ and $ \pp{2} = p \cdot \pp{1} $. Let us assume that $ p \neq 0 $. In this case we need to show the following.
\begin{align*}
& \support{\delta_1} = \support{\delta_2} \subseteq \pconc{\poly_2} = \pconc{\poly_1} \\
& \smin{1}
= \smin{2} \leq \setsize{\support{\delta_2}}
= \setsize{\support{\delta_2}} \leq \smax{2} = \smax{1} \\
& p \cdot \mmin{1}
= \mmin{2} \leq \pmass{\delta_1} \leq \mmax{2} = p \cdot \mmax{1} \\
& \forall \sigma \in \support{\delta_2} \;
. \; \\
& \;\;\;\;\; p \cdot \pmin{1} = \pmin{2} \leq \delta_2(\sigma) \leq \pmax{2} =
p \cdot \pmax{1}
\end{align*}

The first two conditions are trivially satisfied given the lack of
change in the various parameters. For the mass condition, note that
$ \pmass{\delta_2} = \sum_{\sigma} \delta_2 (\sigma) = \sum_{\sigma}
p \cdot \delta_1(\sigma) = p \cdot \pmass{\delta_1} $. The probability
condition is also trivially satisfied as $ \delta_2(\sigma) =
p \cdot \delta_1(\sigma) $.

In the case that $ p = 0 $, the abstract scalar product is defined
with $ \smin{2} = \smax{2} =
\pmin{2} = \pmax{2} = \mmin{2} = \mmax{2} = 0 $ and $ \poly_2 =
\emptypoly $. In this case note that $ \support{\delta_2} = \emptyset
= \pconc{\emptypoly} $, and thus the conditions hold trivially.
\end{proof}

\subsection{Uniform}
\begin{lemma}[Soundness of Uniform] \label{lem:pp:uniform} If $ \delta \in \ppconc{\pp{}} $
and $ S = \suniform{x}{n_1}{n_2} $ then
$ \pevalp{S}{\delta} \in \ppconc{\abspevalp{S}{\pp{}}} $.
\end{lemma}

\begin{proof} Recall the semantics of the statement.
$$ \pevalp{\suniform{x}{n_1}{n_2}}{\delta} = \paren{\project{\delta}{\fv{\delta}
- \set{x}}} \dprod \delta_2 $$

The distribution $ \delta_2 $ is defined as follows.
$$ \delta_2 = \lambda \sigma \lsep \aif n_1 \leq \sigma(x) \leq
n_2 \athen \frac{1}{n_2 - n_1 + 1} \aelse 0 $$

The abstract semantics are similar.
\begin{align*}
\abspevalp{\suniform{x}{n_1}{n_2}}{\pp{}} &
= \paren{\forget{x}{\pp{}}} \pprod \pp{2}
\end{align*}

Here $ \pp{2} $ is defined with $ \pmin{2} = \pmax{2} = \frac{1}{n_2 -
n_1 + 1} $, $ \smin{2} = \smax{2} = n_2 - n_1 + 1 $, $ \mmin{2}
= \mmax{2} = 1 $, and $ \poly_2 = (\set{x \geq n_1, x \leq
n_2}, \set{x}) $.

By construction, we have $ \delta_2 \in \pp{2} $
thus the lemma follows from \lref{lem:pp:forget} (Soundness of
Forget) and \lref{lem:pp:product} (Soundness of Product).
\end{proof}

\subsection{While loops}

%Before we take care of loops we need to establish some facts and make
%some technical points about the abstract semantics. Firstly,
%throughout the abstract interpretation we will always make sure the
%probabilistic polyhedra follow certain invariants. 

\begin{definition} First we have some preliminary definitions. Given
some set of variables, we have the following, where each distribution
or state in each statement is understood
to be defined over the same set of variables.
\begin{itemize} 
%\item{} We say two probabilistic polyhedra are equal, or
%$ \pp{1} \ppeq \pp{2} $ iff $ \ppconc{\pp{1}} = \ppconc{\pp{2}}
%$. The usual equality, on the other hand, denotes that the parameters
%defining $ \pp{1} $ and $ \pp{2} $ are equal. The usual equality
%certainly implies $ \ppeq $, but the inverse doesn't hold.
\item{} Two distributions are \emph{ordered}, or
$ \delta_1 \distleq \delta_2 $ iff for every state $ \sigma $,
$ \delta_1(\sigma) \leq \delta_2(\sigma) $.
\item{} Two probabilistic polyhedra are \emph{ordered}, or
$ \pp{1} \ppleq \pp{2} $ iff for every $ \delta_1 \in \ppconc{\pp{1}}
$, there exists $ \delta_2 \in \ppconc{\pp{2}} $ with
$ \delta_1 \leq \delta_2 $.
%\item{} Two probabilistic polyhedra are \emph{strictly ordered}, or
%$ \pp{1} \pple \pp{2} $ iff $ \pp{1} \ppleq \pp{2} $ and
%$ \pp{1} \ppneq \pp{2} $.
\item{} The \emph{zero distribution} $ \delta $ is the unique
distribution with $ \delta(\sigma) = 0 $ for every $ \sigma $. We will
use $ \distzero $ to refer to this distribution.
\item{} A \emph{zero probabilistic polyhedron} $ \pp{} $, or
$ \iszero{\pp{1}} $ is one whose
concretization contains only the zero distribution, that is
$ \ppconc{\pp{}} = \set{\distzero} $.
\end{itemize}
\end{definition}

%\begin{remark}
%If $ \pp{} $ is a probabilistic polyhedron then we will maintain the following.
%
%\begin{itemize}
%\item{} \emph{Unique zero element}. If $ \smax{} = 0 $ or $ \pmax{} = 0 $ or
%$ \mmax{} = 0 $ or $ \poly = \emptypoly $ then $ \smax{} = 0 $ and
%$ \pmax{} = 0 $ and $ \mmax{} = 0 $ and $ \poly = \emptypoly $. We
%will refer to this element as simply $ \ppzero $.
%\item{} \emph{Consistent support size}. The bound on the number of
%support points is consistent with the size of the convex polyhedron
%bounding the support set, or formally, $ \smax{} \leq \psize{\poly} $.
%\end{itemize}
%
%Note that we can easily maintain these properties, by straightforward
%transformations that do not at all change the concretization of the
%abstract element. That is, from any $ \pp{1} $ we can create $ \pp{2} $
%with $ \pp{1} \ppeq \pp{2} $ where $ \pp{2} $ satisfies the
%two conditions above.
%
%To satisfy the unique zero element condition, we transform any
%$ \pp{1} $ with at least one of the zero conditions to an element
%$ \pp{2} $ with all the zero conditions. This doesn't change the
%concretization as either way, there is only a unique
%$ \delta \in \ppconc{\pp{1}} = \ppconc{\pp{2}} $, the zero $ \delta $,
%which has the property $ \delta(\sigma) = 0 $ for every $ \sigma $. We
%will refer to this $ \delta $ as simply $ \distzero $.
%
%To maintain the support size consistency, we merely reduce the size of
%$ \smax{1} $ to that of $ \psize{\poly_1} $ if it is ever larger than
%this amount. Note that this too doesn't change the concretization as
%we have $ \support{\delta} \subseteq \pconc{\poly_1} $, thus any
%$ \delta \in \ppconc{\pp{1}} $ will already have
%$ \setsize{\support{\delta}} \leq \psize{\poly_1} = \smax{2} $.
%
%\end{remark}
%
%From here on, we will assume that every probabilistic polyhedra
%satisfies our invariants.

\begin{lemma}\label{lem:pp:while:order} Let $ \pp{i} $ be consistent
probabilistic polyhedra, that is, $ \ppconc{\pp{i}} \neq \emptyset $.
%\begin{subtheorem}
\label{lem:pp:while:order:pluszero} Then, $ \pp{1} + \pp{2} \ppleq \pp{1} $ iff $ \iszero{\pp{2}} $.
%\end{subtheorem}
%\begin{subtheorem} \label{lem:pp:while:order:plusgeq} $ \pp{1} \ppleq \pp{1} + \pp{2} $.
%\end{subtheorem}
%\begin{subtheorem} \label{lem:pp:while:order:plusge} $ \pp{1} \pple \pp{1} + \pp{2} $ iff $ \pp{2} = \ppzero $.
%\end{subtheorem}
%\begin{subtheorem} \label{lem:pp:while:order:sandwich} If $ \pp{1} \ppleq \pp{2} \ppleq \pp{1} $ then
%$ \pp{2} \ppeq \pp{1} $.
%\end{subtheorem}
\end{lemma}

\begin{proof}
%\begin{subproof}{\clref{lem:pp:while:order:pluszero}}
In the
forward direction, we have $ \pp{1} + \pp{2} \ppleq \pp{1} $. Now, let
us consider a $ \pp{2} $ with not $ \iszero{\pp{2}} $. Thus there is
$ \delta_2 \in \ppconc{\pp{2}} $ with $ \pmass{\delta_2} > 0 $. Let
$ \delta_1 \in \ppconc{\pp{1}} $ be the distribution in
$ \ppconc{\pp{1}} $ maximizing mass, that is
$ \pmass{\delta_1} \geq \pmass{\delta'_1} $ for every
$ \delta'_1 \in \ppconc{\pp{1}} $. By \lref{lem:pp:plus}, $ \delta_1
+ \delta_2 \in \ppconc{\pp{1} + \pp{2}} $ and by the definition of
$ \pp{1} + \pp{2} \ppleq \pp{1} $, there must be $ \delta_3 \in \ppconc{\pp{1}}
$ with $ \delta_1 + \delta_2 \distleq \delta_3 $. Thus
$ \pmass{\delta_3} \geq \pmass{\delta_1 + \delta_2} = \pmass{\delta_1}
+ \pmass{\delta_2} > \pmass{\delta_1} $. This contradicts that
$ \delta_1 $ was mass maximizing in $ \ppconc{\pp{1}} $.

In the backward direction, our definition of abstract plus makes
$ \pp{1} + \pp{2} $ identical to $ \pp{1} $. Thus $ \pp{1} + \pp{2}
= \pp{1} \ppleq \pp{1} $.
%\end{subproof}

%\begin{subproof}{\clref{lem:pp:while:order:pluszero}} In the forward
%direction, let us assume $ \pp{1} + \pp{2} \ppeq \pp{1} $ but
%$ \pp{2} \neq \ppzero $. Thus we have $ \delta_2 \in \ppconc{\pp{2}} $
%with $ \pmass{\delta_2} > 0 $. Let $ \delta_1 $ be the element of
%$ \ppconc{\pp{1}} = \ppconc{\pp{1} + \pp{2}} $ that maximizes mass
%over all the distributions in the
%concretization. By \lref{lem:pp:plus}, we have $ \delta_1
%+ \delta_2 \in \ppconc{\pp{1} + \pp{2}} = \ppconc{\pp{1}} $ but
%$ \pmass{\delta_1 + \delta_2} = \pmass{\delta_1} + \pmass{\delta_2}
%> \pmass{\delta_1} $ which contradicts the fact that $ \delta_1 $ had the
%maximal mass.
%
%In the backwards direction, let us assume that $ \pp{2} = \ppzero
%$. It is immediate that $ \pp{1} + \pp{2} = \pp{1} $, given the
%special condition in the abstract plus construction.
%\end{subproof}

%\begin{subproof}{\clref{lem:pp:while:order:plusgeq}} Let
%$ \delta_1 \in \ppconc{\pp{1}} $ and let us take any
%$ \delta_2 \in \ppconc{\pp{2}} $. By \lref{lem:pp:plus}, we have
%$ \delta_1 + \delta_2 \in \ppconc{\pp{1} + \pp{2}} $. Also, $
%(\delta_1 + \delta_2) \geq \delta_1 $ as $ (\delta_1
%+ \delta_2)(\sigma) = \delta_1(\sigma)
%+ \delta_2(\sigma) \geq \delta_1(\sigma) $ for every $ \sigma $. Thus $ \pp{1} \ppleq \pp{1}
%+ \pp{2} $.
%Let $ \pp{3}
%= \pp{1} + \pp{2} $. We need to show the following.
%\begin{align*}
%\poly_1 & \pleq \poly_3 \\
%\mmax{1} & \leq \mmax{3} \\
%\pmax{1} & \leq \pmax{3} \\
%\smax{1} & \leq \smax{3}
%\end{align*}
%
%The first condition holds as per definition of $ \pjoin $ as $ \poly_3
%= \poly_1 \pjoin \poly_2 $. The total mass $ \mmax{3} $ is equal to
%$ \mmax{1} + \mmax{2} \geq \mmax{1} $, thus this condition is also satisfied.
%
%The max probability per point $ \pmax{3} $ as per definition of
%abstract plus is either $ \pmax{1} $ (whenever $ \mmax{2} = 0 $),
%$ \pmax{1} + \pmax{2} $, or $ \maxparen{\pmax{1}, \pmax{2}} $. In all
%cases, this quantity at least as large as $ \pmax{1} $.
%
%Finally consider $ \smax{3} $. If $ \mmax{2} = 0 $ then $ \smax{3}
%= \smax{1} $ and we are done. Otherwise we expand the definition of
%$ \smax{3} $ including the overlap.
%
%\begin{align*}
%\smax{3} & = \smax{1} + \smax{2} - \pessoverlap{\ppa}{\ppb} \\
% & = \smax{1}
% + \smax{2} - \maxparen{(\smin{1} - n_1) + (\smin{2} - n_2) - n_3,
% 0} \\
% & = \smax{1} + \smax{2} - \\
% & \;\;\; \maxparen{(\smin{1}
% - \psize{\poly_1} + n_3) + (\smin{2} - \psize{\poly_2} + n_3) - n_3,
% 0} \\
% & = \smax{1} + \smax{2} - \\
% & \;\;\; \maxparen{(\smin{1}
% - \psize{\poly_1} + n_3) + (\smin{2} - \psize{\poly_2}), 0}
%\end{align*}
%
%\end{subproof}

%\begin{subproof}{\clref{lem:pp:while:order:plusge}}
%\end{subproof}
%\begin{subproof}{\clref{lem:pp:while:order:sandwich}}
%\end{subproof}
\end{proof}

\begin{definition} Given a statement $ S = \swhile{\bexp}{S'} $, a distribution $ \delta $
and a probabilistic polyhedron $ \pp{} $, let us define a few
useful items.

\begin{itemize}
\item{} $ \omega(f) \defeq \lambda \delta \lsep
f(\pevalp{S'}{\paren{\dcond{\delta}{\bexp}}}) + \dcond{\delta}{\neg \bexp}
$
\item{} $ \delta_1 \defeq \delta $
\item{} $ \delta_{i+1} \defeq \pevalp{S'}{\paren{\dcond{\delta_i}{\bexp}}} $
\item{}
$ \Delta_n \defeq \sum_{i=1}^{n} \paren{\dcond{\delta_i}{\neg \bexp}}
$
\item{} $ \distfunzero $ is the function that takes in any
distribution and produces the zero distribution $ \distzero $, that is
$ \distfunzero(\delta) = \distzero $.
\end{itemize}
Similarly we have the abstract versions of the definitions.
\begin{itemize}
\item{} $ \Omega(F) \defeq \lambda \pp{} \lsep
F(\abspevalp{S'}{\paren{\absdcond{\pp{}}{\bexp}}}) + \absdcond{\pp{}}{\neg \bexp}
$
\item{} $ \pp{1} \defeq \pp{} $
\item{} $ \pp{i+1} \defeq \abspevalp{S'}{\paren{\absdcond{\pp{i}}{\bexp}}} $
\item{}
$ \Phi_n \defeq \sum_{i=1}^{n} \paren{\absdcond{\pp{i}}{\neg \bexp}} $
\item{} $ \ppfunzero $ is a function that takes in any
probabilistic polyhedron and produces a zero probabilistic polyhedron, that is
$ \iszero{\ppfunzero(\pp{})} $ for every $ \pp{} $.
\end{itemize}

\end{definition}

The semantics of while loops are defined as such:
$$ \pevalp{S}{} = \pevalp{\swhile{\bexp}{S'}}{} = \lfp(\omega) $$
$$ \abspevalp{S}{} = \abspevalp{\swhile{\bexp}{S'}}{} = \lfp(\Omega) $$

While such definitions are of theoretical interest, they are not
particularly useful for implementations, given our lack of a widening
operator. Thus, our security checks will always be conditioned on
termination of the abstract interpretation, defined below.
We show that  termination of the abstract interpretation implies termination
of all corresponding concrete executions.  This is crucial, as our
concrete semantics (due to Clarkson et
al.~\cite{clarkson09quantifying}) assumes termination to avoid leaks.
To make this termination condition explicit,
we provide an alternate concrete
semantics for terminating while loops and show that this
gives results equivalent to those of the original semantics.

\begin{definition} The termination of $ \pevalp{S}{\delta} $ is defined
as follows.
\begin{itemize}
\item{} If $ S $ is an elementary statement (assignment, skip,
uniform), then $ \pevalp{S}{\delta} $ terminates.
\item{} If $ S $ is a sequence, if statement, or a probabilistic
choice statement, then $ \pevalp{S}{\delta} $ terminates iff the
various evaluations steps to evaluate $ S $ terminate. This depends on
the statement type, for $ S = \sseq{S_1}{S_2} $, for example, it means
that $ \pevalp{S_1}{\delta} $ terminates and so does
$ \pevalp{S_2}{\paren{\pevalp{S_1}{\delta}}} $.
\item{} If $ S = \swhile{\bexp}{S_1} $ is a while statement, then $ \pevalp{S}{\delta} $
terminates iff there exists $ n $ with $ \delta_n = \distzero $ and the evaluation steps as per definition of $ \delta_i
$ terminate for all $ i $ up to $ n $.
\end{itemize}

The termination of $ \abspevalp{S}{\pp{}} $ is framed similarly,
except in the while case, we require the existence of $ n $ with
$ \iszero{\pp{n}} $ and the termination of the abstract evaluations as
in the definitions of $ \pp{i} $ for all $ i $ up to $ n $.
\end{definition}

The $ \Delta_i $ and $ \Phi_i $ capture exactly the concrete and
abstract values when termination is assumed.
\begin{align*}
\omega^1(\distfunzero)(\delta) & = \dcond{\delta}{\neg \bexp} \\
 & = \Delta_1
\end{align*}
\begin{align*}
\omega^2(\distfunzero)(\delta)
& = \dcond{\paren{\pevalp{S'}{\paren{\dcond{\delta}{\bexp}}}}}{\neg \bexp}
+ \dcond{\delta}{\neg \bexp} \\
& = \dcond{\delta_2}{\neg \bexp} + \dcond{\delta_1}{\neg \bexp} \\
& = \Delta_2
\end{align*}
\begin{align*}
\omega^i(\distfunzero)(\delta)
& = \omega^{i-1}(\distfunzero)(\pevalp{S'}{\dcond{\delta}{\bexp}})
+ \dcond{\delta}{\neg \bexp} \\
& = \Delta_{i-1} + \dcond{\delta}{\neg \bexp} \\
& = \Delta_i
\end{align*}

Likewise $ \Omega^i(\ppfunzero)(\pp{}) = \Phi_i $.

\begin{definition} Terminating semantics of while loops are as follows.
$$ \pevalp{\swhile{\bexp}{S_1}}{\delta} = \Delta_n $$
Where $ n $ is the least index with $ \delta_n = \distzero $. Likewise for the abstract case.
$$ \abspevalp{\swhile{\bexp}{S_1}}{\pp{}} = \Phi_n $$
Where $ n $ is the least index with $ \iszero{\pp{n}} $.
\end{definition}

\begin{lemma} \label{lem:pp:while:infinite} If $ \pevalp{\swhile{\bexp}{S_1}} \delta $ is
terminating, then $ \Delta_n = \paren{\lfp(\omega)}(\delta) $, noting
that $ \lfp(\omega) $ is the original semantics of a while loop.
\end{lemma}
\begin{proof} As noted in \cite{monniaux00prob}, the evaluation of
a while loop on a distribution is equal to an infinite sum:

$$ \pevalp{S}{\delta} = \sum_{i = 1}^{\infty} \dcond{\delta_i}{\neg \bexp} $$

By the termination assumption we have an $ n $ with $ \delta_n
= \distzero $. Now, since $ \delta_{i+1}
= \pevalp{S'}{\dcond{\delta_i}{\bexp}} $ hence the mass of
$ \delta_{i+1} $ cannot exceed the mass of $ \delta_i $, it is the
case that if $ \delta_n = \distzero $, then $ \delta_i = \distzero $
for every $ i \geq n $. Thus the infinite sum above can be shortened.

\begin{align*}
\pevalp{S}{\delta} & = \sum_{i =
1}^{\infty} \dcond{\delta_i}{\neg \bexp} \\
 & = \sum_{i=1}^{n} \dcond{\delta_i}{\neg \bexp} + \distzero \\
 & = \Delta_n
\end{align*}

\end{proof}

\begin{remark}[Nature of Termination] \label{lem:pp:while:terminate} If
$ \abspevalp{S}{\pp{}} $ terminates, then so must the evaluation of
all of its components as defined by the semantics. This is immediate
from the definition of termination.
\end{remark}

%The design of the termination definition lets us do away with
%potential issues stemming from short-circuiting the interpretation in
%programs that contain irrelevant infinite loops as the example below.
%$$ \spif{1}{skip}{\swhile{1 < 2}{\sassign{x}{x+1}}} $$
%
%Such a program is deemed non-terminating. The sound treatment of while loops is demonstrated within the main
%soundness theorem in the next section.
%\sbmcomment{This doesn't make sense to me.  It seems like the program above would be terminating, since the distribution reaching the while loop is already zero (it gets scaled by 0 due to the pif).}

\subsection{Soundness of Abstraction}
\begin{reptheorem}{thm:pp:soundness}
For all $\pp{}, \delta$, if $ \delta \in \ppconc{\pp{}} $ and
$ \abspevalp{S}{\pp{}} $ terminates, then $ \pevalp{S}{\delta} $ terminates and $ \pevalp{S}{\delta} \in \ppconc{\abspevalp{S}{\pp{}}} $.
\end{reptheorem}

\begin{proof} Let us show this by structural induction on $ S $. As base cases we have the following.

\begin{itemize}
\item{} $ S = \sskip $. In this case we have $ \pevalp{S}{\delta}
= \delta $ and $ \abspevalp{S}{\pp{}} = \pp{} $. Termination is not an
issue and the claim holds by assumption.
\item{} $ S = \sassign{x}{E} $. Here non-termination is also not a
possibility given non-recursive definition of assignment. Also, by
 \lref{lem:pp:assign} (Soundness of Assignment) we have
$ \pevalp{S}{\delta} \in \ppconc{\abspevalp{S}{\pp{}}} $.
\item{} $ S = \suniform{x}{n_1}{n_2} $. Again, there is no termination
 issues and the claim follows from \lref{lem:pp:uniform} (Soundness of Uniform).
\end{itemize} 

Let us thus assume the claim for sub-statements of $ S $ and show it
for $ S $ itself. Note that the inductive assumption is general for
all $ \delta $, $ \pp{} $ with $ \delta \in \ppconc{\pp{}}
$. $ S $
has several cases.

\begin{itemize}
\item{} $ S = \sseq{S_1}{S_2} $. By the termination remark, we know
$ \abspevalp{S_1}{\pp{}} $ terminates and thus by induction
$ \pevalp{S_1}{\delta} $ terminates and is in
$ \ppconc{\abspevalp{S_1}{\pp{}}} $. We then apply induction once more
with $S_2$ to find
that $ \pevalp{S_2}{\paren{\pevalp{S_1}{\delta}}} = \pevalp{S}{\delta}
$ terminates and is in
$ \ppconc{\abspevalp{S_2}{\paren{\abspevalp{S_1}{\pp{}}}}}
= \ppconc{\abspevalp{S}{\pp{}}} $.
\item{} $ S = \sif{\bexp}{S_1}{S_2} $. By the termination remark, we
know that $ \abspevalp{S_1}{\paren{\absdcond{\pp{}}{\bexp}}} $ and
$ \abspevalp{S_2}{\paren{\absdcond{\pp{}}{\neg \bexp}}} $
terminate. By \lref{lem:pp:cond} (Soundness of Conditional) we have
$ \dcond{\delta}{\bexp} \in \ppconc{\absdcond{\pp{}}{\bexp}} $ and
$ \dcond{\delta}{\neg \bexp} \in \ppconc{\absdcond{\pp{}}{\neg \bexp}}
$. We thus apply induction to both sub-statements to conclude that
$ \pevalp{S_1}{\paren{\dcond{\delta}{\bexp}}} $ and
$ \pevalp{S_2}{\paren{\dcond{\delta}{\neg \bexp}}} $ both terminate
and are in $ \ppconc{\abspevalp{S_1}{\paren{\absdcond{\pp{}}{\bexp}}}}
$ and $ \ppconc{\abspevalp{S_2}{\paren{\absdcond{\pp{}}{\neg \bexp}}}}
$ respectively. Finally we apply \lref{lem:pp:plus} (Soundness of
Plus) to conclude $ \pevalp{S}{\delta}
= \pevalp{S_1}{\paren{\dcond{\delta}{\bexp}}}
+ \pevalp{S_2}{\paren{\dcond{\delta}{\neg \bexp}}} \in \ppconc{\abspevalp{S_1}{\paren{\absdcond{\pp{}}{\bexp}}}
+ \abspevalp{S_2}{\paren{\absdcond{\pp{}}{\neg \bexp}}}}
= \ppconc{\abspevalp{S}{\pp{}}} $.

\item{} $ S = \spif{p}{S_1}{S_2} $. This case is identical to the
previous except we use \lref{lem:pp:scalar-prod} (Soundness of Scalar
Product) in place of \lref{lem:pp:cond} (Soundness of Conditional). 

\item{} $ S = \swhile{\bexp}{S_1} $.
\end{itemize}
For this last case we must first show a claim. For every $ \delta' $, $
\pp{}' $ with $ \delta' \in \ppconc{\pp{}'} $, and every $ i $ we have
the following.
\begin{align}
\label{lem:pp:while:claim1a} \delta'_i \in \ppconc{\pp{i}'} \\
\label{lem:pp:while:claim1b} \Delta'_i \in \ppconc{\Phi'_i}
\end{align}
Let us show this claim by induction on $ i $. As the base case we have
$ \delta'_1 = \delta' $ and $ \Delta'_1 = \dcond{\delta'_1}{\neg \bexp}
= \dcond{\delta'}{\neg \bexp} $. Also $ \pp{1}' = \pp{}' $ and $ \Phi'_1
= \absdcond{\pp{1}'}{\neg \bexp} = \absdcond{\pp{}'}{\neg \bexp} $. By
assumption we had $ \delta' \in \ppconc{\pp{}'} $ so the first part of
our claim holds trivially. For the other we apply \lref{lem:pp:cond}
(Soundness of Conditional) to conclude $ \Delta'_1 \in \ppconc{\Phi'_1}
$.

Let us assume the claim holds for all $ i < n $ and show that it holds
for $ n $. 

We have, by definition, $ \delta'_n
= \pevalp{S_1}{\paren{\dcond{\delta'_{n-1}}{\bexp}}} $ and $ \pp{n}'
= \abspevalp{S_1}{\paren{\absdcond{\pp{n-1}'}{\bexp}}} $. By the
(inner) induction assumption, we have
$ \delta'_{n-1} \in \ppconc{\pp{n-1}'} $ so by \lref{lem:pp:cond} we
have
$ \dcond{\delta'_{n-1}}{\bexp} \in \ppconc{\absdcond{\pp{n-1}'}{\bexp}}
$. Since $ \abspevalp{S}{\pp{}} $ terminates, then so must
$ \abspevalp{S_1}{\absdcond{\pp{n-1}'}{\bexp}} $ by the termination
remark. Thus, by the (outer) induction hypothesis, we know that
$ \pevalp{S_1}{\paren{\dcond{\delta'_{n-1}}{\bexp}}}
= \delta'_n \in \ppconc{\abspevalp{S_1}{\paren{\absdcond{\pp{n-1}'}{\bexp}}}}
= \ppconc{\pp{n}'} $.

For the second part of the claim, we have $ \Delta'_n = \Delta'_{n-1}
+ \dcond{\delta'_n}{\neg \bexp} $ and $ \Phi'_n = \Phi'_{n-1}
+ \absdcond{\pp{n}'}{\neg \bexp} $. By (inner) induction we know
$ \Delta'_{n-1} \in \ppconc{\Phi'_{n-1}} $. By the first part of the claim above
we know $ \delta'_n \in \ppconc{\pp{n}'} $ so by \lref{lem:pp:cond}
(Soundness of Conditional) we have
$ \dcond{\delta'_n}{\neg \bexp} \in \ppconc{\absdcond{\pp{n}'}{\neg \bexp}}
$. Now we apply \lref{lem:pp:plus} (Soundness of Plus) to conclude
$ \Delta'_n = \Delta'_{n-1}
+ \dcond{\delta'_n}{\neg \bexp} \in \ppconc{\Phi'_{n-1}
+ \absdcond{\pp{n}'}{\neg \bexp}} = \ppconc{\Phi'_{n}} $, finishing the
claim.

Now, since $ \abspevalp{S}{\pp{}'} $ terminates, it must be that
$ \abspevalp{S}{\pp{}'} = \Phi'_n $ for some $ n $, according to the terminating
semantics. Furthermore we have the following, also by definition of termination.
\begin{align}
\label{lem:pp:while:claim2a} \iszero{\absdcond{\pp{n}'}{\neg \bexp}}
\end{align}

This is the case since $ \iszero{\pp{n}'} $ and the fact that the
conditioning operation preserves $ \iszero{\cdot} $.

%That is, no more change to $ \Phi_n $ occurs beyond $ n $. Note again
%that this $ \pp{}' $ is general \pxm{more generality issues, continues
%below}. Since $ \Omega^{n}(\ppfunzero) $ is a fixed point of $ \Omega
%$, we have $ \Omega^{n+1}(\ppfunzero) = \Omega^{n}(\ppfunzero) $ so
%$ \Phi'_{n+1} = \Omega^{n+1}(\ppfunzero)(\pp{}')
%= \Omega^{n}(\ppfunzero)(\pp{}') = \Phi'_{n} $. But $ \Phi'_{n+1}
%= \Phi'_{n} + \absdcond{\pp{n+1}'}{\neg \bexp} $, so $ \Phi'_{n}
%+ \absdcond{\pp{n+1}'}{\neg \bexp} = \Phi'_{n} $. Also, since
%$ \Delta'_{n} \in \ppconc{\Phi'_n} $, $ \Phi'_n $ is consistent. Likewise since
%$ \delta'_{n+1} \in \ppconc{\pp{n+1}'} $, and \lref{lem:pp:cond}
%(Soundness of Conditional), it must be that
%$ \absdcond{\pp{n+1}'}{\neg \bexp} $ is consistent. Therefore,
%by \lref{lem:pp:while:order:pluszero},
%$ \iszero{\absdcond{\pp{n+1}'}{\neg \bexp}} $.

%By \lref{lem:pp:while:order:plusgeq} we have
%$ \Phi_{n} \ppleq \Phi_{n+1} = \Phi_{n}
%+ \absdcond{\pp{n+1}}{\neg \bexp} $ but since $ \Phi_n
%= \abspevalp{S}{\pp{}} $ and $ \Phi_i \leq \abspevalp{S}{\pp{}} $ for
%every $ i $, so $ n + 1 $ in particular, it must be that $ \Phi_{n+1}
%= \Phi_{n} + \absdcond{\pp{n+1}}{\neg \bexp} \ppleq \Phi_{n} $. Thus
%by \lref{lem:pp:while:order:pluszero} it must be that
%$ \absdcond{\pp{n+1}}{\neg \bexp} = \ppzero $.

%Now, by \cnumref{lem:pp:while:claim1a}, we
%have $ \omega^{n+1}(\distfunzero)(\delta) \in \ppconc{\Omega^{n+1}(\ppfunzero)(\pp{})}
%$. Using \cnumref{lem:pp:while:claim2a} we conclude that
%$ \dcond{\delta_{n+1}}{\neg \bexp} = \distzero $.

%Let us thus show that $ \omega^{n+1}(\distfunzero)
%= \omega^{n}(\distfunzero) $. Let $ \delta' $ be any distrubtion. Let
%$ \pp{}' $ be any probabilistic polyhedron with
%$ \delta' \in \ppconc{\pp{}'} $. Thus
%$ \delta'_{n+1} \in \ppconc{\pp{n+1}'} $
%by \cnumref{lem:pp:while:claim1a}. But $ \iszero{\pp{n+1}'} $
%by \cnumref{lem:pp:while:claim2a}. Therefore $ \delta'_{n+1}
%= \distzero $ and thus $ \dcond{\delta'_{n+1}}{\neg \bexp} = \distzero
%$.  Now, $ \omega^{n+1}(\distfunzero)(\delta') = \Delta'_{n}
%+ \dcond{\delta'_{n+1}}{\neg \bexp}
%= \omega^{n}(\distfunzero)(\delta') + \distzero
%= \omega^{n}(\distfunzero)(\delta') $. Therefore we conclude
%$ \omega^{n+1}(\distfunzero) = \omega^{n}(\distfunzero) $, that is,
%$ \omega^{n}(\distfunzero) $ is a fixed point of $ \omega $ as needed.
%
%Finally let $ m \leq n $ with
%$ \omega^{m}(\distfunzero) $ a fixed point of $ \omega $. Thus
%$ \omega^{m+1}(\distfunzero) = \omega^{m}(\distfunzero) $ and further
%$ \omega^{m+i}(\distfunzero) = \omega^{m}(\distfunzero) $. Thus it
%must be that $ \omega^{n}(\distfunzero) = \omega^{m}(\distfunzero)
%$. As noted in \cite{kozen81semantics}, if $ \omega^{n}(\distfunzero)
%$ is a fixed point of $ \omega $, it is the least fixed point,
%therefore $ \pevalp{S}{} = \omega^n $ as required. \pxm{is that enough
%of a citation?.} So $ \pevalp{S}{\delta} = \Delta_n $ and
%given \cnumref{lem:pp:while:claim1b}, we are done.

Therefore by \cnumref{lem:pp:while:claim1a} we can conclude that
$ \delta_{n} = \distzero $ as $ \pconc{\pp{n}} = \set{\distzero}
$. Therefore $ \pevalp{S}{\delta} $ terminates and
by \lref{lem:pp:while:infinite} we have $ \pevalp{S}{\delta}
= \Delta_n $. The issue of whether $ n $ is the least index with
$ \delta_n = \distzero $ is irrelevant as if it were not, the larger
sum includes only additional $ \distzero $
terms. By \cnumref{lem:pp:while:claim1b}, we have
$ \Delta_n \in \ppconc{\Phi_n} $ and we are done as $ \Phi_n
= \abspevalp{S}{\pp{}} $ according to the terminating semantics.
\end{proof}

\subsection{Normalization}
\begin{replemma}{lem:pp:norm}
If $\delta_1 \in \ppconc{\pp{1}}$ then $\normal{\delta_1} \in \ppconc{\normal{\pp{1}}}$.
\end{replemma}

\begin{proof} By assumption we have the following.
\begin{align*}
%\label{thm:pp:scalar:given1} 
& \support{\delta_1} \subseteq \pconc{\poly_1} \\
%\label{thm:pp:scalar:given2} 
& \smin{1} \leq \setsize{\support{\delta_1}} \leq \smax{1} \\
%\label{thm:pp:scalar:given3} 
& \mmin{1} \leq \pmass{\delta_1} \leq \mmax{1} \\
%\label{thm:pp:scalar:given4} 
& \forall \sigma \in \support{\delta_1} \;
. \; \pmin{1} \leq \delta_1(\sigma) \leq \pmax{1}
\end{align*}

If $ \pmass{\delta_1} = 0 $ then $ \normal{\delta_1} $ is
undefined. Since $ \mmin{1} \leq \pmass{\delta_1} $, it must be that
$ \mmin{1} = 0 $ as well, and thus $ \normal{\pp{1}} $ is likewise undefined.

Let us now assume $ \pmass{\delta_1} > 0 $. Let $ \delta_2 = \normal{\delta_1} $ and $ \pp{2} = \normal{\pp{1}}
$. We have two sub-cases,
either $ \mmin{1} = 0 $ or $ \mmin{1} > 0 $. In the first sub case,
$ \pp{2} $ is defined as follows.

\[
\begin{array}{lcl@{\hspace{0.35cm}}|@{\hspace{0.35cm}}lcl}
\pmin{2} &=& \pmin{1} / \mmax{1} &
\smin{2} &=& \smin{1} \\
\pmax{2} &=& 1 &
\smax{2} &=& \smax{1} \\
\mmin{2} &=& \mmax{2} = 1 & \getpoly{2} &=& \getpoly{1} \\
\end{array}
\]

Since $ \support{\delta_2} = \support{\delta_1} $, it must be that
$ \support{\delta_2} \subseteq \pconc{\poly_2} $ as $ \poly_2
= \poly_1 $. Likewise, the number of support point is is unchanged in
both the concrete operation and the abstract one, hence the number of
support points condition for soundness are satisfied as well. Also,
the probability per point in any distribution does not exceed $ 1 $
hence the $ \pmax{2} $ condition is satisfied. As for $ \pmin{2} $,
note that if $ \sigma \in \support{\delta_2} = \support{\delta_1} $,
we have $ \delta_2(\sigma) = \delta_1(\sigma)
/ \pmass{\delta_1} \geq \pmin{1} / \pmass{\delta_1} \geq \pmin{1}
/ \mmax{1} $, by assumption. Finally, $ \pmass{\delta_2} = 1 $ hence
the $ \mmin{2} $ and $ \mmax{2} $ conditions are satisfied.

In the other case, we have $ \pmin{1} > 0 $. Here $ \pp{2} $ is
defined as follows.
\[
\begin{array}{lcl@{\hspace{0.35cm}}|@{\hspace{0.35cm}}lcl}
\pmin{2} &=& \pmin{1} / \mmax{1} &
\smin{2} &=& \smin{1}\\
\pmax{2} &=& \pmax{1} / \mmin{1} &
\smax{2} &=&\smax{1} \\
\mmin{2} &=& \mmax{2} = 1 & \getpoly{2} &=& \getpoly{1} \\
\end{array}
\]

The support, support points, total mass, and $ \pmin{2} $ conditions are satisfied for the same
reason as in the previous case. For $ \pmax{2} $, let
$ \sigma \in \support{\delta_2} = \support{\delta_1} $ and we have
the following.
\begin{align*}
\delta_2(\sigma)
 & = \delta_1(\sigma) / \pmass{\delta_1} \\
 & \leq \pmax{1} / \pmass{\delta_1} \\
 & \leq \pmax{1} / \mmin{1}
\end{align*}
\end{proof}

\subsection{Security}

Before we prove the security theorem, let us show that the definition
of abstract conditioning on a state is sound.

\begin{lemma} \label{lem:pp:scond}
If $ \delta \in \ppconc{\pp{}} $ and $ \sigma_V \in \states_{V} $
with $ V \subseteq \fv{\delta} $ then
$ \dcond{\delta}{\sigma_V} \in \ppconc{\absdcond{\pp{}}{\sigma_V}} $
\end{lemma}

\begin{proof} Recall the definition of $ \absdcond{\pp{}}{\sigma_V} $.

$$ \absdcond{\pp{}}{\sigma_V} = \absdcond{\pp{}}{\bexp} $$

With $ \bexp = \bigwedge_{x \in V} \paren{x = \sigma_V(x)} $. Let us
show that $ \dcond{\delta}{\sigma_V} = \dcond{\delta}{\bexp} $, the
rest will follow from \lref{lem:pp:cond}.

The definition of $ \dcond{\delta}{\sigma_V} $ is as follows.
$$ \dcond{\delta}{\sigma}
= \lambda \sigma \lsep \aif \project{\sigma}{V} = \sigma_V
\athen \delta(\sigma) \aelse 0 $$

Meanwhile, $ \dcond{\delta}{\bexp} $ is defined as follows.
$$ \dcond{\delta}{\bexp}
= \lambda \sigma \lsep \aif \eval{\bexp}{\sigma}
= \atrue \athen \delta(\sigma) \aelse 0 $$

The correspondence is immediate as $ \eval{\bexp}{\sigma} = \atrue $
if and only if $ \project{\sigma}{V} = \sigma_V $ as per construction
of $ \bexp $.
\end{proof}

\begin{reptheorem}{thm:pp:secure}
  Let $\delta$ be an attacker's initial belief.  If
  $\delta \in \ppconc{\pp{}}$ and $\abspolicy{S}{\pp{}}{t}$, then $ S
  $ is threshold secure for threshold $t$ when evaluated with initial
  belief $\delta$.
\end{reptheorem}

\begin{proof}Let us consider the contrapositive. That is, assuming
  $ \delta \in \ppconc{\pp{}} $, if $ S $ is not threshold secure for
  $ t $ and initial belief $ \delta $, then it is not the case that
  $ \abspolicy{S}{\pp{}}{t} $.

Let $ \delta_2 = \eval{S}{\delta} $ and $ \delta_3
= \project{\delta_2}{L} $. Since $ S $ is not secure, we have
$ \sigma_L \in \support{\delta_3} $ and $ \sigma'_H \in \states_H $
with $
(\normal{\project{\paren{\dcond{\delta_2}{\sigma_L}}}{H}})(\sigma'_H)
> t $. This implies that
$ \project{\paren{\dcond{\delta_2}{\sigma_L}}}{H} \neq \distzero $ and
therefore $ \dcond{\delta_2}{\sigma_L} \neq \distzero $ as projection
preserves mass.

If $ \abspevalp{S}{\pp{}} $ is not terminating, then we are done as
termination is a condition for $ \abspolicy{S}{\pp{}}{t} $. So let us
assume $ \abspevalp{S}{\pp{}} $ is terminating. Let $ \pp{2}
= \abspevalp{S}{\pp{}} $. By \tref{thm:pp:soundness}, we have
$ \delta_2 \in \ppconc{\pp{2}} $. By \lref{lem:pp:scond},
$ \dcond{\delta_2}{\sigma_L} \in \ppconc{\absdcond{\pp{2}}{\sigma_L}}
$. Therefore not $ \iszero{\absdcond{\pp{}}{\sigma_L}} $ as
$ \dcond{\delta_2}{\sigma_L} \neq \distzero $. Continuing,
by \lref{lem:pp:project},
$ \project{ \paren{\dcond{\delta_2}{\sigma_L}}}{H} \in \ppconc{\project{\paren{\absdcond{\pp{2}}{\sigma_L}}}{H}}
$ and finally, by \lref{lem:pp:norm}, we have
$ \normal{\project{\paren{\dcond{\delta_2}{\sigma_L}}}{H}} \in \ppconc{\normal{\project{\paren{\absdcond{\pp{2}}{\sigma_L}}}{H}}}
$. Let $ \delta_4
= \normal{\project{\paren{\dcond{\delta_2}{\sigma_L}}}{H}} $ and
$ \pp{4} = \normal{\project{\paren{\absdcond{\pp{2}}{\sigma_L}}}{H}}
$. Since $ \sigma'_H \in \support{\delta_4} $, we have
$ \delta_4(\sigma'_H) \leq \pmax{4} $. Since $ \delta_4(\sigma'_H) > t
$, we have $ t < \pmax{4} $.

Also, let $ \pp{3} = \project{\pp{2}}{L} $. By \lref{lem:pp:project},
we have $ \delta_3 \in \ppconc{\pp{3}} $ so
$ \sigma_L \in \pconc{\poly_3} $. We already had that not
$ \iszero{\absdcond{\pp{}}{\sigma_L}} $ above. Thus $ \sigma_L $ is indeed the
witness to the failure of $ \abspolicy{S}{\pp{1}}{t} $.
\end{proof}

\vspace*{.1in}
\section{Soundness proofs for $ \ppowers $} \label{appendix:proof2}

\subsection{Useful Lemmas}

We begin with some lemmas that give properties of the concretization function for powersets of probabilistic polyhedra and addition on sets.
\begin{lemma}
\label{lem:ppp-conc:partition}
If $\pps{} = \pps{1} \cup \pps{2}$ then
$\ppsconc{\pps{}} = \ppsconc{\pps{1}} + \ppsconc{\pps{2}}$.
\end{lemma}
\begin{proof}
From the definition of $\ppsconc{\Delta}$ we have
\[\ppsconc{\pps{}} = \sum_{\pp{} \in \pps{}} \ppconc{\pp{}}\]
Applying $\pps{} = \pps{1} \cup \pps{2}$ and associativity of $+$ allows us to conclude
\[\ppsconc{\pps{}} = \sum_{\pp{1} \in \pps{1}} \ppconc{\pp{1}} + \sum_{\pp{1} \in \pps{2}} \ppconc{\pp{2}}\]
Again applying the definition of $\ppsconc{\ldots}$, we have
\[\ppsconc{\pps{}} = \ppsconc{\pps{1}} + \ppsconc{\pp{2}}\]
\end{proof}

\begin{lemma}
\label{lem:set-plus:subset}
If $D_1 \subseteq D_1'$ and $D_2 \subseteq D_2'$ then $D_1 + D_2 \subseteq D_1' + D_2'$.
\end{lemma}
\begin{proof}
According to the definition of addition for sets, we have
\[D_1 + D_2 = \{\delta_1 + \delta_2 \mid \delta_1 \in D_1 \wedge \delta_2 \in D_2\}\]
Consider some $\delta \in D_1 + D_2$.  We have $\delta = \delta_1 + \delta_2$ with
$\delta_1 \in D_1$ and $\delta_2 \in D_2$.  Since $D_1 \subseteq D_1'$, we have $\delta_1 \in D_1'$.
Similarly, since $D_2 \subseteq D_2'$, we have $\delta_2 \in D_2'$.
Since
\[D_1' + D_2' = \{\delta'_1 + \delta'_2 \mid \delta'_1 \in D_1' \wedge \delta'_2 \in D_2'\}\]
we have $\delta = \delta_1 + \delta_2 \in D_1' + D_2'$.
\end{proof}

\subsection{Bounding Operation}

\begin{replemma}{lem:ppp:bound}[Soundness of Bounding Operation]
$\ppsconc{\pps{}} \subseteq \ppsconc{
    \simplify{\pps{}}{n}} $.
\end{replemma}

\begin{proof}
According to Definition \ref{def:ppp-simp}, there are two cases for 
$\simplify{\pps{}}{n}$.  If $\setsize{\pps{}} \leq n$ then we
have $\simplify{\pps{}}{n} = \pps{}$ and thus
$\ppsconc{\pps{}} = \ppsconc{\simplify{\pps{}}{n}}$.

If $\setsize{\pps{}} > n$, we reason by induction on $\setsize{\pps{}}$.
Since $n \geq 1$, we have that $\setsize{\pps{}} \geq 2$ and thus we
 can partition $\pps{}$ into $\pps{1} \cup \{\pp{1},\pp{2}\}$.  Applying
Definition \ref{def:ppp-simp} we then have
$\simplify{\pps{}}{n} = \simplify{\pps{1} \cup \{\pp{1} + \pp{2}\}}{n}$.
The inductively-passed set has size one less than the original, allowing us to apply
the inductive hypothesis to conclude the following.
\[\ppsconc{\pps{1} \cup \set{\pp{1} + \pp{2}}} \subseteq \ppsconc{
    \simplify{\pps{1} \cup \{\pp{1} + \pp{2}\}}{n}}\]
Our conclusion will follow provided we can show
\[
\ppsconc{\pps{}} \subseteq \ppsconc{\pps{1} \cup \{\pp{1} + \pp{2}\}}
\]
\lref{lem:ppp-conc:partition} allows us to rewrite this to
\begin{equation}
\label{lem:ppp:bound:goal}
\ppsconc{\pps{}} \subseteq \ppsconc{\pps{1}} + \ppsconc{\{\pp{1} + \pp{2}\}}
\end{equation}
We have $\pps{} = \pps{1} \cup \{\pp{1}, \pp{2}\}$ and thus by \lref{lem:ppp-conc:partition}
we have
\[\ppsconc{\pps{}} = \ppsconc{\pps{1}} + \ppsconc{\{\pp{1}, \pp{2}\}}\]
By \lref{lem:set-plus:subset}, we will have (\ref{lem:ppp:bound:goal}) provided we can show
\[\ppsconc{\pps{1}} \subseteq \ppsconc{\pps{1}}\]
which is immediate, and
\[\ppsconc{\{\pp{1},\pp{2}\}} \subseteq \ppsconc{\set{\pp{1} + \pp{2}}}\]
The latter is proven by applying the definitions of
$\ppsconc{\{\pp{1},\pp{2}\}}$ and $\ppsconc{\set{\pp{1}+\pp{2}}}$, resulting in a goal of
\[\ppconc{\pp{1}} + \ppconc{\pp{2}} \subseteq \ppconc{\pp{1} + \pp{2}}\]
which follows directly from \lref{lem:pp:plus}.
\end{proof}

\subsection{Distributive Operations}
The soundness proofs for the majority of the operations on elements of $\ppowers$ are sound
for exactly the same reason: the operations distribute over $+$, allowing us to reduce soundness
for the powerset case to soundness for the case of a single probabilistic polyhedron.  We
start with the Lemma that is used to structure such a proof.

\newcommand{\fflat}{f^{\flat}}
\begin{lemma}
\label{lem:distributes-plus}
Consider $f : \ppolys \rightarrow \ppolys$, $F : \ppowers \rightarrow \ppowers$, and $\fflat : \dists \rightarrow \dists$.  Suppose the following all hold for all $\delta_i, \pp{i}$.
\begin{enumerate}
\setlength{\itemsep}{0.5em}
\item
\label{fflat-dist}
$\fflat(\delta_1 + \ldots + \delta_n) = \fflat(\delta_1) + \ldots + \fflat(\delta_n)$
\item
\label{F-dist}
$F(\{\pp{1}, \ldots, \pp{n}\}) = \{f(\pp{1}), \ldots, f(\pp{n})\}$
\item
\label{f-sound}
$\delta \in \ppconc{\pp{}} \imp \fflat(\delta) \in \ppconc{f(\pp{})}$
\end{enumerate}
Then $\delta \in \ppsconc{\pps{}}$ implies $\fflat(\delta) \in \ppsconc{F(\pps{})}$.
\end{lemma}

\begin{proof}
Suppose $\delta \in \ppsconc{\pps{}}$ and $\pps{} = \{\pp{1},\ldots,\pp{n}\}$.  We have the following by definition of $\ppsconc{\pps{}}$.
\[\ppsconc{\pps{}} = \ppconc{\pp{1}} + \ldots + \ppconc{\pp{n}}\]
Applying the definition of addition on sets, we obtain
\[\ppconc{\pp{1}} + \ldots + \ppconc{\pp{n}} = \{\delta_1 + \ldots + \delta_n \mid \delta_i \in \ppconc{\pp{i}}\}\]
Thus, we have that $\delta = \delta_1 + \ldots + \delta_n$ where $\delta_i \in \ppconc{\pp{i}}$.
By premise \ref{f-sound} we then have $\fflat(\delta_i) \in \ppconc{f(\pp{i})}$ for all $i$.

We now consider $\ppsconc{F(\pps{})}$.  By premise \ref{F-dist} we
have that this is $\ppsconc{\{f(\pp{1}), \ldots, f(\pp{n})\}}$.
Applying the definition of $\ppsconcfun$, this is equal to
$\ppconc{f(\pp{1})} + \ldots + \ppconc{f(\pp{n})}$.

Expanding the definition of $+$ for sets, we have that
\[\ppsconc{F(\pps{})} = \{\delta_1 + \ldots + \delta_n \mid \delta_i \in \ppconc{f(\pp{i})}\}\]
Since $\fflat(\delta_i) \in \ppconc{f(\pp{i})}$ for all $i$ we have $\sum_i(\fflat(\delta_i)) \in \ppsconc{F(\pps{})}$ and thus, by premise \ref{fflat-dist} we have $\fflat(\sum_i\delta_i) \in \ppsconc{F(\pps{})}$ and thus $\fflat(\delta) \in \ppsconc{F(\pps{})}$ as desired.
\end{proof}

\begin{lemma}[Soundness of Forget]
\label{lem:ppp:forget}
If $\delta \in \ppsconc{\pps{}}$ then $\forget{y}{\delta} \in \ppsconc{\forget{y}{\pps{}}}$.
\end{lemma}
\begin{proof}
We will apply \lref{lem:distributes-plus} with $\fflat = \lambda \delta.\ \project{\delta}{(\fv{\delta} - \{y\})}$, $f = \lambda \pp{}.\ \forget{y}{\pp{}}$, and $F = \lambda \pps{}.\ \forget{y}{\pps{}}$.
\lref{lem:pp:forget} gives us premise \ref{f-sound}.  The definition of $\forget{y}{\pps{}}$ satisfies premise \ref{F-dist}.  Let $V = \fv{\delta} - \{y\}$.  It remains to show premise \ref{fflat-dist}, which states
\[\project{(\delta_1 + \ldots + \delta_n)}{V} = \project{\delta_1}{V} + \ldots + \project{\delta_n}{V}\]
We show this for the binary case, from which the $n$-ary version above follows.
\[\project{(\delta_1 + \delta_2)}{V} = \project{\delta_1}{V} + \project{\delta_2}{V}\]
Expanding the definition of projection, we then obtain the following goal.
\begin{multline*}
\lambda \sigma_V \in \states_V \lsep \sum_{\sigma' \mid (\project{\sigma'}{V} = \sigma_V)} (\delta_1+\delta_2)(\sigma') =\\
\lambda \sigma_V \in \states_V \lsep \sum_{\sigma' \mid (\project{\sigma'}{V} = \sigma_V)} \delta_1(\sigma')\\
+ \lambda \sigma_V \in \states_V \lsep \sum_{\sigma' \mid (\project{\sigma'}{V} = \sigma_V)} \delta_2(\sigma')
\end{multline*}
We can now apply the definition of $+$ for distributions to the right-hand side to obtain a goal of
\begin{multline*}
\lambda \sigma_V \in \states_V \lsep \sum_{\sigma' \mid (\project{\sigma'}{V} = \sigma_V)} (\delta_1+\delta_2)(\sigma') =\\
\lambda \sigma_V \in \states_V \lsep \biggl(\sum_{\sigma' \mid (\project{\sigma'}{V} = \sigma_V)} \delta_1(\sigma')
+ \sum_{\sigma' \mid (\project{\sigma'}{V} = \sigma_V)} \delta_2(\sigma')\biggr)
\end{multline*}
These functions are equal if they give equal results for all inputs.  Thus, we must show the following for all $\sigma_V$.
\begin{multline*}
\sum_{\sigma' \mid (\project{\sigma'}{V} = \sigma_V)} (\delta_1+\delta_2)(\sigma') =\\
\biggl(\sum_{\sigma' \mid (\project{\sigma'}{V} = \sigma_V)} \delta_1(\sigma')
+ \sum_{\sigma' \mid (\project{\sigma'}{V} = \sigma_V)} \delta_2(\sigma')\biggr)
\end{multline*}
Finally, applying the definition of $+$ for distributions to the left-hand side of the equality yields
\begin{multline*}
\sum_{\sigma' \mid (\project{\sigma'}{V} = \sigma_V)} \Bigl(\delta_1(\sigma')+\delta_2(\sigma')\Bigr) =\\
\biggl(\sum_{\sigma' \mid (\project{\sigma'}{V} = \sigma_V)} \delta_1(\sigma')
+ \sum_{\sigma' \mid (\project{\sigma'}{V} = \sigma_V)} \delta_2(\sigma')\biggr)
\end{multline*}
This follows by associativity and commutativity of $+$.
\end{proof}

\begin{lemma}[Soundness of Projection]
\label{lem:ppp:project}
If $\delta \in \ppsconc{\pps{}}$ and $ V \subseteq \fv{\delta} $ then
$\project{\delta}{V} \in \ppsconc{\project{\pps{}}{V}} $.
\end{lemma}
\begin{proof} Inductive application of \lref{lem:ppp:forget} (Soundness
of Forget) as was the case in the base domain.
\end{proof}

\begin{lemma}[Soundness of Assignment] \label{lem:ppp:assign}
If $ \delta \in \ppsconc{\pps{}} $ then $
\delta\bparen{x \ra E} \in \ppsconc{\pps{} \bparen{x \ra E}} $.
\end{lemma}
\begin{proof}
  As in \lref{lem:ppp:forget}, we apply \lref{lem:distributes-plus}.
  We have premises \ref{f-sound} (by \lref{lem:pp:assign}) and
  \ref{F-dist} (by definition) and must show premise \ref{fflat-dist}.
This means showing that
\[(\delta_1+\delta_2)\bparen{x \ra E} = \delta_1\bparen{x \ra E} + \delta_2\bparen{x \ra E}\]
Expanding the definition of assignment, we must show that the following
$$ \lambda \sigma \lsep \sum_{\tau \; | \; \tau
  \bparen{x \ra \eeval{\aexp}{\tau}} = \sigma} (\delta_1+\delta_2) (\tau) $$
is equal to
$$ \Biggl(\lambda \sigma \lsep \sum_{\tau \; | \; \tau
  \bparen{x \ra \eeval{\aexp}{\tau}} = \sigma} \delta_1 (\tau)\Biggr) +
\Biggl(\lambda \sigma \lsep  \sum_{\tau \; | \; \tau
 \bparen{x \ra \eeval{\aexp}{\tau}} = \sigma} \delta_1 (\tau)\Biggr)$$
Again applying the definition of $+$ for distributions and using extensional equality for functions yields the following goal, which follows by associativity and commutativity of $+$.
\begin{multline*}
\forall \sigma.\ \Biggl(\sum_{\tau \; | \; \tau
  \bparen{x \ra \eeval{\aexp}{\tau}} = \sigma} \Bigl(\delta_1(\tau)+\delta_2(\tau)\Bigr) = \\
\sum_{\tau \; | \; \tau
  \bparen{x \ra \eeval{\aexp}{\tau}} = \sigma} \delta_1 (\tau) +
\sum_{\tau \; | \; \tau
 \bparen{x \ra \eeval{\aexp}{\tau}} = \sigma} \delta_1 (\tau)\biggr)
\end{multline*}
\end{proof}

\begin{lemma}[Soundness of Scalar Product] \label{lem:ppp:scalar-prod}
If $ \delta \in \ppsconc{\pps{}} $ then $
  p \cdot \delta \in \ppsconc{p \cdot \pps{}} $.
\end{lemma}
\begin{proof}
This proof follows the same format as the others in this section.  We apply \lref{lem:pp:assign}
with the definition of scalar product for powersets and \lref{lem:pp:scalar-prod}.  We must show
\[p \cdot (\delta_1 + \delta_2) = p \cdot \delta_1 + p \cdot \delta_2\]
Expanding according to the definition of scalar product and $+$ for distributions, we obtain the following as a goal.
\[\lambda \sigma \lsep p \cdot (\delta_1(\sigma) + \delta_2(\sigma)) = \lambda \sigma \lsep p \cdot \delta_1(\sigma) + p \cdot \delta_2(\sigma)\]
The result follows by distributivity of $\cdot$ over $+$.
\end{proof}

\begin{lemma}[Soundness of Conditioning] \label{lem:ppp:cond}
If $ \delta \in \ppsconc{\pps{}} $ then $
  \dcond{\delta}{B} \in \ppsconc{\absdcond{\pps{}}{B}} $.
\end{lemma}
\begin{proof}
Again we apply \lref{lem:pp:assign}, this time using \lref{lem:pp:cond}
to satisfy premise \ref{f-sound}.  We let
$\fflat = \lambda \delta.\ \dcond{\delta}{B}$,
$f = \lambda \pp{}.\ \absdcond{\pp{}}{B}$, and
$F = \lambda \pps{}.\ \absdcond{\pps{}}{B}$.
We must show
\[\dcond{(\delta_1 + \delta_2)}{B} = \dcond{\delta_1}{B} + \dcond{\delta_2}{B}\]
Applying the definition of conditioning and addition for distributions, we have
to show the following for all $\sigma$.
\begin{multline*}
\aif \eeval{\bexp}{\sigma} \athen
(\delta_1+\delta_2)(\sigma) \aelse 0 =\\
\bigl(\aif \eeval{\bexp}{\sigma} \athen
\delta_1(\sigma) \aelse 0\bigr) + {}\\
\bigl(\aif \eeval{\bexp}{\sigma} \athen
\delta_2(\sigma) \aelse 0\bigr)
\end{multline*}
We proceed via case analysis.  If $\eeval{\bexp}{\sigma} = \afalse$ then we
have $0 = 0 + 0$, which is a tautology.  If $\eeval{\bexp}{\sigma} = \atrue$,
we have to show
\[(\delta_1+\delta_2)(\sigma) = \delta_1(\sigma) + \delta_2(\sigma)\]
which follows directly from the definition of $+$ on distributions.
\end{proof}

\subsection{Other Powerset Lemmas}

We now show the lemmas for operations in the powerset domain that do not immediately
follow from distributivity over plus of the operations in the base domain.

\begin{lemma}[Soundness of Product] \label{lem:ppp:prod}
  If $ \delta \in \ppsconc{\pps{}} $ and $ \delta' \in
  \ppsconc{\pp{}'} $ and $\fv{\pps{}} \cap \fv{\pp{}'} = \emptyset$ then $
  \delta \times \delta' \in \ppsconc{\pps{} \times \pp{}'} $.
\end{lemma}
\begin{proof}
Let $\pps{} = \{\pp{1},\ldots,\pp{n}\}$.  We first expand definitions
in our goal, obtaining
\[\delta \times \delta' \in \ppconc{\pp{1} \times \pp{}'} + \ldots + \ppconc{\pp{n} \times \pp{}'}\]
Applying the definition of addition for sets, we obtain a goal of
\[\delta \times \delta' \in \Bigl\{\sum_i \delta_i \mid \delta_i \in \ppconc{\pp{i} \times \pp{}'} \Bigr\}\]
This holds provided we can find $\delta_i \in \ppconc{\pp{i} \times \pp{}'}$ such that $\delta \times \delta' = \sum_i \delta_i$.
We have from $\delta \in \ppsconc{\pps{}}$ that $\delta = \sum_j \delta_j$ for some $\delta_j \in \ppconc{\pp{j}}$.  We then have from \lref{lem:pp:product} and $\delta' \in \ppsconc{\pp{}'}$ and
$\fv{\pps{}} \cap \fv{\pp{}'} = \emptyset$ that
$\delta_j \times \delta' \in \ppconc{\pp{j} \times \pp{}'}$ for all $j$.
We now show that the $\delta_i$ we were searching for are these $\delta_j \times \delta'$.
To do so, we must show that $\delta \times \delta' = \sum_j(\delta_j \times \delta')$.
We have $\delta = \sum_j \delta_j$ and thus the result follows by distributivity of $\times$
over $+$, which we show now.

\paragraph{Goal: $\times$ distributes over $+$}
We want to show the following when $\dom{\delta_1} = \dom{\delta_2}$ and $\dom{\delta_1} \cap \dom{\delta'} = \emptyset$.
\[(\delta_1 + \delta_2) \times \delta' = \delta_1 \times \delta' + \delta_2 \times \delta'\]
Expanding the definition of $+$ and of $\times$, we obtain
\begin{multline*}
\lambda(\sigma, \sigma') \lsep \bigl(\delta_1(\sigma) + \delta_2(\sigma)\bigr) \cdot \delta'(\sigma') =\\
\lambda(\sigma, \sigma') \lsep \bigl(\delta_1(\sigma) \cdot \delta'(\sigma') +
\delta_2(\sigma) \cdot \delta'(\sigma')\bigr)
\end{multline*}
This holds due to distributivity of $\cdot$ over $+$.
\end{proof}

\begin{lemma}[Soundness of Addition]
\label{lem:ppp:plus}
 If $ \delta_1 \in \ppsconc{\pps{1}} $ and $ \delta_2
  \in \ppsconc{\pps{2}} $ then $ \delta_1 + \delta_2 \in
  \ppsconc{\pps{1} + \pps{2}} $.
\end{lemma}
\begin{proof} First let us take care of the special cases that occur
  when $ \iszero{\pps{1}} $ or $ \iszero{\pps{2}} $. Without the loss
  of generality let us say $ \iszero{\pps{2}} $. The sum is defined to
  be identical to $ \pps{1} $. Since $ \iszero{\pps{2}} $, it must be
  that $ \ppsconc{\pps{2}} $ contains only the zero distribution
  $ \distzero $, therefore $ \delta_2 = \distzero $. Therefore
  $ \delta_1 + \delta_2 = \delta_1 $ and by assumption,
  $ \delta_1 \in \ppsconc{\pps{1}} = \ppsconc{\pps{1} + \pps{2}} $.

In the case where $\pps{1}$ and $\pps{2}$ are both non-zero, we have
$\pps{1} + \pps{2} = \simplify{\pps{1} \cup \pps{2}}{n}$.
Suppose $\delta_1 \in \ppsconc{\pps{1}}$ and $\delta_2 \in \ppsconc{\pps{2}}$.
By \lref{lem:ppp-conc:partition} we have
$\ppsconc{\pps{1}} + \ppsconc{\pps{2}} = \ppsconc{\pps{1} \cup \pps{2}}$.
The set $\ppsconc{\pps{1}} + \ppsconc{\pps{2}}$ is
$\{\delta_1' + \delta_2' \mid \delta_1' \in \ppsconc{\pps{1}} \wedge \delta_2' \in \ppsconc{\pps{2}} \}$.
Our distributions $\delta_1$ and $\delta_2$ satisfy these conditions and thus
are in $\ppsconc{\pps{1} \cup \pps{2}}$.  It remains to show that
$\ppsconc{\pps{1} \cup \pps{2}} \subseteq \ppsconc{\simplify{\pps{1} \cup \pps{2}}{n}}$,
but this is exactly \lref{lem:ppp:bound}.
\end{proof}

\subsection{Main Soundness Theorem for Powerset Domain}

The main soundness theorem is an identical restatement of the main
soundness theorem in the base domain and the proof is likewise
identical, save for replacement of the relevant base domain definitions
and lemmas with the powerset ones. The only corresponding lemma which
has not yet been proven follows below.

\begin{lemma}\label{lem:ppp:while:order} Let $ \pps{i} $ be consistent
probabilistic polyhedron sets, that is, $ \ppsconc{\pps{i}} \neq \emptyset $.
\label{lem:pps:while:order:pluszero} Then, $ \pps{1} + \pps{2} \ppleq \pps{1} $ iff $ \iszero{\pps{2}} $.
\end{lemma}

\begin{proof} The proof is identical to the \lref{lem:pp:while:order},
replacing the base domain lemmas and definitions with the powerset ones.
\end{proof}

\begin{reptheorem}{thm:ppp:soundness}[Soundness of Abstraction]
For all $\delta, S, \pps{}$, if $\delta \in \ppsconc{\pps{}}$ and $
\abspevalp{S}{\pps{}} $ terminates, then $ \pevalp{S}{\delta} $ terminates and
$\pevalp{S}{\delta} \in \ppsconc{\abspevalp{S}{\pps{}}}$.
\end{reptheorem}

\begin{proof} The proof is identical to the main soundness proof for
the base domain (\tref{thm:pp:soundness}), replacing definitions and
lemmas about the base domain abstraction with the corresponding
definitions and lemmas about the powerset domain.
\end{proof}

\begin{lemma}[Soundness of Normalization]
\label{lem:ppp:norm}
If $ \delta \in \ppsconc{\pps{}} $ then $
  \normal{\delta} \in \ppsconc{\normal{\pps{}}} $.
\end{lemma}
\begin{proof}
\newcommand{\munder}{\underline{m}}
\newcommand{\mover}{\overline{m}}
Whenever $ \pmass{\delta} = 0 $, the normalization in the concrete
sense is undefined, likewise it is undefined in the abstract sense. So
let us assume $ \pmass{\delta} > 0 $.

Let $ \munder = \sum_i \mmin{i} $ and $ \mover = \sum_i \mmax{i} $. By
assumption we have $ \delta = \sum_i \delta_i $ with
\begin{align}
\label{thm:ppp:normal:given} \delta_i \in \ppconc{\pp{i}}
\end{align}

Thus we have $ \pmass{\delta} = \sum_i \pmass{\delta_i} $ and we
conclude $ \munder
= \sum_i \mmin{i} \leq \pmass{\delta} \leq \sum_i \mmax{i} = \mover $
via \cnumref{thm:ppp:normal:given}.
\begin{align} \label{thm:ppp:normal:claim1a} \munder \leq \pmass{\delta} \leq \mover \end{align}

Let $ \delta' = \normal{\delta} = \frac{1}{\pmass{\delta}} \delta
= \sum_i \frac{1}{\pmass{\delta}} \delta_i $, due to linearity of
scalar product. Let us thus show that
$ \frac{1}{\pmass{\delta}} \delta_i \in \ppconc{\normal{\pp{i}}(\munder, \mover)}
= \ppconc{\normal{\pps{}}} $ which would conclude the proof. Let us
write $ \pp{i'} = \normal{\pp{i}}(\munder, \mover) $ and $ \delta_{i'}
= \frac{1}{\pmass{\delta}} \delta_i $. We must thus show the
following.
\begin{align}
\label{thm:ppp:normal:cond1} & \support{\delta_{i'}} \subseteq \pconc{\getpoly{i'}} \\
\label{thm:ppp:normal:cond2} & \smin{i'} \leq \setsize{\support{\delta_{i'}}} \leq \smax{i'} \\
\label{thm:ppp:normal:cond3} & \mmin{i'} \leq \pmass{\delta_{i'}} \leq \mmax{i'} \\
\label{thm:ppp:normal:cond4} & \forall \sigma \in \support{\delta_{i'}} \qsep \pmin{i'} \leq \delta_{i'}(\sigma) \leq \pmax{i'}
\end{align}

Claim \cnumref{thm:ppp:normal:cond1} holds trivially as
$ \support{\delta_{i'}} = \support{\delta_i}$, $ \getpoly{i'}
= \getpoly{i} $,
and \cnumref{thm:ppp:normal:given}. Claim \cnumref{thm:ppp:normal:cond2}
holds due to the same reasoning.

For \cnumref{thm:ppp:normal:cond3}, in the case where $ \munder > 0 $, we reason, via \cnumref{thm:ppp:normal:claim1a}, as follows.

$$ \mmin{i} \leq \pmass{\delta_i} \leq \mmax{i} $$
$$ \frac{\mmin{i}}{\mover} \leq \frac{1}{\pmass{\delta}} \pmass{\delta_i} \leq \frac{\mmax{i}}{\munder} $$
$$ \frac{\mmin{i}}{\mover} \leq \pmass{\frac{1}{\pmass{\delta}} \delta_i} \leq \frac{\mmax{i}}{\munder} $$
$$ \mmin{i'}
= \frac{\mmin{i}}{\mover} \leq \pmass{\delta_{i'}} \leq \frac{\mmax{i}}{\munder}
= \mmax{i'}$$

If $ \munder = 0 $, the definition of normalization makes $ \mmax{i'}
= 1 $, which is also sound as all distributions have mass no more than
$ 1 $.

The \cnumref{thm:ppp:normal:cond4} claim is shown using reasoning
identical to the mass claim above.
\end{proof}

\begin{replemma}{lem:ppp:prob-bound}[Soundness of Simple Maximal Bound Estimate]
%$ \maxprobof{\pps{}}{\sigma} \leq
%  \maxprobof{\paren{\Sigma_i \pp{i}}}{\sigma} $
If $ \delta \in \ppsconc{\set{\pp{i}}} $ and $ \pp{} = \sum_i \pp{i}
$ then $ \max_{\sigma} \delta(\sigma) \leq \pmax{} $.
\end{replemma}
\begin{proof} By assumption we have $ \delta = \sum_i \delta_i $ with
$ \delta_i \in \ppconc{\pp{i}} $ thus by \lref{lem:pp:plus} (Soundness
of Plus), we have $ \delta \in \ppconc{\sum_i \pp{i}} = \ppconc{\pp{}}
$, thus for every $ \sigma \in \support{\delta} $, $ \delta(\sigma) \leq
\pmax{} $, hence $ \max_{\sigma} \delta(\sigma) \leq \pmax{} $.
\end{proof}

\pxm{the below has been reworked to use the ``poly partitioning'' method}

The above lemma shows soundness of the very simple method of
estimating the maximum probability but in the implementation we use
the method based on poly partitioning and the following lemma.

\begin{replemma}{lem:ppp:pp} $ \ppsmax{\pps{}} \defeq \max_{\sigma \in
R}\ppsprob{\pps{}}{\sigma} = \max_{\sigma}\ppsprob{\pps{}}{\sigma} $
where $ \parta $ is a poly partition of $ \pps{} $ and $ R $ is a
representative set of $ \parta $.
\end{replemma}

\begin{proof} Let $ \parta $ be the poly partition of $ \pps{}
= \set{\getpoly{i}} $ as in the statement of the lemma. Let us first
show a claim: if $ \sigma, \sigma' \in L \in \parta $ then
\begin{equation} \label{lem:ppp:lp:claim1}
A \defeq \set{\getpoly{} \in \pps{} \given \sigma \in \pconc{\getpoly{}}}
= \set{\getpoly{} \in \pps{} \given \sigma' \in \pconc{\getpoly{}}} \defeq B
\end{equation}

Let $ \getpoly{} \in A $. Thus $ \sigma \in \pconc{\getpoly{}} $ so
by \dref{def:ppp:pp} (2), we have $ \sigma \in \pconc{L'} $ for some
$ L' \in \parta $. By (1) it must be that $ L = L' $ and by (3), we have
$\pconc{L} = \pconc{L'} \subseteq \pconc{\getpoly{}} $. Therefore
$ \sigma' \in \pconc{\getpoly{}} $ and thus $ \getpoly{} \in B $,
showing $ A \subseteq B $. The other direction is identical,
concluding $ A = B $ as claimed. \qed{}.

Now we can get back to the main lemma. Let $ \sigma^* $ be the state
with $ \ppsprob{\pps{}}{\sigma^*}
= \max_{\sigma} \ppsprob{\pps{}}{\sigma} $. Thus
$ \sigma^* \in \pconc{L} $ for some $ L \in \parta $, by \dref{def:ppp:pp}
(2). Let $ \sigma_L $ be any representative of $ L $, that is
$ \sigma_L \in \pconc{L} $.
\vspace{15mm} % !!! formatting
\begin{align*}
\ppsprob{\pps{}}{\sigma^*} & = \sum_{i} \ppprob{\pp{i}}{\sigma^*} \\
                 &
                 = \sum_{i \given \sigma^* \in \pconc{\getpoly{i}}} \pmax{i} \\
                 &
                 = \sum_{i \given \sigma_L \in \pconc{\getpoly{i}}} \pmax{i}
                 & \mathstep{by \cnumref{lem:ppp:lp:claim1}} \\
                 & = \sum_{i} \ppprob{\pp{i}}{\sigma_L} \\
                 & = \ppsprob{\pps{}}{\sigma_L}
\end{align*}

Now we see that  $ \max_{\sigma}\ppsprob{\pps{}}{\sigma}
= \ppsprob{\pps{}}{\sigma^*} = \ppsprob{\pps{}}{\sigma_L} = \ppsmax{\pps{}} $ as claimed.
\end{proof}

Before we prove the security theorem, let us show that the definition
of abstract conditioning on a state is sound.

\begin{lemma} \label{lem:ppp:scond}
If $ \delta \in \ppconc{\pps{}} $ and $ \sigma_V \in \states_{V} $
with $ V \subseteq \fv{\delta} $ then
$ \dcond{\delta}{\sigma_V} \in \ppsconc{\absdcond{\pps{}}{\sigma_V}} $
\end{lemma}

\begin{proof} Recall the definition of $ \absdcond{\pps{}}{\sigma_V} $.

$$ \absdcond{\pps{}}{\sigma_V} = \absdcond{\pps{}}{\bexp} $$

With $ \bexp = \bigwedge_{x \in V} \paren{x = \sigma_V(x)} $. Let us
show that $ \dcond{\delta}{\sigma_V} = \dcond{\delta}{\bexp} $, the
rest will follow from \lref{lem:ppp:cond}.

The definition of $ \dcond{\delta}{\sigma_V} $ is as follows.
$$ \dcond{\delta}{\sigma}
= \lambda \sigma \lsep \aif \project{\sigma}{V} = \sigma_V
\athen \delta(\sigma) \aelse 0 $$

Meanwhile, $ \dcond{\delta}{\bexp} $ is defined as follows.
$$ \dcond{\delta}{\bexp}
= \lambda \sigma \lsep \aif \eval{\bexp}{\sigma}
= \atrue \athen \delta(\sigma) \aelse 0 $$

The correspondence is immediate as $ \eval{\bexp}{\sigma} = \atrue $
if and only if $ \project{\sigma}{V} = \sigma_V $ as per construction
of $ \bexp $.
\end{proof}

\begin{reptheorem}{thm:ppp:secure}[Soundness for Threshold Security]
  Let $\delta$ be an attacker's initial belief.  If $\delta \in
  \ppsconc{\pps{}}$ and $\abspolicy{S}{\pps{}}{t}$, then $ S $ is threshold secure
  for threshold $t$ when evaluated with initial belief $\delta$.
\end{reptheorem}

\begin{proof} Let us consider the contrapositive. That is, assuming
  $ \delta \in \ppsconc{\pps{}} $, if $ S $ is not threshold secure for
  $ t $ and initial belief $ \delta $, then it is not the case that
  $ \abspolicy{S}{\pps{}}{t} $.

Let $ \delta_2 = \eval{S}{\delta} $ and $ \delta_3
= \project{\delta_2}{L} $. Since $ S $ is not secure, we have
$ \sigma_L \in \support{\delta_3} $ and
$ \sigma'_H \in \states_H $ with $
(\normal{\project{\paren{\dcond{\delta_2}{\sigma_L}}}{H}})(\sigma'_H) >
t $. This implies that
$ \project{\paren{\dcond{\delta_2}{\sigma_L}}}{H} \neq \distzero $ and
therefore $ \dcond{\delta_2}{\sigma_L} \neq \distzero $ as projection
preserves mass.

If $ \pevalp{S}{\pps{}} $ is not terminating, then we are done as
termination is a condition for $ \abspolicy{S}{\pps{}}{t} $. So let us
assume $ \abspevalp{S}{\pps{}} $ is terminating. Let $ \pps{2}
= \abspevalp{S}{\pps{}} $. By \tref{thm:ppp:soundness}, we have
$ \delta_2 \in \ppsconc{\pps{2}} $. By \lref{lem:ppp:scond},
$ \dcond{\delta_2}{\sigma_L} \in \ppsconc{\absdcond{\pps{2}}{\sigma_L}}
$. Therefore not $ \iszero{\absdcond{\pps{2}}{\sigma_L}}
$. Continuing, by \lref{lem:ppp:project},
$ \project{ \paren{\dcond{\delta_2}{\sigma_L}}}{H} \in \ppsconc{\project{\paren{\absdcond{\pps{2}}{\sigma_L}}}{H}}
$ and finally, by \lref{lem:ppp:norm}, we have
$ \delta_4 \defeq \normal{\project{\paren{\dcond{\delta_2}{\sigma_L}}}{H}} \in \ppsconc{\normal{\project{\paren{\absdcond{\pps{2}}{\sigma_L}}}{H}}}
$. Let $ \pps{4}
= \normal{\project{\paren{\absdcond{\pps{2}}{\sigma_L}}}{H}} $.

By \rref{rem:ppp:prob}, we have
$ \delta_4(\sigma'_H) \leq \max_{\sigma}\ppsprob{\pps{4}}{\sigma} $
and by \lref{lem:ppp:pp} we have
$ \max_{\sigma}\ppsprob{\pps{4}}{\sigma} = \ppsmax{\pps{4}} $.  But
$ \delta_4(\sigma'_H) > t $ so $ \ppsmax{\pps{4}} > t $, a potential
failure of $\abspolicy{S}{\pps{}}{t} $.

To finish the proof we need to make sure that $ \sigma_L $ was indeed
a valid witness to the failure of $ \abspolicy{S}{\pp{1}}{t} $. Let
$ \pps{3} = \set{\pp{i}''} = \project{\pps{2}}{L}
$. By \lref{lem:ppp:project}, we have $ \delta_3 \in \ppsconc{\pps{3}}
$ so $ \delta_3 = \sum_i \delta'_i $ with
$ \delta'_i \in \ppconc{\pp{i}''} $. Since
$ \sigma_L \in \support{\delta_3} $ it must be that
$ \delta_3(\sigma_L) > 0 $ and thus $ \delta'_i(\sigma_L) > 0 $ for at
least one $ i $. Thus
$ \sigma_L \in \support{\delta'_i} \subseteq \pconc{\poly''_i} $ for
at least one $ i $ and therefore
$ \sigma_L \in \psconc{\set{\poly''_i}} $. Also, we have already
shown that not $ \iszero{\absdcond{\pps{2}}{\sigma_L}} $, thus
$ \sigma_L $ is indeed the witness as needed.
\end{proof}
%\end{DIFnomarkup}

