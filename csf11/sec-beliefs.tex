\section{Tracking beliefs}
\label{sec:belief}

This section reviews Clarkson et al.'s method of revising a querier's
belief of the possible valuations of secret variables based on the
result of a query involving those
variables~\cite{clarkson09quantifying}.

\begin{figure}[t]
\[
\begin{array}{llcl}
\mathit{Variables} & x & \in & \vars \\ 
\mathit{Integers} & n & \in & \Integer \\
\mathit{Rationals} & q & \in & \Rational \\
\mathit{Arith. ops} & \arithop &::= & + \mid \times \mid - \\
\mathit{Rel. ops} & \relop &::= & \leq \;\mid\; < \;\mid\; = \;\mid\; \neq \;\mid\; \cdots
\\
\mathit{Arith. exps} & \aexp &::= & x \mid n \mid \aop{\aexp_1}{\aexp_2} \\
\mathit{Bool. exps} & \bexp &::= & \bop{\aexp_1}{\aexp_2} \mid \\
% \strue \mid \sfalse \mid \\
& && \bexp_1 \wedge \bexp_2 \mid \bexp_1 \vee \bexp_2 \mid \aneg{\bexp} \\

\mathit{Statements} & \stmt &::= & \sskip \mid \sassign{x}{\aexp} \mid \\
&     && \sif{\bexp}{\stmt_1}{\stmt_2} \mid \\
&     && \spif{q}{\stmt_1}{\stmt_2} \mid \\
&     && \sseq{\stmt_1}{\stmt_2} \mid \swhile{\bexp}{\stmt} % \mid \\
% &     && \suniform{x}{n_1}{n_2} \\
\end{array}
\]
\caption{Core language syntax}
\label{fig:syntax}
\end{figure}

\subsection{Core language}

The programming language we use for queries is given in
Figure~\ref{fig:syntax}.  A computation is defined by a statement
$\stmt$ whose standard semantics can be viewed as a relation between
states: given an input state $\sigma$, running the program will
produce an output state $\sigma'$.  States are maps from variables to
integers:
$$\begin{array}{l}
\sigma, \tau \in \states \defeq \vars \rightarrow \Integer
\end{array}$$
Sometimes we consider states with domains restricted to a
subset of variables $V$, in which case we write $\sigma_V \in \states_V \defeq V
\rightarrow \Integer$.  We may also \emph{project}
states to a set of variables $V$:
\[\project{\sigma}{V} \defeq \lambda x \in \vars_V \lsep \sigma(x)\]

The language is essentially standard. 
\ifacita
We limit the form of expressions to support our abstract
interpretation-based semantics (see the TR version of this paper \cite{TR}).
\else
We limit the form of expressions to support our abstract
interpretation-based semantics (Section~\ref{sec:absinterp}).
\fi
The semantics of the statement form $\spif{q}{\stmt_1}{\stmt_2}$ is
non-deterministic: the result is that of $\stmt_1$ with probability
$q$, and $\stmt_2$ with probability $1 - q$.
% We omit a formal semantics as it is straightforward.

\subsection{Probabilistic semantics for tracking beliefs}
\label{sec:clarkson-semantics}

To enforce a knowledge-based policy, an agent must be able to
estimate what a querier could learn from the output of his query.  To
do this, the agent keeps a distribution $\delta$ that represents
the querier's \emph{belief} of the likely valuations of the user's
secrets.  A distribution is a map from states to positive real
numbers, interpreted as probabilities (in range $[0,1]$).
$$\begin{array}{l}
\delta \in \dists \defeq \states \rightarrow \Real+
\end{array}$$
We sometimes focus our attention on distributions over
states of a fixed set of variables $V$, in which case we write
$\delta_V \in \dists_V$ to mean $\states_V \rightarrow \Real+$.
Projecting distributions onto a set of variables is as
follows:\footnote{The notation $\sum_{x \mid \pi} \rho$ can be read
  \emph{$\rho$ is the sum over all $x$ such that formula $\pi$ is
    satisfied} (where $x$ is bound in $\rho$ and $\pi$).}
\[\project{\delta}{V} \defeq \lambda \sigma_V \in \states_V \lsep \sum_{\sigma' \mid (\project{\sigma'}{V} = \sigma_V)} \delta(\sigma')\]
The \emph{mass} of a distribution, written $ \pmass{\delta} $ is the sum
of the probabilities ascribed to states, $ \sum_{\sigma}
\delta(\sigma) $.  A \emph{normalized distribution} is one such that $
\pmass{\delta} = 1 $.  A normalized distribution can 
be constructed by scaling a distribution according to its
mass:
$$ \normal{\delta} \defeq \frac{1}{\pmass{\delta}} \cdot \delta $$
The \emph{support} of a distribution is the set of states which have
non-zero probability: $\nzset{\delta} \defeq \{\sigma \mid
\delta(\sigma) > 0\}$.

% We take as our basis the denotational semantics for reasoning about belief
% presented by Clarkson et al. \cite{clarkson09quantifying}.


\begin{figure}
%{\small
\centering
\begin{displaymath}
\begin{array}{rcl}
\pevalp{\sskip}{\delta} & = & \delta \\
%
\pevalp{\sassign{x}{\aexp}}{\delta} & = & \delta \bparen{x \ra \aexp} \\
%
\pevalp{\sif{B}{\stmt_1}{\stmt_2}}{\delta} & = &
\pevalp{\stmt_1}{(\dcond{\delta}{B})} + \pevalp{\stmt_2}{(\dcond{\delta}{\neg B})} \\
%
\evalp{\spif{q}{\stmt_1}{\stmt_2}}{\delta} & = & 
\evalp{\stmt_1}{(q \cdot\delta)} + \evalp{\stmt_2}{((1-q) \cdot \delta)} \\
%
\pevalp{\sseq{\stmt_1}{\stmt_2}}{\delta} & = & \pevalp{\stmt_2}{\paren{\evalp{\stmt_1}{\delta}}} \\
\pevalp{\swhile{\bexp}{\stmt}}{} & = & \lfp\left[\lambda
f :\ \dists
\rightarrow \dists \lsep \lambda \delta \lsep \right. \\
& & \left. \quad f\paren{\pevalp{\stmt}{(\dcond{\delta}{B})}} +
       \paren{\dcond{\delta}{\neg B}}\right]
\end{array} 
\end{displaymath} 
where
\begin{displaymath}
\begin{array}{l@{\;\defeq\;}l}
\delta \bparen{x \ra \aexp} & \lambda \sigma \lsep \sum_{\tau \; | \; \tau
  \bparen{x \ra \eeval{\aexp}{\tau}} = \sigma} \delta (\tau) \\
\delta_1 + \delta_2 & \lambda \sigma \lsep \delta_1(\sigma) +
\delta_2(\sigma) \\
\dcond{\delta}{\bexp} & \lambda \sigma \lsep \aif \eeval{\bexp}{\sigma} \athen
\delta(\sigma) \aelse 0 \\
p \cdot \delta & \lambda \sigma \lsep p \cdot \delta(\sigma) \\
\end{array}
\end{displaymath}
%}
\vspace*{-.1in}
\caption{Probabilistic semantics for the core language}
\label{fig-sem-nondet2-core}
\end{figure}

The agent evaluates a query in light of the querier's initial belief using a
probabilistic semantics.  Figure~\ref{fig-sem-nondet2-core} defines a
semantic function $\pevalp{\cdot}{}$ whereby $\pevalp{\stmt}{\delta} =
\delta'$ indicates that, given an input distribution $\delta$, the semantics
of program $\stmt$ is the output distribution $\delta'$.  The semantics is
defined in terms of operations on distributions, including \emph{assignment}
$\delta \bparen{v \ra E}$ (used in the rule for $v := E$),
\emph{conditioning} $\dcond{\delta}{B}$ and \emph{addition} $\delta_1 +
\delta_2$ (used in the rule for $\sifk$), and \emph{scaling} $q \cdot \delta$
where $q$ is a rational (used for $\spifk$).  The semantics is
standard (cf. Clarkson et al.~\cite{clarkson09quantifying}).
\iffull
A brief review is given in Appendix~\ref{appendix:concrete}.
\fi

\subsection{Belief and security}
\label{sec:experiment}

Clarkson et al.~\cite{clarkson09quantifying} describe how a belief
about possible values of a secret, expressed as a probability
distribution, can be revised according to an experiment using the
actual secret.  Such an experiment works as follows.

The values of the set of secret variables $ H $ are given by the hidden
state $\sigma_H$.  The attacker's initial belief as to the possible
values of $\sigma_H$ is represented as a distribution $ \delta_H $.  A
query is a program $ S $ that makes use of variables $ H $ and
possibly other, non-secret variables from a set $L$; the final values
of $L$, after running $S$, are made visible to the attacker.  Let
$\sigma_L$ be an arbitrary initial state of these variables such that
$\dom{\sigma_L} = L$.  Then we take the following steps:

\myparagraph{Step 1} Evaluate $S$ probabilistically using the
querier's belief about the secret to produce an output distribution
$\delta'$, which amounts to the attacker's prediction of the possible output
states.  This is computed as $\delta' = \eval{S}{\delta} $, where
$\delta$, a distribution over variables $ H \uplus L $, is defined
as $ \delta = \delta_H \times \dot{\sigma}_L $.  We make use of
the distribution product operator and point operator.  
That is, given $
\delta_1 $, $ \delta_2 $, which are distributions over states having
disjoint domains, the \emph{distribution product} is
 $$ \delta_1 \times \delta_2 \defeq \lambda(\sigma_1, \sigma_2)
 \lsep \delta_1(\sigma_1) \cdot \delta_2(\sigma_2) $$ where
 $(\sigma_1,\sigma_2)$ is the ``concatenation''
 of the two states, which is itself a state and is
 well-defined because the two states' domains are disjoint.
% \sbmcomment{I wouldn't use the term ``composition'' here.  Maybe ``union''?}
% \sbmcomment{Here is an alternate approach: define projection earlier, and then define product as below.}
%  $$ \delta_1 \times \delta_2 \defeq \lambda \sigma
%  \lsep \delta_1(\project{\sigma}{\dom{\delta_1}}) \cdot \delta_2(\project{\sigma}{\dom{\delta_2}}) $$
% \mwh{This isn't quite right: the domain of $\delta$ is a set of
%   states, but we want to project on a set of variables, i.e., the
%   domain of the domain.  This seems heavier than what's here now.}
And, given a
 state $ \sigma $, the \emph{point distribution} $ \dot{\sigma} $ is a
 distribution in which only $ \sigma $ is possible:
$$ \dot{\sigma} \defeq \lambda \tau \lsep \aif \sigma = \tau \athen 1 \aelse
0 $$ Thus, the initial distribution $\delta$ is the attacker's belief about the
secret variables combined with an arbitrary valuation of the
public variables.

\myparagraph{Step 2} Using the actual
secret $\sigma_H$, evaluate $S$ ``concretely'' to produce an output
state $\hat{\sigma}_L$, in three steps. 
First, we have $\hat{\delta}' = \eval{S}{\hat{\delta}}$, where
$\hat{\delta} = \dot{\sigma}_H \times \dot{\sigma}_L $.  
% Notice here
% that we use the actual secret in constructing the initial
% distribution, rather than the attacker's belief about it. 
Second, we
have $\hat{\sigma} \in \Gamma(\hat{\delta})$ where $\Gamma$ is a sampling
operator that produces a state $\sigma$ from the domain of a
distribution $\delta$ with probability $\delta(\sigma) /
\pmass{\delta}$. Finally, we extract the attacker-visible output of
the sampled state by projecting away the high variables:
$\hat{\sigma}_L = \project{\hat{\sigma}}{L}$.  
% The projection operator
% is defined as $\project{\sigma}{V} \defeq \lambda x \in
% \vars_V \lsep \sigma(x)$.

\myparagraph{Step 3} Revise the attacker's initial belief $\delta_H$
according to the observed output $\hat{\sigma}_L$, yielding a new
belief $\hat{\delta}_H =
\project{\dcond{\delta'}{\hat{\sigma}_L}}{H}$.  Here, $\delta'$ is
\emph{conditioned} on the output $\hat{\sigma}_L$, which yields a new
distribution, and this distribution is then projected to the variables
$H$.  The conditioning is defined as follows:
$$ \dcond{\delta}{\sigma_V} \defeq \lambda \sigma
\lsep \aif \project{\sigma}{V} = \sigma_V \athen \delta (\sigma) \aelse 0 $$
% Second, we have $\hat{\delta}_H =
% \project{\hat{\delta}}{H}$.
%   The latter projection is similar to
% projection on states, given above, but applied to distributions, as
% follows:
% \[
% \project{\delta}{V} \defeq \lambda \sigma_V \in \states_V \lsep \sum_{\sigma' \mid (\project{\sigma'}{V} = \sigma_v)} \delta(\sigma')
% \]
% It allows the querier to
% revise his belief about the secret variables based on the output
% resulting from a query of those secrets.  

\ifacita
\else
Note that this protocol assumes that $S$ always terminates and does not
modify the secret state.  The latter assumption can be eliminated by
essentially making a copy of the state before running the program, while
eliminating the former depends on the observer's ability to detect
nontermination~\cite{clarkson09quantifying}.
\fi
% We say more about
% nontermination as it relates to our approach in the next section.

% A distribution can be projected to a set of variables $ V $:
% $$ \project{\delta}{V}
% = \lambda \sigma_V \lsep \sum_{\sigma : \project{\sigma}{V}
% = \sigma_V} \delta(\sigma) $$
% The projection of a state to $ V $, written $ \project{\sigma}{V} $ as
% above, removes all variables except those of $ V $ from the
% state.

% Given $ \delta $, and a set of
%   states $ S $, we have $\dcond{\delta}{S}$, the revision of $ \delta $,
% The exact behavior is specified as follows:

% $$ \delta | S = \lambda \sigma \lsep \aif \sigma \in S \athen
%   \delta(\sigma) \aelse 0 $$

%   Revision will also be performed based on logical expressions. Having
%   a means of determining whether a logical expression $ B $ is true on a
%   state $ \sigma $, written $ \eeval{B}{\sigma} $, we define revision:
